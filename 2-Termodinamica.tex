\chapter{Termodinámica de cuerdas}

\section{Cálculo de la densidad de estados para una cuerda excitada}

La fórmula de masas para una cuerda abierta es
\begin{equation}
  \alpha' m^2=-1+N
\end{equation}
y para una cuerda cerrada
\begin{equation}
  \alpha' m^2=-4+2(N+\bar N).
\end{equation}

\todo{Explicar}
Queremos calcular la densidad de estados de una cuerda en función del número de 
oscilador, 

\begin{equation}
  N=\sum_{m=1}^N m n_m, \qquad n_m\in\mathbb N
\end{equation}

Las particiones de $N$
\todo{Demostracion particiones de N}
\begin{equation}
  p(N)\approx \alpha N^ {\frac{d-1}{2}}\exp\qty(2\pi\sqrt{\frac{N(d-2)}{6}}).
\end{equation}

En el caso de cuerdas bosónicas cerradas, expresando la densidad de estados en función de la masa
\begin{equation}
  p(M)\approx \alpha N^ {\frac{d-1}{2}}\exp\qty(2\pi\sqrt{\frac{N(d-2)}{6}}).
\end{equation}


\begin{equation}
  \omega(E) \approx V \frac{e^{\beta_H E}}{E^{D/2+1}}.
\end{equation}

La función de partición en la colectividad macrocanónica es
\begin{equation}
  z(\beta)=\int^\infty_{E_0} dE \frac{e^{\beta_H E}}{{E^{D/2+1}}}e^{-\beta E}
\end{equation}

\begin{equation}
  \begin{aligned}
    z&=(\beta-\beta_H)^{D/2}\ln(\beta-\beta_H), &\text{si $D$ par}\\
    z&=(\beta-\beta_H)^{D/2}, &\text{si $D$ impar}.
  \end{aligned}
\end{equation}

\section{Modelo random walk}

\todo{Revisar planteamiento}
Intuitivamente, la masa (energía) de una cuerda es su densidad lineal $\sigma$ por su longitud $L$.
En unidades naturales, la masa tiene unidades inversas a la longitud, por lo que si pensamos
que la cuerda está dividida en $N$ segmentos de longitud $l_s$, 
\begin{equation}
  E\sim  \frac{N}{l_s} \sim \frac{L}{l_s^2}.
\end{equation}

Cada segmento podrá tener una orientación arbitraria. Para simplificar los cálculos,
supongamos que toma direcciones ortogonales. Entonces el número de orientaciones posibles
de cada segmento es $2n$, donde $n$ es el número de dimensiones espaciales accesibles.
El número de segmentos es $L/l_s$.
Por tanto, el número de microestados para una energía dada es
\begin{equation}
  \omega(E)=(2n)^{(L/l_s)}\propto e^{l_s E \ln (n)}.
\end{equation}

La entropía de la cuerda es $S\propto El_s$, con temperatura $\beta_H\sim l_S$.

Se puede calcular que el radio del random walk es $R=\expval{r}\sim E^{1/2}$. Esto
nos dice que cuerdas muy largas estarán contenidas en un volumen pequeño $V\sim E^{D/2}$.

Para una cuerda cerrada, habría que dividir el número de microestados por el 
volumen de walk, multiplicar por el volumen accesible y dividir por la longitud de la cuerda.
Entonces,
\begin{equation}
  \omega(E)\sim V \frac{e^{\beta_H E}}{E^{1+D/2}}.
\end{equation}

\todo{No necesitamos conocer los detalles de la teoría}
De esta forma, obtenemos la densidad de estados 

El cálculo de la extensión de una cuerda altamente excitada con respecto a su centro de masas 
conduce a que $R^2_i$ es proporcional a la suma de las inversas de los números oscilatorios para
la dirección $i$.
Por ello, para un nivel energético $N$ fijo, $(\alpha^i_{-1})^N\ket{0}$ tendría la mayor extensión en la dirección $i$, $\sim N$ y
$\alpha^i_N\ket{0}$ tendría la menor extensión.
Tomando el promedio sobre todas las direcciones espaciales, $R^2\sim (d-1)\sqrt{N}$, siendo
$d-1$ el número de direcciones espaciales. Volvemos a  observa la importante peculariadad $R\sim \sqrt{L}$,
por lo que las cuerdas largas tiende a enmarañarse.

\section{Coalescencia multicuerda}

\todo{Repaso física estadística}

\subsection{Formulación microcanónica}

La función de partición de una cuerda es
\begin{equation}
  z(\beta) = \int_0^\infty \omega(E)e^{-\beta E}.
\end{equation}

La función de partición de un sistema de cuerdas es
\begin{equation}
  Z(\beta) = \int_0^\infty \Omega(E)e^{-\beta E}.
\end{equation}

Si el número de 


\begin{equation}
  \begin{aligned}
      Z = e^z&= 1+ \int_0^\infty dE_1\omega(E_1)e^{-\beta E_1} +\frac 1 2 \int_0^\infty \int_0^\infty
      \omega(E_1)\omega(E_2)e^{-\beta(E_1+E_2)}\dots \\
      &=\int_0^\infty dE \delta(E) + \int_0^\infty dE e^{-\beta E}\int  dE_1\omega(E_1)\delta(E-E_1)\\
      &+\frac 1 2\int_0^\infty dE e^{-\beta E} \int_0^\infty\int_0^\infty dE_1 dE_2 \omega(E_1)\omega(E_2) \delta(E-E_1-E_2)\dots
  \end{aligned}
\end{equation}

Entonces
\begin{equation}
  \Omega(E)=\sum_{n=0}^\infty \frac{1}{n!}\int \prod_{i=1}^n dE_i \omega(E_i)\delta\qty(E-\sum_{j=1}^n E_j).
\end{equation}

\begin{equation}
  \omega(E)\sim \frac{e^{\beta_H E}}{E^{D/2+1}}.
\end{equation}

\todo{Acabar deducción}

\subsection{Formulación canónica}

\section{Termodinámica de cuerdas en espacio plano}

%Termodinámica de cuerdas. Energía libre. Invariancia modular explícita. Escalar termal.

La función de partición bosónica en un espectro discreto es $z=\prod_i 1/(1-\exp(-\beta E_i))$.
La energía libre viene dada por $F=-1/\beta \ln z=\beta\sum_i \ln(1-\exp(-\beta E_i))$.
En el caso continuo, donde $E=E(k)$, reemplazamos el sumatorio por $V\int d^{d-1}k/(2\pi)^{d-1}$.
Por tanto,
\begin{equation}
  F=\frac{V}{\beta}\int \frac{d^{d-1}k}{(2\pi)^{d-1}}\ln(1-\exp(-\beta E))
\end{equation}

Mediante el desarrollo de Taylor $\ln(1-x)=-\sum^\infty_{r=1} x^r/r$
\begin{equation}
  F=-\frac{V}{\beta}\sum^\infty_{r=1} \frac 1 r\int \frac{d^{d-1}k}{(2\pi)^{d-1}}\exp(-r\beta E)
\end{equation}

Aplicando la identidad
\begin{equation}
  \frac 1 r \exp(-\beta r E)=\frac{\beta}{\sqrt{2\pi}}\int_0^\infty \frac{ds}{s^{3/2}}\exp\qty(-\frac{E^2s}{2}-\frac{r^2\beta^2}{2s})
\end{equation}

y como $E^2=k^2+m^2$

\begin{equation}
  F=-\frac{V}{\sqrt{2\pi}}\sum^\infty_{r=1} \int_0^\infty \frac{ds}{s^{3/2}} \exp \qty(-\frac{m^2s}{2} -\frac{r^2\beta^2}{2s})
  \int \frac{d^{d-1}k}{(2\pi)^{d-1}}\exp\qty(-\frac{k^2s}{2})
\end{equation}

La última integral tiene como resultado $(2\pi s)^{\frac{d-1}{2}}$, por lo que
\begin{equation}
  F=-V \int_0^\infty \frac{ds}{s(2\pi s)^{d/2}}\sum^\infty_{r=1} \exp \qty(-\frac{m^2s}{2}- \frac{r^2\beta^2}{2s})
\end{equation}

Por otro lado, la función de partición de estado multicuerda es $Z=e^z=-\beta F$. 
Entonces podemos hacer la asociación de la energía libre con integral de camino de 
una partícula tras aplicar una rotación de Wick en la dimensión temporal y compactificando el
tiempo con periodo $\beta$
\begin{equation}
  Z=-\beta F= \int_0^\infty \frac{ds}{2s}\sum^{'\infty}_{w=-\infty} \int \mathcal{D}x 
  \exp\qty(-\frac 1 2 \int_0^s d\tau \qty[\qty(\pdv{X^\mu}{t})^2+m^2])
\end{equation}

Se llega al mismo resultado calculando la función de partición de una partícula  si identificamos 
el parámetro $s$ con el tiempo propio  de la partícula y compactificamos 
la coordenada $X^0$ con periodo $\beta$. El número de vueltas que da la
partícula en la dimensión temporal es $w$. La función de partición se suma sobre
todas las vueltas posibles y se integra en todos los tiempos propios.

En una teoría de cuerdas bosónica, $d=26$ y la energía libre se obtendría sumando las
energías libres para todo el espectro posible, con masas
\begin{equation}
  m^2=\frac{2}{\alpha'}(N+\bar N-2)
\end{equation}

Puesto que se tiene que cumplir el level-matching $N=\bar N$, habrá que incluir en la 
integral la expresión de la delta de Kronecker
\begin{equation}
   \delta_{N\bar N}=\int_{-1/2}^{1/2}d\tau_1 \exp(2\pi i\tau_1 (N-\bar N))
\end{equation}

Entonces
\begin{equation}
  F=\sum_i \delta_{N_i \bar N_i} F(N_i,\bar N_i)
\end{equation}

donde $i$ recorre todas las posibles combinaciones de $N_i$ y $\bar N_i$.
Haciendo el cambio de variable $s=2\pi\alpha'\tau_2$
\begin{equation}
  F=-V \int_0^\infty \frac{d\tau_2}{\tau_2(4\pi^2\alpha'\tau_2)^{d/2}}\frac 1 2\sum_{r=-\infty}^{'\infty} 
  \sum_i e^{-(N_i+\bar N_i -2)2\pi\tau_2} e^{-\frac{r^2\beta^2}{4\pi\alpha'\tau_2}}\int_{-1/2}^{1/2} d\tau_1 e^{2\pi i\tau_1(N_i-\bar N_i)}
\end{equation}

Definiendo $\tau=\tau_1+i\tau_2$ y $q=e^{2\pi i\tau_1}$, se introduce la $\eta$ de 
Dedekind como
\begin{equation}
  \eta(\tau)=q^{1/24}\prod_{n=1}^{\infty} (1-q^n)
\end{equation}

Se llega a
\begin{equation}
  F=-V \sum_{r=-\infty}^{'\infty} \int_0^\infty\frac{d\tau_2}{2\tau_2} \int_{-1/2}^{1/2} d\tau_1  \frac{1}{(4\pi^2\alpha'\tau_2)^{d/2}}
  \abs{\eta(\tau)}^{-2d+4}\exp\qty(-\frac{r^2\beta^2}{4\pi\alpha'\tau_2})
\end{equation}

El mismo resultado se obtiene calculando la integral de camino en el toro
\begin{equation}
  Z= \int_0^\infty\frac{d\tau_2}{2\tau_2} \int_{-1/2}^{1/2} d\tau_1 \Delta_{FP} \int \mathcal Dx
  \exp\qty(-\frac{1}{4\pi\alpha'}\int d^2\sigma \sqrt h h^{\alpha\beta}\partial_\alpha X^\mu \partial_\beta X_\mu)
\end{equation}

donde la región de 

Se ha llegado a una expresión cuyo integrando no es modular invariante. Es decir, 
$\tau\to\tau+1$ y $\tau\to1/\tau$ no son invariancias.
Para conseguir la invariancia modular se restringe el dominio de integración al dominio
fundamental y se modifica el integrado, añadiendo el número cuántico $w$ sobre el que 
se suma, de modo que para $d=26$ se tiene
\begin{equation}
  F=-V \sum_{r,w=-\infty}^{\infty} \int_\mathcal{F}  \frac{d\tau_1d\tau_2}{2\tau_2}   \frac{1}{(4\pi^2\alpha'\tau_2)^{13}}
  \abs{\eta(\tau)}^{-48}\exp\qty(-\frac{\abs{r^2-w\tau}^2\beta^2}{4\pi\alpha'\tau_2})
\end{equation}

Haciendo la suma, se llega a
\begin{equation}
  Z=-\beta F=\int_{\mathcal F}\frac{d\tau_1d\tau_2}{2\tau_2}\Tr\qty(q^{L_0}\bar q^{\bar L_0})
\end{equation}

donde
\begin{align}
  L_0&=\alpha'\frac{p^2}{4}+\frac{\alpha'}{4}\qty(\frac{2\pi n}{\beta}+\frac{w\beta}{2\pi\alpha'})^2+N\\ 
 \bar L_0&=\alpha'\frac{p^2}{4}+\frac{\alpha'}{4}\qty(\frac{2\pi n}{\beta}-\frac{w\beta}{2\pi\alpha'})^2+\bar N
\end{align}


%\section{Cuerdas en background estático}
%Supongamos una cuerda en presencia de un background $G_{\mu\nu}$ independiente del tiempo.
%La función de partición calculada como la integral de camino en el toro es
%\begin{equation}
%  Z= \int_0^\infty\frac{d\tau_2}{2\tau_2} \int_{-1/2}^{1/2} d\tau_1 \Delta_{FP} \int \mathcal Dx \sqrt G
%  \exp\qty(-\frac{1}{4\pi\alpha'}\int d^2\sigma \sqrt h h^{\alpha\beta}\partial_\alpha X^\mu \partial_\beta X^\nu G_{\mu\nu})
%\end{equation}
%
%La métrica de la worldsheet

\section{Discrepancia entre colectividad microcanónica y canónica}

\todo{Comentar más allá de Hagedorn? No perturbativo, interacciones, transición de fase\ldots}
