\chapter{Termodinámica de cuerdas}

En este capítulo estudiaremos la termodinámica de cuerdas a altas energías.
Deduciremos que la densidad de estados de una cuerda muy energética tiene un comportamiento
exponencial, lo que conduce a la divergencia de la función de partición a la temperatura de 
Hagedorn.
Para entender mejor el comportamiento cerca de la temperatura de Hagedorn, razonaremos 
que las cuerdas tienden a formar una sola cuerda muy larga.
La energía libre de un gas de cuerdas relacionarlo con la integral de camino una cuerda
en un tiempo compacto.
Esta interpretación será la que nos permita hacer una generalización a espacios curvos.

\section{Cálculo de la densidad de estados para una cuerda excitada}

Queremos calcular cuántos estados distintos puede tomar una cuerda para una energía dada.
La fórmula de masas \ref{eq:massdef} para una cuerda bosónica cerrada en $d=26$ es 
\begin{equation}
  \alpha' m^2=-4+2(N+\bar N).
\end{equation}

La masa de un estado depende de los valores que tomen los números de oscilador $N$ y $\bar N$, que además
han de cumplir el level-matching $N=\bar N$.
Las distintas excitaciones de una cuerda se obtienen aplicando los operadores
de creación $\alpha^i_{-n}$ y $\bar \alpha^i_{-n}$ sucesivamente sobre el estado fundamental (en este
capítulo el acento circunflejo de los operadores queda implícito).
Como para un mismo número de oscilador, la cuerda puede haber sido excitada en distintas direcciones
$i$ y hay aplicando distintas combinaciones de operadores de creación, hay degeneración energética.

Tomemos como ejemplo el primer estado excitado, $N=1$, que se forma mediante $\bar \alpha^i_{-1} \alpha^j_{-1} \ket{0;p}$.
Como $i$ y $j$ toman $d-2$ valores independientes (al trabajar en la gauge del cono de luz desaparecen dos grados de libertad)
hay $(d-2)^2=576$ estados posibles.

Para $N=2$, los estados posibles son del tipo $\bar \alpha^i_{-2} \alpha^j_{-2}\ket{p;0}$, 
$\bar \alpha^i_{-1} \bar \alpha^j_{-1} \alpha^k_{-1} \alpha^l_{-1} \ket{p;0}$,
$\bar \alpha^i_{-2} \alpha^j_{-1} \alpha^k_{-1}\ket{p;0}$ y $\bar \alpha^i_{-1} \bar \alpha^j_{-1} \alpha^k_{-2} \ket{p;0}$.
En total, hay $(d-2)^2 + (d-2)^2(d-1) + \frac{1}{4}(d-2)^2(d-1)^2= 104976$ estados, muchos más que para $N=1$.
Para simplificar los cálculos, podemos obtener el número de estados asociados $\alpha^i_n$,
sin preocuparnos de $\bar \alpha^i_n$ y luego tomar el cuadrado.

El número de estados para un valor de $N$ (sin tener en cuenta las combinaciones de $\bar \alpha^i_n$) se denomina la particiones de $N$ 
y se denota por $p(N)$.
El cálculo preciso de $p(N)$ es muy complicado, pero podemos obtener una aproximación para
valores grandes de $N$ mediante la fórmula de Hardy-Ramanujan
\begin{equation}
  p(N)\propto \alpha N^ {\frac{d-1}{2}}\exp\qty(2\pi\sqrt{\frac{N(d-2)}{6}}).
\end{equation}
Como se intuía, el crecimiento de el número de estados es muy rápido.

Si $N$ es muy grande, se puede aproximar la masa de la cuerda por un espectro continuo y entonces se define la
densidad de estados $\rho(M)$ de forma que $\int_0^M dm \rho(m)$ proporciona el número total
de estados con masa $M$.

En el caso de cuerdas bosónicas cerradas, la densidad de estados en función de la masa es
\begin{equation}
  \rho(M)\approx \alpha N^ {\frac{d-1}{2}}\exp\qty(2\pi\sqrt{\frac{N(d-2)}{6}}).
\end{equation}

Ahora buscamos la densidad de estados en función de la energía, generalizando el problema a $d-D-1$ dimensiones compactas 
de radio $R$.
La energía de la cuerda es
\begin{equation}
  \alpha E^2 = \alpha' \mathbf p ^2 + \alpha'\frac{n^2}{R^2} + \frac{w^2 R^2}{\alpha'} -4+ 2(N+\bar N).
\end{equation}

Puesto que la densidad de estados en función de $N$ crece exponencialmente, mientras 
que la densidad de estados en función de $\abs{\mathbf p}$ crece como una potencia,
a altas energías $\abs{\mathbf p} \ll E$.
También podemos despreciar la contribución de las dimensiones compactas a la energía.
Bajo estas aproximaciones, la densidad de estados es
\begin{equation}
  \omega(E) \approx V \frac{e^{\beta_H E}}{E^{D/2+1}},
\end{equation}
donde 
\begin{equation}
  \beta_H = 2\pi \sqrt{\alpha'}\sqrt{\frac{d-2}{6}}.
\end{equation}

%Vemos que la densidad de estados no depende del número de dimensiones compactas.
La temperatura $T_H=1/(k_B \beta_H)$ se denomina temperatura de Hagedorn.

Si la cuerda está en contacto con una fuente de calor a la temperatura inversa $\beta$, la función de partición asociada es la transformada de Laplace de la densidad de estados,
\begin{equation}
  z(\beta)=\int^\infty_{E_0} dE \frac{e^{\beta_H E}}{{E^{D/2+1}}}e^{-\beta E},
\end{equation}
donde se integra desde $E_0$ porque la expresión de la densidad de estados solo es válida a altas
energías.
La integral diverge si $\beta < \beta_H$, por lo que $T_H$ aparece como una temperatura límite
a partir de la cual la función de partición deja de tener sentido.

La contribución no analítica dominante de la función de partición a alta temperatura se aproxima
por 
\begin{equation}
  \begin{aligned}
    z&\propto(\beta-\beta_H)^{D/2}\ln(\beta-\beta_H), &\text{si $D$ par,}\\
    z&\propto(\beta-\beta_H)^{D/2}, &\text{si $D$ impar}.
  \end{aligned}
  \label{eq:zp}
\end{equation}

\todo{QCD}
La temperatura de Hagedorn también aparece al 
%\cite{}

\section{Modelo del random walk}

Una forma sencilla pero heurística de obtener la densidad de estados 
de una cuerda es mediante el modelo del \emph{random walk}.
Consideremos de momento una cuerda abierta, 
cuya energía viene dada por su densidad lineal $\sigma$ multiplicada por su longitud $L$.
Si la cuerda está dividida en $N$ segmentos de longitud $l_s$ con masa asociada $m_s$, 
\begin{equation}
  E = \sigma L  =  \frac{m_s}{l_s} L.
\end{equation}
En unidades naturales, la masa tiene unidades inversas a la longitud, por lo que identificamos $m_s\sim l_s^{-1}$.
Entonces,
\begin{equation}
  E \sim \frac{L}{l_s^2}.
\end{equation}

Cada segmento podrá tener una orientación arbitraria. 
Para simplificar los cálculos,supongamos que solo toman direcciones ortogonales entre sí.
Entonces el número de orientaciones posibles de cada segmento es $2n$, donde $n$ es el número de dimensiones espaciales accesibles.
El número de segmentos es $L/l_s$.
Por tanto, el número de configuraciones (microestados) de una cuerda a energía fija es
\begin{equation}
  \omega(E)=(2n)^{(L/l_s)}\sim e^{l_s E \ln (2n)}.
\end{equation}

%Se puede calcular que el radio del random walk es $R=\expval{r}\sim E^{1/2}$. Esto
%nos dice que cuerdas muy largas estarán contenidas en un volumen pequeño $V\sim E^{D/2}$.

Para una cuerda cerrada, habría que dividir el número de microestados entre el 
volumen que ocupa, multiplicar por el volumen accesible y dividir entre su energía.
De esta forma, recuperamos la densidad de estados vista en la sección anterior,
\begin{equation}
  \omega(E)\sim V \frac{e^{\beta_H E}}{E^{1+D/2}}.
\end{equation}

%Sin embargo, este procedimiento no permite obtener el valor concreto de $\beta_H$.

\section{Coalescencia multicuerda}

Tras haber estudiado una solo cuerda, pasamos a considerar un gas de cuerdas.
Las cuerdas pueden interaccionar entre sí, uniéndose o dividiéndose, por lo que
no tiene sentido fijar el número de cuerdas. 
Como no cuesta ninguna energía crear una cuerda, el potencial químico es nulo, $\mu=0$.
Veremos dos formas alternativas de concluir que al aumentar la temperatura, las cuerdas
tienden a unirse en una sola cuerda:
comparando las densidades de estados y las funciones de partición para una sola 
cuerda y para un gas de cuerdas, 

\subsection{Densidad de estados}

La función de partición de una cuerda se relaciona con la densidad de estados de una cuerda $\omega(E)$, mediante
una transformada de Laplace
\begin{equation}
  z(\beta) = \int_0^\infty \omega(E)e^{-\beta E}.
\end{equation}

Del mismo modo, la función de partición de un sistema de cuerdas se obtiene a partir de la
densidad de estados de un gas de cuerdas $\Omega(E)$, como
\begin{equation}
  Z(\beta) = \int_0^\infty \Omega(E)e^{-\beta E}.
  \label{eq:z1}
\end{equation}

Aplicando la aproximación de Maxwell-Boltzmann $Z=e^z$ y haciendo un desarrollo de Taylor,
\begin{equation}
  \begin{aligned}
      Z = e^z&= 1+ \int_0^\infty dE_1\omega(E_1)e^{-\beta E_1} +\frac 1 2 \int_0^\infty \int_0^\infty
      \omega(E_1)\omega(E_2)e^{-\beta(E_1+E_2)}\dots \\
      &=\int_0^\infty dE \delta(E) + \int_0^\infty dE e^{-\beta E}\int_0^\infty  dE_1\omega(E_1)\delta(E-E_1)\\
      &+\frac 1 2\int_0^\infty dE e^{-\beta E} \int_0^\infty\int_0^\infty dE_1 dE_2 \omega(E_1)\omega(E_2) \delta(E-E_1-E_2)\dots
  \end{aligned}
  \label{eq:z2}
\end{equation}
Donde hemos aplicado la propiedad de la delta de Dirac $\int_0^\infty dx f(x)\delta(x-x_0) = f(x_0)$.
Comparando las ecuaciones \ref{eq:z1} y \ref{eq:z2},
\begin{equation}
  \Omega(E)=\sum_{n=0}^\infty \frac{1}{n!}\int_0^\infty \prod_{i=1}^n dE_i \omega(E_i)\delta\qty(E-\sum_{j=1}^n E_j).
\end{equation}

Procedemos a sustituir la densidad de estados para una sola cuerda,
\begin{equation}
  \omega(E)\sim \frac{e^{\beta_H E}}{E^{D/2+1}}.
  \label{eq:dd}
\end{equation}

Puesto que la densidad de estados \ref{eq:dd} solo es válida para altas energías, acotamos inferiormente
las integrales por $E_0$.
Suponiendo además que para la variable de integración $E_1$ se cumple $E=E_1$ e ignorando
la contribución del vacío ($n=0$), pues solo es relevante a $E=0$, la densidad de estados es
\begin{equation}
  \begin{aligned}
    &\Omega(E) \sim \frac{e^{\beta_H E}}{E^{D/2+1}} \sum_{n=1}^\infty \frac{1}{(n-1)!}\qty(\frac{2}{DE_0})^{n-1},\quad D>0,\\
    &\Omega(E) \sim \frac{e^{\beta_H E}}{E} \sum_{n=1}^\infty \frac{1}{(n-1)!}\qty(\ln \frac{E}{E_0})^{n-1},\quad D=0.\\
  \end{aligned}
\end{equation}
La suma da lugar a una exponencial,
\begin{equation}
  \begin{aligned}
    &\Omega(E) \sim \frac{e^{\beta_H E}}{E^{D/2+1}}e^{\frac{2}{DE_0}}, \quad D>0\\
    &\Omega(E) \sim \frac{e^{\beta_H E}}{E_0}, \quad D=0.
  \end{aligned}
\end{equation}
Comparando la densidad de estados de un gas de cuerdas con la densidad de estados de una sola cuerda, vemos que 
ambas coinciden si $D>0$.
Esto quiere decir que a altas energías, predomina una sola cuerda muy larga.

\subsection{Función de partición}

Otra forma de estudiar el comportamiento termodinámico es examinando la falta de analiticidad de la función de partición.
La función de partición de una sola cuerda es (\ref{eq:zp}),
\begin{equation}
  \begin{aligned}
    z&=(\beta-\beta_H)^{D/2}\ln(\beta-\beta_H), &\text{si $D$ par}\\
    z&=(\beta-\beta_H)^{D/2}, &\text{si $D$ impar}.
  \end{aligned}
\end{equation}

Aplicando la aproximación Maxwell-Boltzmann $Z=e^z$,
\begin{equation}
  \begin{aligned}
   Z(\beta)= &1-C(\beta-\beta_H)^{D/2} +\frac{C^2}{2}(\beta-\beta_H)^D-\frac{C^3}{6}(\beta-\beta_H)^{3D/2}+\cdots, \text{ si $D$ impar}\\
   Z(\beta)= &1-C(\beta-\beta_H)^{D/2}\ln(\beta-\beta_H) +\frac{C^2}{2}(\beta-\beta_H)^D\ln(\beta-\beta_H) ^2\\
         &-\frac{C^3}{6}(\beta-\beta_H)^{3D/2}\ln(\beta-\beta_H)^3 +\cdots, \quad \text{si $D\neq 0$ par},\\
   Z(\beta)=& \frac{1}{\beta-\beta_H}, \quad \text{si $D=0$}.
  \end{aligned}
\end{equation}

Ignorando el término constante, la contribución principal a la no analiticidad de la función de partición coincide
con la asociada a una sola cuerda, salvo si $D=0$.


\section{Termodinámica de cuerdas en espacio plano}

\label{sec:free}

En esta sección calcularemos la energía libre de un gas de cuerdas en el espacio de Minkowski,
con el fin de generalizar la temperatura de Hagedorn a espacios curvos.

La función de partición bosónica para un solo grado de libertad con un espectro discreto es $z=\prod_i 1/(1-\exp(-\beta E_i))$.
La energía libre se obtiene a partir de la función de partición como $F=-1/\beta \ln z=1/\beta\sum_i \ln(1-\exp(-\beta E_i))$.
Si el espectro es continuo, con $E=E(k)$, reemplazamos el sumatorio por la integral $V\int d^{d-1}k/(2\pi)^{d-1}$.
Por tanto,
\begin{equation}
  F=\frac{V}{\beta}\int \frac{d^{d-1}k}{(2\pi)^{d-1}}\ln(1-\exp(-\beta E)).
\end{equation}

Expandiendo el logaritmo por el desarrollo de Taylor $\ln(1-x)=-\sum^\infty_{r=1} x^r/r$,
\begin{equation}
  F=-\frac{V}{\beta}\sum^\infty_{r=1} \frac 1 r\int \frac{d^{d-1}k}{(2\pi)^{d-1}}\exp(-r\beta E).
\end{equation}

Aplicando la identidad
\begin{equation}
  \frac 1 r \exp(-\beta r E)=\frac{\beta}{\sqrt{2\pi}}\int_0^\infty \frac{ds}{s^{3/2}}\exp\qty(-\frac{E^2s}{2}-\frac{r^2\beta^2}{2s}),
\end{equation}

y como $E^2=k^2+m^2$, se llega a 
\begin{equation}
  F=-\frac{V}{\sqrt{2\pi}}\sum^\infty_{r=1} \int_0^\infty \frac{ds}{s^{3/2}} \exp \qty(-\frac{m^2s}{2} -\frac{r^2\beta^2}{2s})
  \int \frac{d^{d-1}k}{(2\pi)^{d-1}}\exp\qty(-\frac{k^2s}{2}).
\end{equation}

La integral en momentos tiene como resultado $(2\pi s)^{\frac{d-1}{2}}$, por lo que
\begin{equation}
  F=-V \int_0^\infty \frac{ds}{s(2\pi s)^{d/2}}\sum^\infty_{r=1} \exp \qty(-\frac{m^2s}{2}- \frac{r^2\beta^2}{2s}).
\end{equation}

%Por otro lado, la función de partición de estado multicuerda es $Z=e^z=e^{-\beta F}$. 
%Entonces podemos hacer la asociación de la energía libre con integral de camino de 
%una partícula tras aplicar una rotación de Wick en la dimensión temporal y compactificando el
%tiempo con periodo $\beta$
%\begin{equation}
%  Z=-\beta F= \int_0^\infty \frac{ds}{2s}\sum^{'\infty}_{w=-\infty} \int \mathcal{D}x 
%  \exp\qty(-\frac 1 2 \int_0^s d\tau \qty[\qty(\pdv{X^\mu}{t})^2+m^2])
%\end{equation}

%Se llega al mismo resultado calculando la función de partición de una partícula si identificamos 
%el parámetro $s$ con el tiempo propio  de la partícula y compactificamos 
%la coordenada $X^0$ con periodo $\beta$. El número de vueltas que da la
%partícula en la dimensión temporal es $w$. La función de partición se suma sobre
%todas las vueltas posibles y se integra en todos los tiempos propios.

En una teoría de cuerdas bosónica, para la cual $d=26$, la energía libre se obtendría sumando las
energías libres para todo el espectro posible, con la fórmula de masas
\begin{equation}
  m^2=\frac{2}{\alpha'}(N+\bar N-2).
\end{equation}

La energía libre sería
\begin{equation}
  F=\sum_i \delta_{N_i \bar N_i} F(N_i,\bar N_i),
\end{equation}
donde la suma recorre todos los estados posibles de la cuerda, incluyendo aquellos con $N_i\neq \bar N_i$.
La delta de Kronecker impone la condición $N_i=\bar N_i$. 
Empleando la forma integral de la delta de Kronecker
\begin{equation}
   \delta_{N\bar N}=\int_{-1/2}^{1/2}d\tau_1 \exp(2\pi i\tau_1 (N-\bar N)),
\end{equation}
y haciendo el cambio de variable $s=2\pi\alpha'\tau_2$,
\begin{equation}
  F=-V \int_0^\infty \frac{d\tau_2}{\tau_2(4\pi^2\alpha'\tau_2)^{d/2}}\frac 1 2\sum_{r=-\infty}^{'\infty} 
  \sum_i e^{-(N_i+\bar N_i -2)2\pi\tau_2} e^{-\frac{r^2\beta^2}{4\pi\alpha'\tau_2}}\int_{-1/2}^{1/2} d\tau_1 e^{2\pi i\tau_1(N_i-\bar N_i)}.
\end{equation}

La prima indica que se ha omitido el término con $r=0$ del sumatorio, pues es la contribución
de vacío, que no depende de la temperatura.

Definiendo $\tau=\tau_1+i\tau_2$ y $q=e^{2\pi i\tau_1}$, introducimos la $\eta$ de 
Dedekind como
\begin{equation}
  \eta(\tau)=q^{1/24}\prod_{n=1}^{\infty} (1-q^n).
\end{equation}
Sustituyendo, llegamos a 
\begin{equation}
  F=-V \sum_{r=-\infty}^{'\infty} \int_0^\infty\frac{d\tau_2}{2\tau_2} \int_{-1/2}^{1/2} d\tau_1  \frac{1}{(4\pi^2\alpha'\tau_2)^{d/2}}
  \abs{\eta(\tau)}^{-2d+4}\exp\qty(-\frac{r^2\beta^2}{4\pi\alpha'\tau_2}).
  \label{eq:strip}
\end{equation}

Para valores próximos a $\tau_2=0$, los términos con $r=\pm 1$ proporcionan la divergencia 
dominante.
El comportamiento de la exponencial en \ref{eq:strip} es
\begin{equation}
  \exp\qty(\frac{16\pi^2 \alpha' -\beta^2}{4\pi\alpha'\tau_2}),
\end{equation}
por lo que al integrar $\tau_2$ desde cero, la energía libre diverge si $\beta\leq \beta_H=4\pi\sqrt{\alpha'}$.
Hay otra divergencia al integrar hasta infinito en $\tau_2$, pero esta se debe a la presencia 
del taquión en el estado fundamental, que desaparece al considerar cuerdas supersimétricas.

El mismo resultado se obtiene calculando la integral de camino para una worldsheet toroidal,
\begin{equation}
  Z= \int_0^\infty\frac{d\tau_2}{2\tau_2} \int_{-1/2}^{1/2} d\tau_1 \Delta_{FP} \int \mathcal Dx
  \exp\qty(-\frac{1}{4\pi\alpha'}\int d^2\sigma \sqrt h h^{\alpha\beta}\partial_\alpha X^\mu \partial_\beta X_\mu),
\end{equation}
considerando cuerdas que se enrollan $r$ veces alrededor de una dimensión temporal de periodo $\beta$
y aplicando $F=-Z/\beta$.
La importancia de este resultado es que permite calcular la energía libre en espacio curvo.

\subsection{Formulación invariante modular}
\label{sec:taq}
\todo{Dibujo}
Se ha llegado a una expresión cuyo integrando no es invariante modular. Es decir, las
transformaciones modulares $\tau\to\tau+1$ y $\tau\to-1/\tau$ en el integrando modifican el resultado
de la integral.
Cualquier punto del plano complejo puede ser llevado al dominio fundamental,
\begin{equation}
  \mathcal F = \qty{ \tau \in \mathbb{C} \mid \abs{\tau} \geq 1, \Re\tau \in [-1/2,1/2]}
\end{equation}
mediante composición de transformaciones modulares.

Para conseguir la invariancia modular, se restringe el dominio de integración al dominio
fundamental y se modifica el integrado, añadiendo el número cuántico $w$ sobre el que 
se suma, de modo que para $d=26$ se tiene
\begin{equation}
  F=-V \sum_{r,w=-\infty}^{\infty} \int_\mathcal{F}  \frac{d\tau_1d\tau_2}{2\tau_2}   \frac{1}{(4\pi^2\alpha'\tau_2)^{13}}
  \abs{\eta(\tau)}^{-48}\exp\qty(-\frac{\abs{r^2-w\tau}^2\beta^2}{4\pi\alpha'\tau_2}).
  \label{eq:fund}
\end{equation}

Esta expresión de la energía libre se puede calcular como $F=-Z/\beta$, siendo $Z$ la integral de camino del toro en
un espacio donde la dimensión temporal forma un círculo de radio $R=\beta/(2\pi)$ y con enrollamientos
de las cuerdas e integrando solo sobre el dominio fundamental.

Una forma equivalente de expresar la integral es tomando la traza sobre todos los estados posibles,
\begin{equation}
  F =-\frac{1}{\beta} \int_{\mathcal F} \frac{d\tau_1 d\tau_2}{2\tau_2} \Tr\qty(q^{L_0} \bar q^{\bar L_0}),
\end{equation}
donde 
\begin{equation}
  \begin{gathered}
         L_0 =  -\alpha'\frac{m^2}{4} + \frac{\alpha'}{4}\qty(\frac{2\pi n}{\beta} + \frac{w\beta}{2\pi\alpha'})^2 + N,\\
    \bar L_0 =  -\alpha'\frac{m^2}{4} + \frac{\alpha'}{4}\qty(\frac{2\pi n}{\beta} - \frac{w\beta}{2\pi\alpha'})^2 +\bar N.
  \end{gathered}
\end{equation}

Al haber excluido la región próxima a $\tau_2=0$ en la integral, la divergencia de Hagedorn aparece para
valores muy altos de $\tau_2$.
La divergencia se manifiesta al tomar la traza sobre el estado con $w=\pm 1$, $n=0$ y $N=\bar N=0$.
La masa de este estado es
\begin{equation}
  \alpha' \frac{m^2}{4} = \frac{\beta^2}{16\pi^2\alpha'} -1.
\end{equation}

A la temperatura de Hagedorn, el estado tiene masa nula y a temperaturas superiores, masa
imaginaria.

En resumen, si compactificamos el tiempo e identificamos el periodo temporal con el inverso de la
temperatura, podemos calcular la energía libre mediante una integral de camino restringida
al dominio fundamental.
El estado que se enrolla una vez en el tiempo es el responsable de la divergencia de Hagedorn y
a temperaturas por encima de la temperatura de Hagedorn se vuelve taquiónico.
Es importante notar que este es un estado efectivo, pues no aparece en el espectro de la cuerda en 
el espacio de Minkowski, sino que es consecuencia de compactificar el tiempo.
Sin embargo, este enfoque es muy útil para poder hacer una generalización a espacios curvos.
