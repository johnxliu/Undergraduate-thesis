\chapter{Cuerdas en espacios curvos}

Queremos ver cómo se modifica la temperatura de Hagedorn cerca de un agujero negro.

La temperatura de Hagedorn se extrae del comportamiento divergente de la energía libre.


Por tanto, necesitamos ver cómo calcular la energía libre en un espacio-tiempo curvo.
Veremos dos métodos distintos: mediante la integral de camino en el toro de una cuerda
que se enrolla en el tiempo de la variedad termal (tiempo euclídeo con periodo $\beta$) y
la integral de camino a un loop de un campo taquiónico.

\section{Path integral en la worldsheet}

La integral de camino del toro en una campo $G_{\mu\nu}$, el cual representa la curvatura
del espacio-tiempo, en $d$ dimensiones es
\begin{equation}
  Z=\int_0^\infty \frac{d\tau_2}{2\tau_2} \int_{-1/2}^{1/2} d\tau_1 \Delta_{FP} \int \mathcal DX
  \sqrt G e^{-S},
  \label{eq:toruspi}
\end{equation}
donde el determinante de Faddeev-Popov se introduce debido a la libertad gauge bajo difeomorfismos
y trasformaciones Weyl.
La acción es 
\begin{equation}
  S= \frac{1}{4\pi \alpha'}\int_{0}^{1} d^2\sigma \sqrt h h^{\alpha\beta} \partial_\alpha X^\mu\partial_\beta X^\nu G_{\mu\nu}.
\end{equation}
La métrica de la worldsheet es
\begin{equation}
  h_{\alpha\beta} =
\qty(
\begin{array}{cc}
  1 & \tau_1 \\
  \tau_1 & \tau_1^2+\tau_2^2
\end{array}
).
\end{equation}

Definimos las nuevas coordenadas $\sigma=\sigma_1/\tau_2$ y $\tau=\sigma_2$, de modo 
que la acción se convierte en 
\begin{equation}
  \begin{aligned}
    S= \frac{1}{4\pi \alpha'}\int_0^{1/\tau_2} d\sigma  \int_0^1 d\tau 
    &\biggl[
      \qty(1+\frac{\tau_1^2}{\tau^2_2})\partial_\sigma X^\mu \partial_\sigma X^\nu G_{\mu\nu} +
      2\frac{\tau_1}{\tau_2}\partial_\sigma X^\mu \partial_\tau X^\nu G_{\mu\nu}  \\
      &+ \partial_\tau X^\mu \partial_\tau X^\nu G_{\mu\nu}\biggl].
  \end{aligned}
\end{equation}

Haciendo el desarrollo de Fourier de $X^\mu(\sigma,\tau)$ en la coordenada $\sigma$, obtenemos
\begin{equation}
  \begin{aligned}
    X^i(\sigma,\tau) =& \sum_{n=-\infty}^\infty X_n^i(\tau) e^{in2\pi \tau_2 \sigma} \\
    X^0(\sigma,\tau) =& \pm \beta \tau_2 \sigma +  \sum_{n=-\infty}^\infty X_n^0(\tau) e^{in2\pi \tau_2 \sigma}.
  \end{aligned}
\end{equation}

Como la divergencia de Hagedorn se produce en $\tau_2\to\infty$, nos quedamos con el modo $n=0$.
%Esto significa que despreciamos las vibraciones de la cuerda ortogonales a 
Al integrar en $\tau_2$, obtenemos la acción de una partícula
\begin{equation}
  \begin{aligned}
    S=\frac{1}{4\pi\alpha'\tau_2}\int_0^1 d\tau \big[
      \beta^2 (\tau_1^2+\tau_2^2)G_{00} \pm 2\beta\tau_1 G_{0\mu}\partial_\tau X^\mu  +G_{\mu\nu} \partial_\tau X^\mu \partial_\tau X^\nu
    \big].
  \end{aligned}
\end{equation}

Depreciar el resto de modos en la serie de Fourier hace que a la acción resultante le falten
términos adicionales. 
En una métrica plana, la corrección a la acción es
\begin{equation}
  \Delta S = - \frac{\tau_2^2 \beta^2_{H0}}{4\pi\alpha' \tau_2}.
\end{equation}

Definiendo un nuevo parámetro $t=\tau_2 \tau$ e incluyendo la corrección a la acción
\begin{equation}
  \begin{aligned}
    S=\frac{1}{4\pi\alpha'}\biggl[ -\beta^2\tau_2 + \int_0^{\tau_2} dt \biggl\{
      \beta^2 \frac{\tau_1^2+\tau_2^2}{\tau_2^2}G_{00} \pm 2\beta\frac{\tau_1}{\tau_2} G_{0\mu}\partial_t X^\mu  +G_{\mu\nu} \partial_t X^\mu \partial_t X^\nu
    \biggl\}\biggl].
  \end{aligned}
\end{equation}

En una métrica estacionaria y que cumpla $G_{0i}=0$
\begin{equation}
  \begin{aligned}
    S=\frac{1}{4\pi\alpha'}\biggl[ -\beta^2\tau_2 + \int_0^{\tau_2} dt \biggl\{&
      \beta^2 \frac{\tau_1^2+\tau_2^2}{\tau_2^2}G_{00} \pm 2\beta\frac{\tau_1}{\tau_2} G_{00}\partial_t X^0  +G_{00} \qty(\partial_t X^0)^2 \\
    &+G_{ij} \partial_t X^i \partial_t X^j
    \biggl\}\biggl].
  \end{aligned}
\end{equation}

Si descomponemos la coordenada temporal en una parte clásica y una corrección adicional,
$X^0=X^{cl}+\tilde X^0$, podemos resolver la ecuación de movimiento para el término clásico y
así simplificar la acción.
La ecuación de Euler-Lagrange es 
\todo{Reemplazar indices}
\begin{equation}
  \partial_t\qty[G_{00}\partial_t X^0 \pm \beta \frac{\tau_1}{\tau_2}G_{00}]=0.
\end{equation}

Por lo que 
\begin{equation}
  G_{00}\partial_t X^0 \pm \beta \frac{\tau_1}{\tau_2}G_{00}=C.
\end{equation}

Como la coordenada $X^0$ es periódica,
\begin{equation}
  0=X^0(\tau_2) -X^0(0) = X^0(t)\eval_0^{\tau_2}= \int_0^{\tau_2}\partial_t X^0(t) = \ev{\partial_t X^0}
\end{equation}

Promediando,
\begin{equation}
  \ev{\partial_t X^0} \pm \ev{\beta \frac{\tau_1}{\tau_2}}=\ev{\frac{C}{G_{00}}} .
\end{equation}

Obtenemos
\begin{equation}
  C = \pm \beta \frac{\tau_1}{\ev{G_{00}^{-1}}}
\end{equation}

La acción clásica es
\begin{equation}
  S^{cl}=\frac{1}{4\pi\alpha'}\ev{G_{00}(\partial_t X^0)^2 \pm 2\beta \frac{\tau_1}{\tau_2}G_{00}\partial_t X^0}.
\end{equation}

Aplicando 
\begin{equation}
  \ev{G_{00}(\partial_t X^0)^2 \pm \beta \frac{\tau_1}{\tau_2}G_{00}\partial_t X^0} = C\ev{\partial_t X^0} = 0,
\end{equation}
y
\begin{equation}
  \ev{G_{00}}\partial_t X^0 = \tau_2 C\mp \frac{\tau_1}{\tau_2}\beta \ev{G_{00}},
\end{equation}
la acción clásica puede escribirse como
\begin{equation}
  S^{cl}= \frac{1}{4\pi\alpha'} \qty[\frac{\tau_1^2\beta^2}{\ev{G_{00}^{-1}}}-\beta^2\frac{\tau_1^2}{\tau_2^2}\ev{G_{00}}].
\end{equation}

La integral de camino en $\tilde X^0$ es
\begin{equation}
  Z_0=\frac{1}{\sqrt{4\pi^2\alpha'\ev{G_{00}^{-1}}}}\beta.
\end{equation}

La integral en $\tau_1$ en la ecuación \ref{eq:toruspi}, da lugar al término 
\begin{equation}
  \frac{2\pi\sqrt{\alpha'}\sqrt{\ev{G_{00}^{-1}}}}{\beta}.
\end{equation}

La integral de camino se reduce a 
\begin{equation}
  Z=2\int_0^\infty \frac{d\tau_2}{2\tau_2}\int \mathcal D \vec {X} \sqrt{\prod_t \det G_{ij}}e^{-Sp}
\end{equation}
donde la acción 
\begin{equation}
  S=\frac{1}{4\pi\alpha'}\qty[-\beta^2_{H0}\tau_2+\beta^2 \int_0^{\tau_2}dt \qty(G_{00}+G_{ij}\partial_t X^i\partial_t X^j)]
\end{equation}


\section{Campo efectivo}

La acción de un campo taquiónico a orden más bajo en $\alpha'$ es
\begin{equation}
  S=\frac{1}{2}\int d^d x\sqrt G e^{-2\Phi}  \qty(G^{\mu\nu}\partial_\mu T\partial_\nu T+m^2T^2),
\end{equation}
donde $T$ es un campo escalar y $\Phi$ el campo del dilatón.

Si el tiempo tiene periodo $2\pi R$, el deserrollo de Fourier de $T$ es
\begin{equation}
  T(x^0,x^i)=\sum_{n=-\infty}^{\infty} T_n(x^i)e^{\frac{inx^0}{R}}.
\end{equation}
En la variedad termal, $R=\beta/2\pi$. 
Si la métrica es estacionaria y $G_{i0}=0$,
\begin{equation}
  S=\pi R \int d^{d-1} \sqrt{G}e^{-2\Phi}\qty(G^{ij}\partial_i T_n \partial_j T_{-n} + \frac{k^2 G^{00}}{R^2}T_n T_{-n} + m^2 T_n T_{-n}).
\end{equation}
\todo{Demostrar}

Como $T$ es un campo real, $T_n=T^*_{-n}$ y por tanto
\begin{equation}
  S=\pi R \int d^{d-1} \sqrt{G}e^{-2\Phi}\qty(G^{ij}\partial_i T_n \partial_j T_{n}^* + \frac{k^2 G^{00}}{R^2}T_n T_{n}^* + m^2 T_n T_{n}^*).
\end{equation}

Aplicando la dualidad T y la transformación $R\to \alpha'/R$, los momentos se convierten en 
enrrollamientos
\begin{equation}
  S \sim \int d^{d-1} x \sqrt G e^{-2\Phi}\qty(G^{ij}\partial_i T_w \partial_j T_{w}^* + \frac{w^2 R^2 G_{00}}{{\alpha'}^2}T_w T_{w}^* + m^2 T_w T_{w}^*) .
\end{equation}

Vemos cómo aperece un término de masa efectiva $m^2_{loc} = m^2 +\frac{w^2 R^2 G_{00}}{{\alpha'}^2}$,
que depende de la posición espacio-temporal.
Dependiendo del tipo de teoría que se considere (bośonica, tipo II supersimétrica o heterótica), 
esta masa efectiva será distinta.

De ahora en adelante simplificamos la notación $T=T_1$. 
Integrando por partes la acción 
\todo{Demostrar}
\begin{equation}
  S = \int d^{d-1} x \sqrt G e^{-2\Phi} T^* \qty[-\nabla^2-G^{ij}\frac{\partial_j \sqrt{G_{00}}}{\sqrt G_{00}}\partial_i + m^2_{loc}]T,
\end{equation}
donde $\nabla^2 = G^{ij}\nabla_i \partial_j$.
Para calcular la integral de camino asociada a está acción tenemos que buscar los valores
propios del operador 
\begin{equation}
  \widehat{O} =-\nabla^2-G^{ij}\frac{\partial_j \sqrt{G_{00}}}{\sqrt G_{00}}\partial_i + m^2_{loc}.
\end{equation}
