\chapter{Cuerdas en espacios curvos}

Este capítulo se centra en el estudio de las cuerdas en la proximidad de un agujero negro.
Como  la temperatura de Hagedorn se extrae del comportamiento divergente de la energía libre, 
necesitamos encontrar una forma de calcular la energía libre en un espacio-tiempo curvo.
Veremos dos métodos distintos: mediante la integral de camino toroidal de una cuerda
que se enrolla en un tiempo periódico y la integral de camino de un campo taquiónico.
Aplicando este último método a cuerdas de tipo II cerca de un agujero negro, encontraremos que
la divergencia de Hagedorn aparece a la temperatura de Hawking.
Concluiremos con la justificación de la entropía de un agujero negro a partir de los resultados anteriores.

\section{Integral de camino en la worldsheet}

Partimos de suponer que la integral de camino toroidal de una cuerda cerrada, con
un tiempo compactificado de periodo $\beta$, se relaciona con la energía libre de un gas de cuerdas mediante
\begin{equation}
  Z =  -\beta F.
\end{equation}

Esta relación se ha demostrado exacta para ciertos casos concretos, pero no hay una demostración general.

La integral de camino toroidal en un campo $G_{\mu\nu}$, el cual representa la curvatura
del espacio-tiempo, en $d$ dimensiones es
\begin{equation}
  Z=\int_0^\infty \frac{d\tau_2}{2\tau_2} \int_{-1/2}^{1/2} d\tau_1 \Delta_{FP} \int \mathcal DX
  \sqrt G e^{-S},
  \label{eq:toruspi}
\end{equation}
donde el determinante de Faddeev-Popov se introduce debido a la libertad gauge bajo difeomorfismos
y trasformaciones de Weyl.
La acción es 
\begin{equation}
  S= \frac{1}{4\pi \alpha'}\int_{[0,1]^2} d^2\sigma \sqrt h h^{\alpha\beta} \partial_\alpha X^\mu\partial_\beta X^\nu G_{\mu\nu}.
\end{equation}
Escogemos la métrica de la worldsheet como
\begin{equation}
  h_{\alpha\beta} =
\qty(
\begin{array}{cc}
  1 & \tau_1 \\
  \tau_1 & \tau_1^2+\tau_2^2
\end{array}
).
\end{equation}

La divergencia de Hagedorn aparece calcular la integral de camino de una que se enrolla en el tiempo 
a lo largo de la coordenada  $\sigma_2$, pues diverge en el límite $\tau_2\to0$.
Las condiciones de periodicidad son
\begin{equation}
  \begin{aligned}
    &X^\mu(\sigma_1+1,\sigma_2)=X^\mu(\sigma_1,\sigma_2),\\
    &X^i(\sigma_1,\sigma_2+1)=X^i(\sigma_1,\sigma_2),\\
    &X^0(\sigma_1,\sigma_2+1) = X^0(\sigma_1,\sigma_2)\pm \beta.
  \end{aligned}
\end{equation}

Haciendo la transformación modular $\tau\to-1/\tau$ e intercambiando $\sigma_1$ y $\sigma_2$, las
condiciones de periodicidad son
\begin{equation}
  \begin{aligned}
    &X^\mu(\sigma_1,\sigma_2+1)=X^\mu(\sigma_1,\sigma_2),\\
    &X^i(\sigma_1+1,\sigma_2)=X^i(\sigma_1,\sigma_2),\\
    &X^0(\sigma_1+1,\sigma_2) = X^0(\sigma_1,\sigma_2)\pm \beta.
  \end{aligned}
\end{equation}

Definimos las nuevas coordenadas $\sigma=\sigma_1/\tau_2$ y $\tau=\sigma_2$, de modo 
que la acción se convierte en 
\begin{equation}
  \begin{aligned}
    S= \frac{1}{4\pi \alpha'}\int_0^{1/\tau_2} d\sigma  \int_0^1 d\tau 
    &\biggl[
      \qty(1+\frac{\tau_1^2}{\tau^2_2})\partial_\sigma X^\mu \partial_\sigma X^\nu G_{\mu\nu} +
      2\frac{\tau_1}{\tau_2}\partial_\sigma X^\mu \partial_\tau X^\nu G_{\mu\nu}  \\
      &+ \partial_\tau X^\mu \partial_\tau X^\nu G_{\mu\nu}\biggl].
  \end{aligned}
\end{equation}


Haciendo un desarrollo de Fourier de $X^\mu(\sigma,\tau)$ en la coordenada $\sigma$, obtenemos
\begin{equation}
  \begin{aligned}
    X^i(\sigma,\tau) =& \sum_{n=-\infty}^\infty X_n^i(\tau) e^{in2\pi \tau_2 \sigma}, \\
    X^0(\sigma,\tau) =& \pm \beta \tau_2 \sigma +  \sum_{n=-\infty}^\infty X_n^0(\tau) e^{in2\pi \tau_2 \sigma}.
    \label{eq:four}
  \end{aligned}
\end{equation}

Definimos $X^i(\tau)=X^i_0(\tau)$ y $X^0(\tau)=X_0^0(\tau)$.
En la sección \ref{sec:free}, vimos que la divergencia de Hagedorn se produce al integrar en la región $\tau_2\to\infty$.
Por tanto, nos quedamos con la contribución dominante, $n=0$.
Al integrar en $\tau_2$, obtenemos la acción de una partícula
\begin{equation}
  \begin{aligned}
    S=\frac{1}{4\pi\alpha'\tau_2}\int_0^1 d\tau \big[
      \beta^2 (\tau_1^2+\tau_2^2)G_{00} \pm 2\beta\tau_1 G_{0\mu}\partial_\tau X^\mu  +G_{\mu\nu} \partial_\tau X^\mu \partial_\tau X^\nu
    \big].
  \end{aligned}
\end{equation}

Despreciar el resto de modos en la serie de Fourier hace que a la acción resultante le falten
términos adicionales. 
En una métrica plana, la corrección a la acción es
\begin{equation}
  \Delta S = - \frac{\tau_2^2 \beta^2_{H0}}{4\pi\alpha' \tau_2},
\end{equation}

siendo $\beta_{H0}$ la temperatura de Hagedorn en espacio plano.
Incluyendo esta corrección y definiendo un nuevo parámetro $t=\tau_2 \tau$ 
\begin{equation}
  \begin{aligned}
    S=\frac{1}{4\pi\alpha'}\biggl[ -\beta_{H0}^2\tau_2 + \int_0^{\tau_2} dt \biggl\{
      \beta^2 \frac{\tau_1^2+\tau_2^2}{\tau_2^2}G_{00} \pm 2\beta\frac{\tau_1}{\tau_2} G_{0\mu}\partial_t X^\mu  +G_{\mu\nu} \partial_t X^\mu \partial_t X^\nu
    \biggl\}\biggl].
  \end{aligned}
\end{equation}

En una métrica estacionaria y que cumpla $G_{0i}=0$
\begin{equation}
  \begin{aligned}
    S=\frac{1}{4\pi\alpha'}\biggl[ -\beta_{H0}^2\tau_2 + \int_0^{\tau_2} dt \biggl\{&
      \beta^2 \frac{\tau_1^2+\tau_2^2}{\tau_2^2}G_{00} \pm 2\beta\frac{\tau_1}{\tau_2} G_{00}\partial_t X^0  +G_{00} \qty(\partial_t X^0)^2 \\
    &+G_{ij} \partial_t X^i \partial_t X^j
    \biggl\}\biggl].
  \end{aligned}
\end{equation}

Si descomponemos la coordenada temporal en una parte clásica y una corrección adicional,
$X^0=X^{c}+\tilde X^0$, podemos hacer la integral de camino sólo sobre $\tilde X^0$. 
Primero hemos de buscar la acción clásica.
La ecuación de Euler-Lagrange es 
\begin{equation}
  \partial_t\qty[G_{00}\partial_t X^c \pm \beta \frac{\tau_1}{\tau_2}G_{00}]=0.
\end{equation}

Por lo que 
\begin{equation}
  G_{00}\partial_t X^c \pm \beta \frac{\tau_1}{\tau_2}G_{00}=C.
\end{equation}

Como la coordenada $X^0$ es periódica,
\begin{equation}
  0=X^0(\tau_2) -X^0(0) = X^0(t)\eval_0^{\tau_2}= \int_0^{\tau_2}\partial_t X^0(t) = \ev{\partial_t X^0} 
\end{equation}

Promediando,
\begin{equation}
  \ev{\partial_t X^c} \pm \ev{\beta \frac{\tau_1}{\tau_2}}=\ev{\frac{C}{G_{00}}} .
\end{equation}

Obtenemos la constante
\begin{equation}
  C = \pm \beta \frac{\tau_1}{\ev{G_{00}^{-1}}}
\end{equation}

La acción clásica es
\begin{equation}
  S^{cl}=\frac{1}{4\pi\alpha'}\ev{G_{00}(\partial_t X^0)^2 \pm 2\beta \frac{\tau_1}{\tau_2}G_{00}\partial_t X^0}.
\end{equation}

Aplicando 
\begin{equation}
  \ev{G_{00}(\partial_t X^0)^2 \pm \beta \frac{\tau_1}{\tau_2}G_{00}\partial_t X^0} = C\ev{\partial_t X^0} = 0,
\end{equation}
y
\begin{equation}
  \ev{G_{00}}\partial_t X^0 = \tau_2 C\mp \frac{\tau_1}{\tau_2}\beta \ev{G_{00}},
\end{equation}
la acción clásica puede escribirse como
\begin{equation}
  S^{cl}= \frac{1}{4\pi\alpha'} \qty[\frac{\tau_1^2\beta^2}{\ev{G_{00}^{-1}}}-\beta^2\frac{\tau_1^2}{\tau_2^2}\ev{G_{00}}].
\end{equation}

Como la trayectoria clásica hace estacionaria la acción, los términos lineales en $\tilde X^0$
se anulan y la integral de camino en $\tilde X^0$, tras varios cálculos, es
\begin{equation}
  Z_0=\frac{1}{\sqrt{4\pi^2\alpha'\ev{G_{00}^{-1}}}}\beta.
\end{equation}

La integral en $\tau_1$ en la ecuación \ref{eq:toruspi}, da lugar al término 
\begin{equation}
  \frac{2\pi\sqrt{\alpha'}\sqrt{\ev{G_{00}^{-1}}}}{\beta}.
\end{equation}

La integral de camino se reduce a 
\begin{equation}
  Z=\int_0^\infty \frac{d\tau_2}{\tau_2}\int \mathcal D \vec {X} \sqrt{\prod_t \det G_{ij}}e^{-Sp}
  \label{eq:final}
\end{equation}
donde la acción es
\begin{equation}
  S_p=\frac{1}{4\pi\alpha'}\qty[\int_0^{\tau_2}dt \qty(-\beta^2_{H0}+\beta^2 G_{00}+G_{ij}\partial_t X^i\partial_t X^j)].
\end{equation}

Hemos reducido la función de partición de una cuerda a la integral de camino de
una partícula no relativista moviéndose en un espacio curvo.

Aunque este método parece una opción viable, haremos los cálculos con un campo efectivo, razonando
que ambos procedimientos deberían conducir a los mismos resultados.

\section{Campo efectivo}
En la sección \ref{sec:taq} vimos que en un espacio plano, la divergencia de Hagedorn se manifiesta
como una campo que se vuelve taquiónico a temperaturas por encima de la temperatura de
Hagedorn.
Por tanto, intentaremos hallar la energía libre a partir de la acción de un campo taquiónico
en espacio curvo.

La acción de un campo taquiónico a orden más bajo en $\alpha'$ es
\begin{equation}
  S=\frac{1}{2}\int d^d x\sqrt G e^{-2\Phi}  \qty(G^{\mu\nu}\partial_\mu T\partial_\nu T+m^2T^2),
\end{equation}
donde $T$ es un campo escalar real y $\Phi$ el campo del dilatón.

Si el tiempo tiene periodo $2\pi R=\beta$, el desarrollo de Fourier de $T$ es
\begin{equation}
  T(x^0,x^i)=\sum_{n=-\infty}^{\infty} T_n(x^i)e^{\frac{inx^0}{R}}.
\end{equation}
Hay que tener en cuenta que $T_n(x^i)$ es un campo escalar complejo
y $n$ es el momento discretizado en la dimensión compacta.
Suponemos una métrica estacionaria y $G_{i0}=0$.
Sustituyendo,
\begin{equation}
  \begin{aligned}
    S=\sum_{n,m} \int d^{d-1} \sqrt{G}e^{-2\Phi}\biggl(&-\frac{nm}{R^2}G^{00}\partial_0 T_n \partial_0 T_{m} 
    + G^{ij}\partial_i T_n \partial_j T_{m} \\
    &+ m^2 T_n T_{m}\biggl) \int_{-\pi R}^{\pi R} \frac{dx^0}{2}e^{i\frac{n+m}{R}x^0}.
  \end{aligned}
\end{equation}

La integral en $x^0$ es $\pi R \delta_{-n,m}$, por lo que
\begin{equation}
  S=\sum_n\pi R \int d^{d-1} \sqrt{G}e^{-2\Phi}\qty(G^{ij}\partial_i T_n \partial_j T_{-n} + \frac{n^2 G^{00}}{R^2}T_n T_{-n} + m^2 T_n T_{-n}).
\end{equation}

Como $T$ es un campo real, $T_n=T^*_{-n}$ y por tanto
\begin{equation}
  S=\sum_n \pi R \int d^{d-1} \sqrt{G}e^{-2\Phi}\qty(G^{ij}\partial_i T_n \partial_j T_{n}^* + \frac{n^2 G^{00}}{R^2}T_n T_{n}^* + m^2 T_n T_{n}^*).
\end{equation}

Aunque en la acción solo aparece el momento $n$ y no los enrollamientos $w$, la acción completa debería satisfacer la dualidad T, 
la cual apareció en la sección \ref{sec:dual}.
Aprovechando la dualidad T, las  transformaciones necesarias para intercambiar el momento discreto por enrollamientos son
\begin{equation}
  \begin{aligned}
    G_{00}&\to \frac{1}{G_{00}},\\
    \Phi&\to \Phi -\frac{1}{2}\ln G_{00},\\
    T_n&\to T_w,\\
    R &\to \frac{\alpha'}{R}.
  \end{aligned}
\end{equation}

La acción en términos de enrollamientos del campo taquiónico es
\begin{equation}
  S \sim\sum_w  \int d^{d-1} x \sqrt G e^{-2\Phi}\qty(G^{ij}\partial_i T_w \partial_j T_{w}^* + \qty(\frac{w^2 R^2 G_{00}}{{\alpha'}^2} + m^2) T_w T_{w}^*) .
\end{equation}

Vemos que multiplicando a $\abs{T_w}^2$, aparece un término de masa efectiva $m^2_{loc} = m^2 +\frac{w^2 R^2 G_{00}}{{\alpha'}^2}$
que depende de la posición espacio-temporal.
\footnote{Haciendo el cálculo para cuerdas heteróticas, hay una corrección a esta fórmula.}

La contribución dominante de vendrá dada por $w=\pm 1$, por lo que definimos $T=T_1$.
Ignorando el campo del dilatón,
\begin{equation}
  S  =  \int d^{d-1} x \sqrt{G_{00}}\sqrt{G_{ij}} \qty(G^{ij}\partial_i T \partial_j T^* +m^2_{local} TT^*) .
\end{equation}

Aquí $\sqrt{G_{ij}}$ indica la raíz cuadrada del determinante de la parte espacial de la métrica.
Intentaremos interpretar esta acción como la de una partícula que se mueve en un espacio-tiempo con 
curvatura solo en la parte espacial y no en la temporal, para poder compararla con la acción obtenida
en la sección anterior.
Con este fin, hemos de eliminar el factor $\sqrt{G_{00}}$ del término cinético $G^{ij}\partial_i T \partial_j T^*$.

Integrando por partes la acción,
\begin{equation}
  S = \int d^{d-1} x \sqrt G e^{-2\Phi} T^* \qty[-\nabla^2-G^{ij}\frac{\partial_j \sqrt{G_{00}}}{\sqrt G_{00}}\partial_i + m^2_{loc}]T,
\end{equation}
donde $\nabla^2 = G^{ij}\nabla_i \partial_j$.
La integral de camino asociada a está acción se calcula buscando los
propios del operador 
\begin{equation}
  \widehat{O} =-\nabla^2-G^{ij}\frac{\partial_j \sqrt{G_{00}}}{\sqrt G_{00}}\partial_i + m^2_{loc}.
\end{equation}

Definiendo el producto escalar
\begin{equation}
  \braket{\psi_1}{\psi_2} = \int d^{d-1} x \sqrt G \psi_1(x)^* \psi_2(x),
\end{equation}
el operador $\widehat O$ es hermítico y por tanto posee un conjunto ortonormal de funciones
propias $\psi_n$ con valores propios reales, $\lambda_n$.

La integral de camino da como resultado
\begin{equation}
  Z=\int \mathcal D T e^{-S} =\prod_n \frac{1}{\lambda_n} = \det(\widehat O)^{-1}.
\end{equation}

La contribución taquiónica a la energía libre de un gas de cuerdas es
\begin{equation}
  F=-\frac{1}{\beta}\ln Z. 
\end{equation}

Teniendo en cuenta que $\Tr \ln X = \ln \det X$, la energía libre es
\begin{equation}
  F=\frac{1}{\beta} \Tr \ln \widehat O.
\end{equation}

Aplicando la fórmula de Schwinger del logaritmo, 
\begin{equation}
  \ln a = -\int_0^\infty \frac{dT}{T}\qty(e^{-aT}-e^{-T}),
\end{equation}
y despreciando el término $e^{-T}$, la energía libre es
\begin{equation}
  F=-\frac{1}{\beta}\int_0^\infty \frac{dT}{T} \Tr e^{-T\widehat O}.
  \label{eq:fO}
\end{equation}

Sigue habiendo una contribución de $\sqrt{G_{00}}$ al término cinético, que aparece al tomar la traza.
Para solucionarlo, hacemos el cambio de base
\begin{equation}
  \ket{\phi_n}=G_{00}^{1/4}  \ket{\psi_n}.
\end{equation}

El producto escalar asociado es
\begin{equation}
  \braket{\phi_1}{\phi_2} = \int d^{d-1} x \sqrt G_{ij} \phi_1(x)^* \phi_2(x).
\end{equation}

Tras el cambio de base,
\begin{equation}
  F=-\frac{1}{\beta}\int_0^\infty \frac{dT}{T} \Tr e^{-T\widehat D},
\end{equation}

donde aparece el operador hermítico
\begin{equation}
  \widehat D = -\nabla^2  + m_{loc}^2- \frac{3}{16}\frac{G^{ij} \partial_i G_{00}\partial_j G_{00}}{G_{00}^2}
  +\frac{\nabla^2 G_{00}}{4G_{00}}.
\end{equation}
Al tomar la traza, ya no aparece $\sqrt{G_{00}}$.
Los últimos dos términos se corresponden con un potencial efectivo
\begin{equation}
  K(x) = -\frac{3}{16}\frac{G^{ij} \partial_i G_{00}\partial_j G_{00}}{G_{00}^2}
  +\frac{\nabla^2 G_{00}}{4G_{00}}.
\end{equation}

Ahora se puede interpretar el término $\Tr e^{-T\widehat{D}}$ como una integral de camino
de una partícula 
\begin{equation}
Z=  \int_{S_1} \mathcal D x \sqrt{G_{ij}} e^{-\int_0^T dt \qty(\frac{1}{4}G_{ij}\dot x^i\dot x^j+m_{loc}^2+K(x))}.
\end{equation}

En resumen, haciendo el cambio de variable en la integral $t\to\pi \alpha't$, la energía libre es
\begin{equation}
  F=-\frac{1}{\beta}\int_0^\infty \frac{dT}{T} \int_{S^1} \mathcal D x\sqrt G_{ij}  e^{ -\frac{1}{4\pi\alpha'}\int_0^T dt \qty(  
  G_{ij}\dot x^i \dot x^j  + 4\pi^2{\alpha'}^2 (m_{loc}^2+K(x)))},
\end{equation}
donde
\begin{equation}
  \begin{aligned}
    m_{loc}^2 &= -\frac{4}{\alpha'}+\frac{\beta^2G_{00}}{4\pi^2 \alpha'^2} \quad\text{para cuerdas bosónicas},\\
    m_{loc}^2 &= -\frac{2}{\alpha'}+\frac{\beta^2G_{00}}{4\pi^2 \alpha'^2} \quad\text{para cuerdas de tipo II},\\
    m_{loc}^2 &= -\frac{3}{\alpha'}+\frac{\pi^2}{\beta^2 G_{00}}+\frac{\beta^2G_{00}}{4\pi^2 \alpha'^2}\quad\text{para cuerdas heteróticas}.
  \end{aligned}
\end{equation}

Si comparamos con la energía libre que obtendríamos a partir de la acción de la cuerda obtenida
mediante \ref{eq:final},
\begin{equation}
  F=-\frac{1}{\beta} \int_0^\infty \frac{d\tau_2}{\tau_2} \int_{S^1} \mathcal D X\sqrt G_{ij}  e^{ -\frac{1}{4\pi\alpha'}\int_0^{\tau_2} dt \qty(  
  G_{ij}\dot X^i \dot X^j  + \beta^2 G_{00} - \beta^2_{H0})},
\end{equation}
ambas energías libres coinciden al sustituir el valor apropiado de la temperatura de Hagedorn
en espacio plano, $\beta_{H0}$ salvo por el término $K(x)$. 
Cabe esperar que este término apareciese al tener en cuenta todos los términos del desarrollo de 
Fourier \ref{eq:four}.

Además, podrían estudiarse posibles correcciones a la acción del campo taquiónico a orden superior en $\alpha'$.

\section{Cuerdas cerca de agujeros negros}

Aplicaremos ahora el cálculo de la energía libre a un gas de cuerdas en las proximidades
de un agujero negro de Schwarzschild. 
Tomaremos como ejemplo la teoría supersimétrica de tipo II, se puede seguir un procedimiento
similar para cuerdas heteróticas y bosónicas, pero presentan más sutilezas.
Partimos de la expresión de la energía libre deducida a partir del campo taquiónico \ref{eq:fO},
\begin{equation}
  F=-\frac{1}{\beta}\int \frac{dT}{T} \Tr e^{-T\qty(  -\nabla^2-G^{ij}\frac{\partial_j \sqrt{G_{00}}}{\sqrt G_{00}}\partial_i + m^2_{loc}) }.
\end{equation}

Cerca del agujero negro, es conveniente emplear las coordenadas de Rindler, introducidas en \ref{eq:ri}.
El laplaciano en coordenadas de Rindler es
\begin{equation}
  \nabla^2  = \partial_\rho^2 + \frac{1}{\rho}\partial_\rho.
\end{equation}

La masa efectiva para cuerdas de tipo II es 
\begin{equation}
  m_{loc}^2=-\frac{2}{\alpha'}+\frac{\beta^2G_{00}}{4\pi^2\alpha'^2}.
\end{equation}

Para calcular la traza buscamos los valores propios del operador en el exponente,
pues la traza de un operador es la suma de sus valores propios.
Las funciones propias regulares (descartamos las que divergen) son del tipo
\begin{equation}
  \psi_n(\rho)\propto e^{-\frac{\beta \rho^2}{4\pi\alpha'^{3/2}} }
  L_n\qty(\frac{\beta\rho^2}{2\pi\alpha'^{3/2}}),
\end{equation}
donde $L_n$ es un polinomio de Laguerre de grado $n\geq0$.
Sus correspondientes valores propios son
\begin{equation}
  \lambda_n=\frac{\beta-2\pi\sqrt{\alpha'}+2\beta n }{\pi\alpha'^{3/2}}.
\end{equation}

La contribución dominante a la energía libre viene dada por el modo más bajo, $n=0$.
Esto no basta para calcular la traza pues tenemos que tener en cuenta las dimensiones adicionales al espacio de Rindler. 
En una teoría supersimétrica, tenemos ocho dimensiones descritas por una variedad que tenemos que especificar.
Suponiendo dimensiones planas, la energía libre es
\begin{equation}
  F= -\frac{V_T}{\beta}\int_0^\infty \frac{dT}{T}\qty(\frac{1}{4\pi T})^4 e^{\frac{\beta-2\pi\sqrt{\alpha'}}{\pi\alpha'^{3/2}}T}.
\end{equation}

En el límite superior de $T$, la energía libre diverge si
\begin{equation}
  \beta \leq 2\pi\sqrt{\alpha'}=\beta_R.
\end{equation}

Concluimos que la temperatura de Hagedorn en un espacio de Rindler coincide con la temperatura
de Hawking.
Hemos supuesto cuerdas de tipo II y que las dimensiones adicionales son planas.
De haber escogido otras condiciones, la temperatura de Hagedorn podría recibir correcciones.

Identificando las funciones propias como funciones de onda de una cuerda, el estado fundamental
\begin{equation}
  \psi_0(\rho)\propto e^{-\frac{\beta \rho^2}{4\pi\alpha'^{3/2}} },
  \label{eq:func}
\end{equation}
representa una cuerda localizada a la distancia $l_s$ del horizonte de sucesos.
Intuitivamente, las cuerdas sufren una presión radial que compensa la atracción gravitatoria del agujero negro
Este resultado refuerza la idea de la existencia de un horizonte \emph{estirado}, comentada en 
\cite{Susskind1993}.

\section{Entropía de agujeros negros}

La igualdad entre la temperatura de Hagedorn y la temperatura Hawking permite justificar
la entropía Bekenstein-Hawking de un agujero negro,
\begin{equation}
  S = \frac{A}{4G},
\end{equation}
donde $A$ es el área del agujero negro.
Pero antes, debemos entender si tiene sentido la existencia del equilibro termodinámico alrededor
de un agujero negro. 
En teoría cuántica de campos, si colocamos un agujero en una fuente de calor y partículas, 
la materia que cae al agujero negro entra en equilibrio con la radiación de Hawking.
En cambio, si mantenemos aislado un sistema formado por materia en caída y un agujero
negro, nunca se alcanzará el equilibrio.
La materia se iría aproximando arbitrariamente cerca del horizonte, ocupando un volumen muy
pequeño, de modo que cerca del horizonte se puede almacenar una cantidad infinita de información.
Esta el la causa de la divergencia de la entropía de un agujero negro en teoría cuántica de campos.

En teoría de cuerdas, el problema se soluciona al considerar que las cuerdas se mantienen a
una cierta distancia del horizonte, como refleja \ref{eq:func}.
De esta forma, las cuerdas alcanzan el equilibrio y al ocupar un volumen extenso, la entropía
permanece finita.

Para deducir la entropía de un agujero negro, calculamos el incremento de entropía asociado 
a lanzar cuerdas a un agujero negro, suponiendo que la interacción de la materia añadida con la
ya presente es pequeña. 
La densidad de estados de una cuerda es
$\omega(E) = E^\alpha e^{\beta_H E}$.

La entropía, en función del número de microestados $\Omega(E)$ es 
\begin{equation}
  S(E) = k_B\ln\Omega(E).
\end{equation}

Como el número de microestados en el intervalo $(E,E+dE)$ es $\Omega(E) = \omega(E)\delta E$,
\begin{equation}
  S(E) = k_B \ln \omega(E) + k_B \ln \delta E.
\end{equation}

El segundo término de la suma es una constante arbitraria que podemos ignorar, luego
la entropía en función de la densidad de estados es
\begin{equation}
  S(E) = k_B \ln \omega(E).
\end{equation}

A altas energías ($E \gg k_B T_H$) solo es relevante la contribución exponencial y por 
tanto el incremento de entropía del agujero negro es
\begin{equation}
  \delta S = k_B  \frac{\delta \omega}{\omega} = k_B \beta_H \delta E
\end{equation}

Teniendo en cuenta que $\beta_H = \beta_{haw}$, en términos de la masa,
\begin{equation}
  \delta S =  8\pi GM\delta M
\end{equation}

El radio de un agujero negro de Schwarzschild es $R=2GM$, lo que conduce a la entropía
de Bekenstein-Hawking,
\begin{equation}
  S=\frac{A}{4G}.
\end{equation}

Una interpretación alternativa de este resultado es que para que las cuerdas proporcionen
la entropía de un agujero negro, se ha de cumplir la igualdad $\beta_H = \beta_{haw}$.
