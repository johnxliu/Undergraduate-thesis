\chapter{Cuerdas en espacios curvos}

Queremos ver cómo se modifica la temperatura de Hagedorn cerca de un agujero negro.

La temperatura de Hagedorn se extrae del comportamiento divergente de la energía libre.


Por tanto, necesitamos ver cómo calcular la energía libre en un espacio-tiempo curvo.
Veremos dos métodos distintos: mediante la integral de camino en el toro de una cuerda
que se enrolla en el tiempo de la variedad termal (tiempo euclídeo con periodo $\beta$) y
la integral de camino a un loop de un campo taquiónico.

\section{Path integral en la worldsheet}

Partimos de la suposición que la integral de camino del toro de una cuerda se relaciona
con la energía libre de un gas de cuerdas mediante
\begin{equation}
  Z =  -\beta F.
\end{equation}

Esta relación se ha demostrado exacta para ciertos casos.


La integral de camino del toro en una campo $G_{\mu\nu}$, el cual representa la curvatura
del espacio-tiempo, en $d$ dimensiones es
\begin{equation}
  Z=\int_0^\infty \frac{d\tau_2}{2\tau_2} \int_{-1/2}^{1/2} d\tau_1 \Delta_{FP} \int \mathcal DX
  \sqrt G e^{-S},
  \label{eq:toruspi}
\end{equation}
donde el determinante de Faddeev-Popov se introduce debido a la libertad gauge bajo difeomorfismos
y trasformaciones Weyl.
La acción es 
\begin{equation}
  S= \frac{1}{4\pi \alpha'}\int_{0}^{1} d^2\sigma \sqrt h h^{\alpha\beta} \partial_\alpha X^\mu\partial_\beta X^\nu G_{\mu\nu}.
\end{equation}
La métrica de la worldsheet es
\begin{equation}
  h_{\alpha\beta} =
\qty(
\begin{array}{cc}
  1 & \tau_1 \\
  \tau_1 & \tau_1^2+\tau_2^2
\end{array}
).
\end{equation}

Definimos las nuevas coordenadas $\sigma=\sigma_1/\tau_2$ y $\tau=\sigma_2$, de modo 
que la acción se convierte en 
\begin{equation}
  \begin{aligned}
    S= \frac{1}{4\pi \alpha'}\int_0^{1/\tau_2} d\sigma  \int_0^1 d\tau 
    &\biggl[
      \qty(1+\frac{\tau_1^2}{\tau^2_2})\partial_\sigma X^\mu \partial_\sigma X^\nu G_{\mu\nu} +
      2\frac{\tau_1}{\tau_2}\partial_\sigma X^\mu \partial_\tau X^\nu G_{\mu\nu}  \\
      &+ \partial_\tau X^\mu \partial_\tau X^\nu G_{\mu\nu}\biggl].
  \end{aligned}
\end{equation}

La divergencia de Hagedorn aparece al tener en cuenta cuerdas que se enrollan en el tiempo una
vez.
\begin{equation}
  X^0(\sigma_1,\sigma_2+1) = X^0(\sigma_1,\sigma_2)\pm \beta.
\end{equation}

Haciendo el desarrollo de Fourier de $X^\mu(\sigma,\tau)$ en la coordenada $\sigma$, obtenemos
\begin{equation}
  \begin{aligned}
    X^i(\sigma,\tau) =& \sum_{n=-\infty}^\infty X_n^i(\tau) e^{in2\pi \tau_2 \sigma} \\
    X^0(\sigma,\tau) =& \pm \beta \tau_2 \sigma +  \sum_{n=-\infty}^\infty X_n^0(\tau) e^{in2\pi \tau_2 \sigma}.
  \end{aligned}
\end{equation}

Definimos $X^i_0(\tau)=X^i(\tau)$ y $X_0^0(\tau)=X^0(\tau)$.
Como la divergencia de Hagedorn se produce en $\tau_2\to\infty$, nos quedamos con el modo $n=0$.
%Esto significa que despreciamos las vibraciones de la cuerda ortogonales a 
Al integrar en $\tau_2$, obtenemos la acción de una partícula
\begin{equation}
  \begin{aligned}
    S=\frac{1}{4\pi\alpha'\tau_2}\int_0^1 d\tau \big[
      \beta^2 (\tau_1^2+\tau_2^2)G_{00} \pm 2\beta\tau_1 G_{0\mu}\partial_\tau X^\mu  +G_{\mu\nu} \partial_\tau X^\mu \partial_\tau X^\nu
    \big].
  \end{aligned}
\end{equation}

Depreciar el resto de modos en la serie de Fourier hace que a la acción resultante le falten
términos adicionales. 
En una métrica plana, la corrección a la acción es
\begin{equation}
  \Delta S = - \frac{\tau_2^2 \beta^2_{H0}}{4\pi\alpha' \tau_2}.
\end{equation}

Donde $\beta_{H0}$ es la temperatura de Hagedorn en espacio plano.
Definiendo un nuevo parámetro $t=\tau_2 \tau$ e incluyendo la corrección a la acción
\begin{equation}
  \begin{aligned}
    S=\frac{1}{4\pi\alpha'}\biggl[ -\beta_{H0}^2\tau_2 + \int_0^{\tau_2} dt \biggl\{
      \beta^2 \frac{\tau_1^2+\tau_2^2}{\tau_2^2}G_{00} \pm 2\beta\frac{\tau_1}{\tau_2} G_{0\mu}\partial_t X^\mu  +G_{\mu\nu} \partial_t X^\mu \partial_t X^\nu
    \biggl\}\biggl].
  \end{aligned}
\end{equation}

En una métrica estacionaria y que cumpla $G_{0i}=0$
\begin{equation}
  \begin{aligned}
    S=\frac{1}{4\pi\alpha'}\biggl[ -\beta_{H0}^2\tau_2 + \int_0^{\tau_2} dt \biggl\{&
      \beta^2 \frac{\tau_1^2+\tau_2^2}{\tau_2^2}G_{00} \pm 2\beta\frac{\tau_1}{\tau_2} G_{00}\partial_t X^0  +G_{00} \qty(\partial_t X^0)^2 \\
    &+G_{ij} \partial_t X^i \partial_t X^j
    \biggl\}\biggl].
  \end{aligned}
\end{equation}

Si descomponemos la coordenada temporal en una parte clásica y una corrección adicional,
$X^0=X^{c}+\tilde X^0$, podemos hacer la integral de camino sólo sobre $\tilde X^0$. 
Primero hemos de buscar la acción clásica.
La ecuación de Euler-Lagrange es 
\begin{equation}
  \partial_t\qty[G_{00}\partial_t X^c \pm \beta \frac{\tau_1}{\tau_2}G_{00}]=0.
\end{equation}

Por lo que 
\begin{equation}
  G_{00}\partial_t X^c \pm \beta \frac{\tau_1}{\tau_2}G_{00}=C.
\end{equation}

Como la coordenada $X^0$ es periódica,
\begin{equation}
  0=X^0(\tau_2) -X^0(0) = X^0(t)\eval_0^{\tau_2}= \int_0^{\tau_2}\partial_t X^0(t) = \ev{\partial_t X^0} 
\end{equation}

Promediando,
\begin{equation}
  \ev{\partial_t X^c} \pm \ev{\beta \frac{\tau_1}{\tau_2}}=\ev{\frac{C}{G_{00}}} .
\end{equation}

Obtenemos la constante
\begin{equation}
  C = \pm \beta \frac{\tau_1}{\ev{G_{00}^{-1}}}
\end{equation}

La acción clásica es
\begin{equation}
  S^{cl}=\frac{1}{4\pi\alpha'}\ev{G_{00}(\partial_t X^0)^2 \pm 2\beta \frac{\tau_1}{\tau_2}G_{00}\partial_t X^0}.
\end{equation}

Aplicando 
\begin{equation}
  \ev{G_{00}(\partial_t X^0)^2 \pm \beta \frac{\tau_1}{\tau_2}G_{00}\partial_t X^0} = C\ev{\partial_t X^0} = 0,
\end{equation}
y
\begin{equation}
  \ev{G_{00}}\partial_t X^0 = \tau_2 C\mp \frac{\tau_1}{\tau_2}\beta \ev{G_{00}},
\end{equation}
la acción clásica puede escribirse como
\begin{equation}
  S^{cl}= \frac{1}{4\pi\alpha'} \qty[\frac{\tau_1^2\beta^2}{\ev{G_{00}^{-1}}}-\beta^2\frac{\tau_1^2}{\tau_2^2}\ev{G_{00}}].
\end{equation}

Como la trayectoria clásica hace estacionaria la acción, los términos lineales en $\tilde X^0$
se anulan y la integral de camino en $\tilde X^0$, tras varios cálculos, es
\begin{equation}
  Z_0=\frac{1}{\sqrt{4\pi^2\alpha'\ev{G_{00}^{-1}}}}\beta.
\end{equation}

La integral en $\tau_1$ en la ecuación \ref{eq:toruspi}, da lugar al término 
\begin{equation}
  \frac{2\pi\sqrt{\alpha'}\sqrt{\ev{G_{00}^{-1}}}}{\beta}.
\end{equation}

La integral de camino se reduce a 
\begin{equation}
  Z=2\int_0^\infty \frac{d\tau_2}{2\tau_2}\int \mathcal D \vec {X} \sqrt{\prod_t \det G_{ij}}e^{-Sp}
\end{equation}
donde la acción es
\begin{equation}
  S_p=\frac{1}{4\pi\alpha'}\qty[-\beta^2_{H0}\tau_2+\beta^2 \int_0^{\tau_2}dt \qty(G_{00}+G_{ij}\partial_t X^i\partial_t X^j)]
\end{equation}

Hemos reducido la función de partición de una cuerda a la integral de camino de
una partícula no relativista moviéndose en un espacio curvo.

No empleamos este método, sino que 

\section{Campo efectivo}

La acción de un campo taquiónico a orden más bajo en $\alpha'$ es
\begin{equation}
  S=\frac{1}{2}\int d^d x\sqrt G e^{-2\Phi}  \qty(G^{\mu\nu}\partial_\mu T\partial_\nu T+m^2T^2),
\end{equation}
donde $T$ es un campo escalar real y $\Phi$ el campo del dilatón.

Si el tiempo tiene periodo $2\pi R$, el desarrollo de Fourier de $T$ es
\begin{equation}
  T(x^0,x^i)=\sum_{n=-\infty}^{\infty} T_n(x^i)e^{\frac{inx^0}{R}}.
\end{equation}
Hay que tener en cuenta que $T_n(x^i)$ es un campo escalar complejo
y $n$ es el momento discretizado en la dimensión compacta.
En la variedad termal, $R=\beta/2\pi$. 
Suponemos una métrica estacionaria y $G_{i0}=0$.
Sustituyendo,
\begin{equation}
  \begin{aligned}
    S=\sum_{n,m} \int d^{d-1} \sqrt{G}e^{-2\Phi}\biggl(&-\frac{nm}{R^2}G^{00}\partial_0 T_n \partial_0 T_{m} 
    + G^{ij}\partial_i T_n \partial_j T_{m} \\
    &+ m^2 T_n T_{m}\biggl) \int_{-\pi R}^{\pi R} \frac{dx^0}{2}e^{i\frac{n+m}{R}x^0}.
  \end{aligned}
\end{equation}

La integral en $x^0$ es $\pi R \delta_{-n,m}$, por lo que
\begin{equation}
  S=\sum_n\pi R \int d^{d-1} \sqrt{G}e^{-2\Phi}\qty(G^{ij}\partial_i T_n \partial_j T_{-n} + \frac{n^2 G^{00}}{R^2}T_n T_{-n} + m^2 T_n T_{-n}).
\end{equation}

Como $T$ es un campo real, $T_n=T^*_{-n}$ y por tanto
\begin{equation}
  S=\sum_n \pi R \int d^{d-1} \sqrt{G}e^{-2\Phi}\qty(G^{ij}\partial_i T_n \partial_j T_{n}^* + \frac{n^2 G^{00}}{R^2}T_n T_{n}^* + m^2 T_n T_{n}^*).
\end{equation}

Aunque en la acción solo aparece el momento $n$ y no los enrollamientos $w$, esto se debe 
al desarrollo de Fourier, pues la acción completa cumpliría la dualidad T. 
Haciendo las transformaciones 
\todo{Dudoso}
\begin{equation}
  \begin{aligned}
    G_{00}&\to \frac{1}{G_{00}},\\
    \Phi&\to \Phi -\frac{1}{2}\ln G_{00},\\
    T_n&\to T_w.
  \end{aligned}
\end{equation}
y $R\to \alpha'/R$, los momentos se convierten en 
enrrollamientos
\begin{equation}
  S \sim\sum_w  \int d^{d-1} x \sqrt G e^{-2\Phi}\qty(G^{ij}\partial_i T_w \partial_j T_{w}^* + \frac{w^2 R^2 G_{00}}{{\alpha'}^2}T_w T_{w}^* + m^2 T_w T_{w}^*) .
\end{equation}

Vemos cómo aparece un término de masa efectiva $m^2_{loc} = m^2 +\frac{w^2 R^2 G_{00}}{{\alpha'}^2}$,
que depende de la posición espacio-temporal.
Dependiendo del tipo de teoría que se considere (bośonica, tipo II supersimétrica o heterótica), 
esta masa efectiva será distinta.
\todo{O no?}

La contribución dominante de la acción vendrá dada por $w=\pm 1$, por lo que definimos $T=T_1$.
Integrando por partes la acción 
\todo{Demostrar}
\begin{equation}
  S = \int d^{d-1} x \sqrt G e^{-2\Phi} T^* \qty[-\nabla^2-G^{ij}\frac{\partial_j \sqrt{G_{00}}}{\sqrt G_{00}}\partial_i + m^2_{loc}]T,
\end{equation}
donde $\nabla^2 = G^{ij}\nabla_i \partial_j$.
Para calcular la integral de camino asociada a está acción tenemos que buscar los valores
propios del operador 
\begin{equation}
  \widehat{O} =-\nabla^2-G^{ij}\frac{\partial_j \sqrt{G_{00}}}{\sqrt G_{00}}\partial_i + m^2_{loc}.
\end{equation}

Definiendo el producto escalar
\begin{equation}
  \braket{\psi_1}{\psi_2} = \int d^{d-1} x \sqrt G \psi_1(x)^* \psi_2(x),
\end{equation}
el operador $\widehat O$ es hermítico y por tanto posee un conjunto ortonormal de funciones
propias $\psi_n$ con valores propios reales, $\lambda_n$.
%La acción se reduce a 
%\begin{equation}
%  S=\int d^{d-1} x\sqrt G T^* \widehat O T 
%  =\sum_{n,m} d^{d-1}\sqrt G \psi_n^* \psi_m a^*_n a_m\lambda_m = \sum_n \abs{a_n}^2 \lambda_n.
%\end{equation}

La integral de camino da como resultado
\begin{equation}
  Z=\int \mathcal D T e^{-S} =\pi_n \frac{1}{\lambda_n} = \det(\widehat O)^{-1}.
\end{equation}

La energía libre de un gas de cuerdas es
\begin{equation}
  F=-\frac{1}{\beta}\ln Z =\frac{1}{\beta} \Tr \ln \widehat O.
\end{equation}

Aplicando la fórmula de Schwinger del logaritmo, 
\begin{equation}
  \ln a = -\int_0^\infty \frac{dT}{T}\qty(e^{-aT}-e^{-T}),
\end{equation}
y despreciando el término $e^{-T}$, la energía libre es
\begin{equation}
  F=-\frac{1}{\beta}\int_0^\infty \frac{dT}{T} \Tr e^{T\widehat O}.
\end{equation}

Para poder interpretar la energía líbre en términos de una partícula
moviéndose en un espacio sin tiempo, tenemos que aislar la componente $G_{00}$
de la métrica, que aparece al tomar la traza.
Haciendo el cambio de base
\begin{equation}
  \ket{\phi_n}=G_{00}^{1/4}  \ket{\psi_n}.
\end{equation}

El producto escalar asociado es
\begin{equation}
  \braket{\phi_1}{\phi_2} = \int d^{d-1} x \sqrt G_{ij} \phi_1(x)^* \phi_2(x).
\end{equation}

Tras el cambio de base
\begin{equation}
  F=\frac{1}{\beta}\int_0^\infty \frac{dT}{T} \Tr e^{T\widehat D}.
\end{equation}

\begin{equation}
  \widehat D = -\nabla^2  + m_{loc}^2- \frac{3}{16}\frac{G^{ij} \partial_i G_{00}\partial_j G_{00}}{G_{00}^2}
  +\frac{\nabla^2 G_{00}}{4G_{00}}.
\end{equation}
Los últimos dos términos corresponden con un potencial efectivo
\begin{equation}
  K(x) = \frac{3}{16}\frac{G^{ij} \partial_i G_{00}\partial_j G_{00}}{G_{00}^2}
  +\frac{\nabla^2 G_{00}}{4G_{00}}.
\end{equation}

\begin{equation}
  F=\frac{1}{\beta}\int_0^\infty \frac{dT}{T} \int_{S^1} \mathcal D x\sqrt G_{ij}  e^{ -\frac{1}{4\pi\alpha'}\int_0^T dt \qty(  
  \frac{1}{4}G_{ij}\dot x^i \dot x^j  + m_{loc}^2+K(x) )}.
\end{equation}
\todo{Turbio}



\section{Cuerdas cerca de agujeros negros}

Al aplicar los cálculos anteriores a un espacio de Rindler, hemos de tener en cuenta si la
correcciones a orden superior en $\alpha'$ modifican la divergencia de la energía libre.

\begin{equation}
  F=-\frac{1}{\beta}\int \frac{dT}{T} \Tr e^{T\qty(  -\nabla^2-G^{ij}\frac{\partial_j \sqrt{G_{00}}}{\sqrt G_{00}}\partial_i + m^2_{loc}) }.
\end{equation}

Tomemos como ejemplo la teoría supersimétrica de tipo II.

El laplaciano en coordenadas de Rindler es
\begin{equation}
  \nabla^2  = \partial_\rho^2 + \frac{1}{\rho}\partial_\rho.
\end{equation}

La masa efectiva es 
\begin{equation}
  m_{loc}^2=-\frac{2}{\alpha'}+\frac{R^2G_{00}}{\alpha'^2}.
\end{equation}


Para calcular la traza buscamos los valores propios del operador en el exponente.
Las funciones propias regulares son del tipo
\begin{equation}
  \psi_n(\rho)\propto e^{-\frac{\beta \rho^2}{4\pi\alpha'^{3/2}} }
  L_n\qty(\frac{\beta\rho^2}{2\pi\alpha'^{3/2}}),
\end{equation}
con valores propios
\begin{equation}
  \lambda_n=\frac{\beta-2\pi\sqrt{\alpha'}+2\beta n }{\pi\alpha'^{3/2}}.
\end{equation}

La contribución dominante a la energía libre viene dada por $\lambda_0$.
La traza depende las dimensiones adicionales al espacio de Rindler. 
En una teoría supersimétrica, tenemos ocho dimensiones cuya variedad tenemos que especificar.
Suponiendo dimensiones planas, la energía libre es
\begin{equation}
  F= -\frac{V_T}{\beta}\int_0^\infty \frac{dT}{T}\qty(\frac{1}{4\pi T})^4 e^{\frac{\beta-2\pi\sqrt{\alpha'}}{\pi\alpha'^{3/2}}T}.
\end{equation}

En el límite superior de $T$, la energía libre diverge si
\begin{equation}
  \beta \leq 2\pi\sqrt{\alpha'}=\beta_R.
\end{equation}

Concluimos que la temperatura de Hagedorn en un espacio de Rindler coincide con la temperatura
de Hawking.
Hemos supuesto cuerdas de tipo II y que las dimensiones adicionales son planas.
De haber escogido otras condiciones, la temperatura de Hagedorn recibiría nuevas correcciones,
como veremos más adelante.

Identificando las funciones propias como funciones de onda de una cuerda, el estado fundamental
\begin{equation}
  \psi_0(\rho)\propto e^{-\frac{\beta \rho^2}{4\pi\alpha'^{3/2}} }
\end{equation}
representa una cuerda localizada a la distancia $l_s$

%Analiticidad vs divergencia

%Densidad de estados

%Streched horizon. Local temp.

\section{Entropía de agujeros negros}
