\chapter*{Resumen}

Este trabajo tiene como objetivo entender cómo la temperatura de Hagedorn, 
característica de todas las teorías de cuerdas, se ve afectada por la presencia de agujeros
negros.
La temperatura de 
La termodinámica de agujeros negros ha  dado lugar.
Se han propuesto 

La base de este trabajo es la tesis \cite{Mertens2015}, en la cual se determina que la temperatura
de Hagedorn .
En el capítulo 1 se introducen los conceptos fundamentales de teoría de cuerdas, siguiendo \cite{Tong:2009np} y \cite{Mertens2015}. 
En el capítulo 2, se analiza la termodinámica de cuerdas en
espacio de Minkowski, deduciendo la temperatura de Hagedorn.
El capítulo 3 trata dos consecuencias de considerar una teoría cuántica de campos 
en presencia de horizonte de sucesos:  el efecto Unruh y el efecto Hawking, siguiendo la tesis \cite{Barbado:2015uua} y \cite{Sus}.
Ambos efectos se deben a que el concepto de partícula no es absoluto, sino que
depende del observador y de la métrica del espacio-tiempo.
El estudio de termodinámica de cuerdas en espacios curvos se realiza en el capítulo 4.
a partir de la energía libre, se determina que la temperatura de Hagedorn en las proximidades
de un agujero negro coincide con la temperatura de Hawking.
posible explicación de la entropía de agujeros negros.
original expuesto en los artículos de investigación en los que se basa \cite{Mertens2015} 
Sigue habiendo muchos aspectos abiertos a la investigación, como se comenta en el capítulo 5.
