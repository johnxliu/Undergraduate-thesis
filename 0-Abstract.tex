\chapter*{Resumen}

Este trabajo tiene como objetivo entender cómo la temperatura de Hagedorn, 
caractéristica de todas las teorías de cuerdas, se ve afectada por la presencia de agujeros
negros.

La base de este trabajo es la tesis \cite{Mertes2015}, en la cual se determina que la temperatura
de Hagedorn 
En el capítulo 1 se introducen los conceptos fundamentales de teoría de cuerdas.
En el capítulo 2, se analiza la termodinámica de una gas de cuerdas en
espacio de Minkowski, donde aparece la temperatura de Hagedorn.
El capítulo 3 trata dos aspectos de considerar una teoría cuántica de campos en 
sistemas de refencia no inerciales: la temperatura de Unruh y la temperatura de Hawking.
El concepto de partícula es dependiente del observador.
El estudio de cuerdas en espacios curvos se realiza en el capítulo 4. 



\ldots

