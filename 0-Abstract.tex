\chapter*{Resumen}

Este trabajo tiene como objetivo entender cómo la temperatura de Hagedorn, 
característica de todas las teorías de cuerdas, se ve afectada por la presencia de agujeros
negros.
Se determinará, que bajo una serie se hipótesis, la temperatura de Hagedorn coincide con la
temperatura de Hawking de un agujero negro.
Además, se dará una justificación de la entropía de los agujeros negros en términos de cuerdas.

La base de este trabajo es la tesis \cite{Mertens2015} (y las referencias mencionadas en ella), en la cual se extienden los cálculos 
de la temperatura de Hagedorn a espacios curvos.

En el capítulo 1 se introducen los conceptos fundamentales de teoría de cuerdas, siguiendo \cite{Tong:2009np} y \cite{Mertens2015}. 
En el capítulo 2, se analiza la termodinámica de cuerdas en espacio de Minkowski, deduciendo la temperatura de Hagedorn.
El capítulo 3 trata dos consecuencias de considerar una teoría cuántica de campos 
en presencia de un horizonte de sucesos:  el efecto Unruh y el efecto Hawking, siguiendo la tesis \cite{Barbado:2015uua} y \cite{Susskind}.
El estudio de la termodinámica de cuerdas en espacios curvos se realiza en el capítulo 4.
A partir de la energía libre, se determina que la temperatura de Hagedorn en las proximidades
de un agujero negro coincide con la temperatura de Hawking.
Este resultado parece indicar una explicación de la entropía de agujeros negros.
Sigue habiendo muchos aspectos abiertos a la investigación, como se comenta en el capítulo 5.
