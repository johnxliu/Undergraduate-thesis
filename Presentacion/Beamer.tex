\documentclass{beamer}
\usepackage{beamerthemesplit} 
\usetheme{Warsaw}
\usepackage[spanish]{babel} 
\usepackage{ucs}
\usepackage[utf8]{inputenc}

\usepackage{amsthm}
\usepackage{amsmath}
\usepackage{amssymb}
\usepackage{esint}

\usepackage{siunitx}
\usepackage{physics}
\usepackage{graphicx}
\usepackage{xcolor}
\setbeamertemplate{navigation symbols}{}

\begin{document}
\title{El problema de Hagedorn y la entropía de agujeros negros} 
\author{John Liu \\
        Tutor: Miguel Ángel Ramos Osorio}
\date{\today} 
\frame{\titlepage} 

T Cuerdas
  Gravedad cuantica
  Cuantizacion cuerdas
  Susy
Hagedorn densidad de estados -> Divergencia
Cuerda larga

Radiacion Hawking -> Temperatura -> Entropia

Hagedorn espacio curvo
  Calculo 
  Hag = Haw
  Entropia

Conclusión

\begin{frame}
\frametitle{Relatividad General}
\begin{itemize}
  \item Gravedad \(\implies\) Espacio curvo
\[
  \eta_{\mu\nu} \to g_{\mu\nu}(x)
\]
\begin{align*}
  G_{\mu\nu}=&kT_{\mu\nu}\\
  \text{Curvatura}\propto& \text{Energía}
\end{align*}

\item Suponemos 
\[
  g_{\mu\nu}=\eta_{\mu\nu}+h_{\mu\nu}, \qquad \abs{h_{\mu\nu}}\ll 1
\]
\end{itemize}

\end{frame}
\end{document}

