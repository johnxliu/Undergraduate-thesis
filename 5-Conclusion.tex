\chapter{Conclusión}

La divergencia a la temperatura Hagedorn de la energía libre de cuerdas en espacio plano
se puede extraer a partir de cuerdas en un estado taquiónico que se enrollan alrededor de
un tiempo periódico.
Asumiendo que esta correspondencia se mantiene en espacio curvo, haciendo la integral 
de camino de un campo taquiónico en espacio de Rindler, se obtuvo que la temperatura de Hagedorn
coincide con la temperatura de Hawking.
Por otro lado, este resultado es consistente con la entropía predicha para agujeros negros.

Es interesante notar que los problemas de agujeros negros en gravedad cuántica suelen ser
no perturbativos, sin embargo los cálculos que hemos hecho se basan completamente en teoría de 
cuerdas perturbativa.
Sería conveniente aplicar un enfoque no perturbativo para tratar un gas de cuerdas en presencia de agujeros negros.

Por otra parte, se podría estudiar cómo se vincula la entropía de agujeros negros debida a cuerdas cercanas al horizonte con la deducción 
de Strominger y Vafa \cite{Strominger}, basada en branas.

Algunos caminos de investigación posibles son las relaciones con el problema de la información, la hipótesis de firewalls y
otras propuestas como los fuzzballs.

De forma recíproca, entender la termodinámica de agujeros negros tal vez arroje luz sobre el 
comportamiento de las cuerdas a altas temperaturas.
