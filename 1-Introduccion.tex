\chapter{Introducción a la teoría de cuerdas}


\section{Motivación}

\todo{Describir brevemente objetivos, utilidad y estado actual}


\todo{Convenio}

\section{La cuerda relativista}

\todo{Métrica, linea de universo}

La teoría de cuerdas parte de considerar que las entidades fundamentales son cuerdas
en vez de partículas. 


La trayectoria $\mathbf x(t)$ de una partícula satisface que su acción $S$ es un extremal.
Coloquialmente, esto significa que la variación de la acción a primer orden es nula bajo
variaciones pequeñas de la trayectoria, supuestas fijas las posiciones iniciales y finales.
La acción de una partícula libre es
\begin{equation}
  S=-m\int dt \sqrt{1-\dot {\vec{x}} \cdot \dot {\vec{x}}},
\end{equation}
donde $\dot x = \dv{x}{t}$.

Para que en la acción aparezca el tiempo y la posición en igualdad de condiciones,
parametrizamos el tiempo y la posición por el tiempo propio $\tau$. 
El tiempo propio es el tiempo que mediría un reloj que se moviese con la partícula.
Con esta transformación, la acción es
\begin{equation}
 S = -m\int d\tau \sqrt{-\dv{x_\mu}{\tau}\dv{x^\mu}{\tau}}=-m\int d\tau .
\end{equation}


Generalizaremos la acción de una partícula libre a una cuerda libre.
Una cuerda está parametrizada por una variable temporal $\tau$ y una variable espacial $\sigma$, adimensional.
De forma más compacta, $(\sigma^0,\sigma^1)=(\tau,\sigma)$. 
La trayectoria de la cuerda en el espacio-tiempo genera un superficie llamada \emph{worldsheet}.
Las coordenadas en la worldsheet determinan un punto del espacio-tiempo $X^\mu$, también llamado
espacio \emph{target} para evitar confusión.

Para una cuerda cerrada con periodicidad $2\pi$, identificamos $X^\mu(\tau,\sigma)=X^\mu(\tau,\sigma+2\pi)$.

La acción de una partícula es proporcional a la longitud de su línea de universo.
De forma análoga, la acción de una cuerda debería ser proporcional al área de la
worldsheet.
Para poder medir el área, es necesario definir una métrica $\gamma_{\alpha\beta}$ en la worldsheet.
La forma natural de definir una métrica en una superficie, a partir de la métrica del espacio que
contiene a la superficie, es mediante el concepto de \emph{pull-back}.
En este caso, la métrica en la worldsheet el pull-back de la métrica de Minkowski
\begin{equation}
  \gamma_{ab}=\pdv{X^\mu}{\sigma^\alpha}\pdv{X^\nu}{\sigma^\beta}\eta_{\mu\nu}.
\end{equation}

La propiedad esencial de la métrica $\gamma_{\alpha\beta}$ es que la distancia entre dos puntos próximos 
coincide con la calcula con la métrica $\eta_{\mu\nu}$, ya que
\begin{equation}
  g_{\mu\nu}(X(\sigma)) dX^\mu(\sigma)dX^\nu(\sigma) = \gamma_{\alpha\beta} d\sigma^\alpha d\sigma^\beta.
\end{equation}



La medida de integración invariante bajo cambios generales de coordenadas más sencilla 
es $d^2\sigma \sqrt{-\det\gamma}$, por lo que la acción, llamada de Nambu-Goto, es
\begin{equation}
  S_{NG}=-T\int d^2\sigma \sqrt{-\det\gamma}.
\end{equation}

El parámetro $T$ se corresponde con la tensión de la cuerda y se puede expresar como
\begin{equation}
  T=\frac{1}{2\pi\alpha'},
\end{equation}
donde $\alpha'$ es la pendiente de Regge.

A la hora de cuantizar la teoría, la raíz cuadrada es problemática, por lo que se introduce
un campo tensorial $h$ definido sobre la worldsheet en la llamada acción de Polyakov
\begin{equation}
  S_P=-\frac{1}{4\pi\alpha'}\int d^2\sigma  \sqrt{-h}h^{\alpha\beta}\partial_\alpha X^\mu \partial_\beta X_\mu.
\end{equation}

Este campo $h$ se comporta como una métrica en dos dimensiones y queda fijada por las
ecuaciones de movimiento
\begin{equation}
  h_{\alpha\beta}=2f(\sigma)\partial_\alpha X^\mu \partial_\beta X_\mu
\end{equation}
donde $f(\sigma)$ es una función cualquiera. La libertad de poder escoger $f(\sigma)$
es una simetría gauge

\todo{Simetría gauge, Weyl}


\section{Cuantización}

Debido a la simetría gauge de la teoría, la cuantización no es directa.
Hay grados de libertad que no son físicos y en algún momento hemos de deshacernos de ellos.
Para obtener directamente una teoría unitaria, cuantizamos solo los grados de libertad 
físicos, buscando primero las soluciones clásicas. 
Como contrapartida, perdemos la invariancia de Lorentz explícita.

Definimos las coordenadas en el cono de luz
\begin{equation}
  \sigma^\pm=\tau\pm\sigma
\end{equation}

y
\begin{equation}
  X^\pm=\frac{1}{\sqrt 2} (X^0 \pm X^{d-1}).
\end{equation}

Gracias a la simetría gauge, imponemos $h_{\alpha\beta}=\eta_{\alpha\beta}$, de forma que 
la acción de Polyakov es
\begin{equation}
  S = -\frac{1}{4\pi\alpha'} \int d^2\sigma \partial_\alpha X^\mu \partial^\alpha X_\mu,
\end{equation}
que da lugar a la ecuaciones de movimiento de ondas libres
\begin{equation}
  \partial_\alpha \partial^\alpha X^\mu=0.
\end{equation}

La solución general de la ecuación de movimiento para $X^+$ se descompone
en una onda moviéndose hacia la izquierda $X^+_L$ y otra hacia la derecha $X^+_R$,
\begin{equation}
   X^+ =X^+_L(\sigma^+) + X^+_R(\sigma^-).
\end{equation}

Teniendo en cuenta la periodicidad en $\sigma$, la expansión de Fourier conduce a
\begin{equation}
  \begin{gathered}
    X^+_L=\frac 1 2 x^+ + \frac 1 2 \alpha' p^+ \sigma^+ + i\sqrt{\frac{\alpha'}{2}}\sum_{n\neq 0} \frac{1}{n}\tilde \alpha^+_n e^{-in\sigma^+},\\
    X^+_R=\frac 1 2 x^+ + \frac 1 2 \alpha' p^+ \sigma^- +i\sqrt{\frac{\alpha'}{2}}\sum_{n\neq 0} \frac{1}{n}\alpha^+_n e^{-in\sigma^-}.
  \end{gathered}
\end{equation}
Donde $x^+$ y $p^+$ se corresponden con la posición y el momento del centro de masas, respectivamente.


Debido a la invariancia baja reparametrizaciones, escogemos
\begin{equation}
  X^+_L=\frac 1 2 x^+ + \frac 1 2 \alpha' p^+ \sigma^+,
\end{equation}
y
\begin{equation}
  X^+_R=\frac 1 2 x^+ + \frac 1 2 \alpha' p^+ \sigma^+.
\end{equation}

Por tanto, 
\begin{equation}
  X^+ = x^+ + \alpha' p^+ \tau.
\end{equation}

Con esta elección gauge, la solución $X^-$ que determinada casi por completo.
La ecuación de movimiento para la métrica $h$ se obtiene variando a acción
\begin{equation}
  \delta S = \frac{1}{4\pi\alpha'}\int d^2\sigma \delta h^{\alpha\beta}
  \qty(
  \sqrt{-h}\partial_\alpha X^\mu \partial_\beta X_\mu -
  \frac 1 2 \sqrt{-h}h_{\alpha\beta}h^{\rho\sigma} \partial_\rho X^\mu \partial_\sigma X_\mu).
\end{equation}
Como habíamos elegido $h_{\alpha\beta}=\eta_{\mu\nu}$, la ecuación de movimiento impone
\begin{equation}
  \partial_\alpha X^\mu \partial_\beta X_\mu - \frac 1 2 \eta_{\alpha\beta} \eta^{\sigma\rho} 
  \partial_\rho X^\mu  \partial_\sigma X_\mu = 0.
\end{equation}

En las coordenadas del cono de luz, esto se traduce en
\begin{equation}
  (\partial_+ X)^2 = (\partial_- X)^2 = 0.
  \label{eq:lig}
\end{equation}

Haciendo la descomposición de la solución $X^-$ 
\begin{equation}
  X^-=X^-_L(\sigma^+)+X^-_R(\sigma^-),
\end{equation}
la ecuación \ref{eq:lig} conduce a 
  \begin{align}
    \partial_+ X_L^- = \frac{1}{\alpha'p^+}\sum_{i=1}^{D-2} \partial_+ X^i \partial_+ X^i, \label{eq:lig1}\\
    \partial_- X_R^- = \frac{1}{\alpha'p^-}\sum_{i=1}^{D-2} \partial_- X^i \partial_- X^i. \label{eq:lig2}
  \end{align}

En el desarrollo de Fourier
\begin{equation}
  \begin{gathered}
    X^-_L(\sigma^+)=\frac 1 2 x^- + \frac 1 2 \alpha' p^- \sigma^+ + i\sqrt{\frac{ \alpha'}{ 2}}
    \sum_{n\neq0} \frac 1 n \tilde{\alpha}^-_n e^{-in\sigma^+} \\                             
    X^-_R(\sigma^+)=\frac 1 2 x^- + \frac 1 2 \alpha' p^- \sigma^- + i\sqrt{\frac{ \alpha'}{ 2}}
    \sum_{n\neq0} \frac 1 n \alpha^-_n e^{-in\sigma^-},
  \end{gathered}
\end{equation}
las constantes $p^-$, $\tilde \alpha^-_n$ y $\alpha^-_n$ quedan determinadas por las ecuaciones
\ref{eq:lig1} y \ref{eq:lig2}.

El momento $p^-$ se puede expresar a través de $\alpha^i_n$ y de $\tilde \alpha^i_n$ como
\begin{equation}
  p^- = \frac{1}{\alpha' p^+}\sum_{i=1}^{D-2} \qty(\frac{1}{2}\alpha'p^i p^i +\sum_{n\neq0}\alpha_n^i\alpha_{-n}^i) 
   = \frac{1}{\alpha' p^+}\sum_{i=1}^{D-2} \qty(\frac{1}{2}\alpha'p^i p^i +\sum_{n\neq0}\tilde \alpha_n^i\tilde \alpha_{-n}^i) .
\end{equation}

La masa de la cuerda es por tanto
\begin{equation}
  M^2=-p_\mu p^\mu = 2p^+p^- - \sum_{i=1}^{D-2} p^i p^i = 
  \frac{2}{\alpha'}\sum_{i=1}^{D-2} \sum_{n\neq 0} \alpha_{-n}^i \alpha_n^i
  =\frac{2}{\alpha'}\sum_{i=1}^{D-2} \sum_{n\neq 0}\tilde \alpha_{-n}^i \tilde\alpha_n^i.
  \label{eq:mass}
\end{equation}

Hemos obtenido la solución clásica mediante los $2(D-2)$ modos de oscilación transversos 
$\alpha^i_n$ y $\tilde \alpha_n^i$ y las constantes $x^i$, $p^i$,$p^+$ y $x^-$.
La cuantización consiste en promover los grados de libertad a operadores (denotados
por un acento circunflejo) que satisfacen una reglas de conmutación.
\footnote{En realidad, debido a la simetría gauge, la cuantización se haría a partir de los
corchetes de Poisson de la teoría clásica. No tendremos en cuenta esta sutileza porque el
resultado es el mismo.}
Estas reglas son
\begin{equation}
  \begin{gathered}
    \qty[\widehat x^i,\widehat p^j] = i\delta^{ij}\qquad , \qquad [\widehat x^-,\widehat p^+] = -i,\\
    \qty[\widehat \alpha_n^i,\widehat \alpha_m^j]= \qty[\widehat {\tilde \alpha}_n^i,\widehat {\tilde \alpha}_m^j]= n \delta^{ij}\delta_{n+m,0}
  \end{gathered}
\end{equation}

Los estados físicos sobre los que actúan los operadores se construyen a partir de un estado 
de vacío $\ket{0;p}$, que describe una cuerda de momento $p$ en el estado fundamental.
El vacío verifica
\begin{equation}
  \widehat p^\mu \ket{0;p} = p^\mu \ket{0;p} \qquad, \qquad \widehat \alpha^i_n\ket{0;p}=\widehat {\tilde\alpha}^i_n = 0 \qquad \text{con} \quad n>0.
\end{equation}

Las distintas excitaciones de la cuerda se obtienen aplicando $\alpha_{-n}^i$ y $\widehat{\tilde\alpha}_{-n}^i$
con $n>0$ sobre el vacío.
Cada excitación de la cuerda corresponde a una partícula, como veremos más adelante.

La fórmula de masas es análoga al caso clásico \ref{eq:mass} salvo una diferencia importante.
Clásicamente el producto de $\alpha^i_{-n}$ y $\alpha^i_n$ conmuta, por lo que hay una 
ambigüedad en el orden a asignar a los operadores $\widehat\alpha^i_{-n}$ y $\widehat \alpha^i_n$.
La ambigüedad introduce una constante $c$ desconocida en el operador de la masa al cuadrado,
\begin{equation}
  \widehat M^2 = \frac{4}{\alpha'}\qty(\sum_{i=1}^{D-2} \sum_{n>0} \alpha_{-n}^i\alpha_n -c)
   = \frac{4}{\alpha'}\qty(\sum_{i=1}^{D-2} \sum_{n>0} \tilde \alpha_{-n}^i\tilde \alpha_n -c).
\end{equation}

Por conveniencia, definimos los operadores número
\begin{equation}
  \widehat N=\sum_{i=1}^{D-2} \sum_{n>0}\widehat \alpha_{-n}^i\widehat\alpha_n^i \quad , \quad   
  \widehat {\tilde N}=\sum_{i=1}^{D-2} \sum_{n>0} \widehat{\tilde \alpha}_n^i\widehat{ \tilde\alpha}_n^i, 
\end{equation}
por lo que 
\begin{equation}
  \widehat M^2 = \frac{4}{\alpha'} (\widehat N -c ) = \frac{4}{\alpha'} (\widehat{\tilde N} -c).
\end{equation}

La condición $\widehat N=\widehat {\tilde N}$ se conoce como \emph{level-matching}.

La forma general de determinar $c$ (la carga central) consiste en imponer la simetría Weyl.
Nosotros no entraremos en los detalles, nos basta saber que para cuerdas bosónicas libres
bosónicas, $c=\frac{D-2}{24}$.


\subsection{Espectro de una cuerda}
En el estado de vacío, $\widehat N\ket{0;p}=\widehat {\tilde N}\ket{0;p}=0$, por lo que la masa es
\begin{equation}
  M^2 = -\frac{1}{\alpha'}\frac{D-2}{6}.
\end{equation}

Si $D>2$ la masa al cuadrado de la cuerda es negativa y la partícula a la que corresponde se denomina taquión.
Los taquiones aparecen también en teoría cuántica de campos al estudiar un campo $\phi$ con un 
potencial $V(\phi)$.
La masa de la partícula asociada al campo es
\begin{equation}
  M^2 = \pdv[2]{V(\phi)}{\phi}.
\end{equation}
Por tanto, una masa al cuadrado negativa indica que estamos haciendo una expansión en torno
a un máximo. 
Un ejemplo es el campo de Higgs cuando el valor del campo es cero.
A pesar de los problemas que introduce un taquión, no lo tendremos más en cuenta. 
De todas formas, al considerar cuerdas fermiónicas en la teoría supersimétrica, el taquión
desaparece.

El primer estado excitado se obtiene como
\begin{equation}
  \widehat{\tilde\alpha}_{-1}^i\widehat \alpha_{-1}^j\ket{0;p},
\end{equation}
con masa
\begin{equation}
  M^2 = \frac{4}{\alpha'}\qty(1-\frac{D-2}{24}).
\end{equation}

Teniendo en cuenta los valores posibles de $i$ y $j$ hay $(D-2)^2$ estados excitados.

\todo{Rematar}

Para que estos estados puedan describirse mediante la clasificación de Wigner de las
representaciones del grupo de Poincaré, los estados han de ser una representación $SO(D-2)$.
Esto significa que las partículas tienen masa nula, en caso contrario, se rompería la simetría
de Lorentz. Como $M^2=0$, obtenemos que la dimensión del espacio-tiempo es
\begin{equation}
  D=26.
\end{equation}
Además, la representación de estos estados se descompone un una parte simétrica, una antisimétrica
y una traza. Los campos asociados son el campo del gravitón $G_{\mu\nu}$, el campo Kalb-Ramond $B_{\mu\nu}$
y el dilatón $\Phi$, respectivamente.

Por tanto, la teoría bosónica predice métrica de la relatividad general $G_{\mu\nu}$ de forma 
natural, además de otro campo tensorial y un campo escalar.

Las siguientes excitaciones empiezan a tener masas del orden de la masa de Planck
\begin{equation}
  M_p=
\end{equation}
y están muy lejos de ser observadas. Sin embargo, el resto del trabajo se centrará en cuerdas
a esta escala de energía.


\subsection{Cuerdas supersimétricas}

La cuerdas supersimétricas describen tanto bosones como fermiones.
Algunas 

Existen distintas formas de incorporar fermiones:
en las cuerdas de tipo II , mientras que las cuerdas heteróticas

\todo{Acabar}

\subsection{La integral de camino}

Función beta

Dualidad T





Definiendo
\begin{equation}
  L_n=\frac 1 2 \sum_m \alpha_{n-m} \cdot \alpha_m
\end{equation}

\begin{equation}
  \tilde{L}_n=\frac 1 2 \sum_m \tilde{\alpha}_{n-m} \cdot \tilde{\alpha}_m.
\end{equation}

Es importante ver que hay una expresión en términos de modos moviéndose a la derecha
y otra con modos moviéndose a la izquierda.

