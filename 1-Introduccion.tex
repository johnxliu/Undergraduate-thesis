\chapter{Introducción a la teoría de cuerdas}


\section{Motivación}

\todo{Describir brevemente objetivos, utilidad y estado actual}

\section{La partícula relativista}

\todo{Fundamentos relatividad especial}

\section{La cuerda relativista}

\todo{Explicar distintas acciones, }

La teoría de cuerdas parte de considerar que las entidades fundamentales son cuerdas
en vez de partículas. 
Consideremos primero la descripción relativista de una cuerda. En el caso de una
partícula la acción es
\begin{equation}
  S=-m\int ds = -m\int d\tau \sqrt{-\dot{x}_\mu\dot{x}^\mu}=-m\int dt \sqrt{1-\dot {\vec{x}} \cdot \dot {\vec{x}}}.
\end{equation}

donde la posición de la partícula $X^\mu$ está parametrizada por $\tau$.
Una cuerda estará parametrizada por una variable temporal $\tau$ y una variable espacial $\sigma$.
Para una cuerda cerrada con periodicidad $2\pi$, $X^\mu(\tau,\sigma)=X^\mu(\tau,\sigma+2\pi)$.
De forma más compacta, $(\sigma^0,\sigma^1)=(\tau,\sigma)$.

La acción de una partícula es proporcional a la longitud de su línea de universo.
De forma análoga, la acción de una cuerda debería ser proporcional al área de la
\emph{worldsheet}. 
La métrica inducida en la \emph{worldsheet} es la \emph{pull-back} de la métrica de Minkowski
en la \emph{worldsheet},
\begin{equation}
  \gamma_{ab}=\pdv{X^\mu}{\sigma^\alpha}\pdv{X^\nu}{\sigma^\beta}\eta_{\mu\nu}.
\end{equation}

La medida de integración invariante bajo cambios generales de coordenadas más sencillas 
es $d^2\sigma \sqrt{-\det\gamma}$, por lo que la acción es
\begin{equation}
  S_{NG}=-T\int d^2\sigma \sqrt{-\det\gamma}.
\end{equation}

El parámetro $T$ se corresponde con la tensión de la cuerda y se puede expresar como
\begin{equation}
  T=\frac{1}{2\pi\alpha'}
\end{equation}

donde $\alpha'$ es la pendiente de Regge.
Para cuantizar la teoría, la raíz cuadrada es problemática, por lo que se introduce
el campo $h$ definido sobre la \emph{worldsheet} en la llamada acción de Polyakov
\begin{equation}
  S_P=-T\int d^2  \sqrt{-h}h^{\alpha\beta}\partial_\alpha X^\mu \partial_\beta X_\mu.
\end{equation}

Este campo $h$ se comporta como una métrica en dos dimensiones y queda fijada por las
ecuaciones de movimiento
\begin{equation}
  h_{\alpha\beta}=2f(\sigma)\partial_\alpha X^\mu \partial_\beta X_\mu
\end{equation}

donde $f(\sigma)$ es una función cualquiera.

%Simetría diffxWeyl y conformal transormations.

Debido a la simetría gauge de la teoría, la cuantización no es directa.
Para obtener directamente una teoría unitaria, se cuantizan solo los grados de libertad 
físicos. Como contrapartida, se pierde la invariancia de Lorentz explícita.

Escojamos las coordenadas en el cono de luz
\begin{equation}
  \sigma^\pm=\tau\pm\sigma
\end{equation}

y
\begin{equation}
  X^\pm=\frac{1}{\sqrt 2} (X^0 \pm X^{d-1}).
\end{equation}

Las ecuaciones de movimiento llevan a que 
\begin{equation}
  X^+=x^+\alpha p^+ \tau
\end{equation}

y
\begin{equation}
  X^-=X^-_L(\sigma^+)+X^-_R(\sigma^-)
\end{equation}

donde 
\begin{equation}
  X^-_L(\sigma^+)=\frac 1 2 x^- + \frac 1 2 \alpha p^- \sigma^+ + i\sqrt{\frac \alpha 2}
  \sum_{n\neq0} \frac 1 n \tilde{\alpha}^-_n e^{-in\sigma^+}
\end{equation}

y
\begin{equation}
  X^-_R(\sigma^+)=\frac 1 2 x^- + \frac 1 2 \alpha p^- \sigma^- + i\sqrt{\frac \alpha 2}
  \sum_{n\neq0} \frac 1 n \alpha^-_n e^{-in\sigma^-}
\end{equation}

Definiendo
\begin{equation}
  L_n=\frac 1 2 \sum_m \alpha_{n-m} \cdot \alpha_m
\end{equation}

y
\begin{equation}
  \tilde{L}_n=\frac 1 2 \sum_m \tilde{\alpha}_{n-m} \cdot \tilde{\alpha}_m.
\end{equation}

Puesto que $M^2=-p_\mu p^\mu$.

La cuantización conduce a la fórmula de masas 
\begin{equation}
  M^2=\frac{4}{\alpha}\sum_{i=1}^{d-2}\sum_{n>0}\qty(\alpha^i_{-n}\alpha^i_n - \frac{d-2}{24})=
  \frac{4}{\alpha}\sum_{i=1}^{d-2}\sum_{n>0}\qty(\tilde{\alpha}^i_{-n}\tilde{\alpha}^i_n -\frac{d-2}{24})
\end{equation}

Es importante ver que hay una expresión en términos de modos moviéndose a la derecha
y otra con modos moviéndose a la izquierda.

\section{Cuantización}

\todo{Cuantización en el cono de luz, espectro, fórmula de masas, path integral}

