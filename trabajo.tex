\documentclass[secnumarabic,nobalancelastpage,amsmath,amssymb,10pt,
nofootinbib]{memoir}
%\documentclass[aps,secnumarabic,nobalancelastpage,amsmath,amssymb,
%nofootinbib]{revtex4}

% secnumarabic is a particularly nice way of identifying sections by
% number to aid electronic review and commentary.

\usepackage[spanish, activeacute]{babel} %Definir idioma español
\usepackage[utf8]{inputenc} %Codificacion utf-8

\usepackage{physics}
\usepackage{graphics}      % standard graphics specifications
\usepackage{graphicx}      % alternative graphics specifications
\usepackage{longtable}     % helps with long table options
\usepackage{url}           % for on-line citations
\usepackage{bm}            % special 'bold-math' package
\usepackage{hyperref}

\usepackage{amsfonts}
\usepackage{booktabs}
\usepackage{physics}
\usepackage{epigraph}
\usepackage{tabularx}
\usepackage{babelbib}
\usepackage[exponent-product=\cdot,range-phrase=--,range-units=single]{siunitx}
%\setlength{\paperheight}{11in}

\newcommand{\reals}{\mathbb{R}}
\newtheorem{defi}{Definición}
\newcommand{\ra}[1]{\renewcommand{\arraystretch}{#1}}
\makeatletter
\def\Dated@name{Fecha: }
\makeatother


\begin{document}

\title{La temperatura de Hagedorn y la entropía de agujeros negros}
\author         {John Liu Anta\\ Universidad de Oviedo}
\date{\today}


\maketitle
\thispagestyle{empty}


\frontmatter

\tableofcontents*



\mainmatter


\renewcommand{\baselinestretch}{1.50}\normalsize

\chapter{Introducción a la teoría de cuerdas}
%ST clásica y cuantización

\section{La partícula relativista}


\section{La cuerda relativista}

La teória de cuerdas parte de considerar que las entidades fundamentales son cuerdas
en vez de partículas. 
Consideremos primero la descripción relativista de una cuerda. En el caso de una
partícula la acción es
\begin{equation}
  S=-m\int ds = -m\int d\tau \sqrt{-\dot{x}_\mu\dot{x}^\mu}=-m\int dt \sqrt{1-\dot {\vec{x}} \cdot \dot {\vec{x}}}.
\end{equation}

donde la posición de la partícula $X^\mu$ está parametrizada por $\tau$.
Una cuerda estará parametrizada por una variable temporal $\tau$ y una variable espacial $\sigma$.
Para una cuerda cerrada con periodicidad $2\pi$, $X^\mu(\tau,\sigma)=X^\mu(\tau,\sigma+2\pi)$.
De forma más compacta, $(\sigma^0,\sigma^1)=(\tau,\sigma)$.

La acción de una partícula es proporcional a la longitud de su línea de universo.
De forma análoga, la acción de una cuerda debería ser proporcional al área de la
\emph{worldsheet}. 
La métrica inducida en la \emph{worldsheet} es la \emph{pull-back} de la métrica de Minkowski
en la \emph{worldsheet},
\begin{equation}
  \gamma_{ab}=\pdv{X^\mu}{\sigma^\alpha}\pdv{X^\nu}{\sigma^\beta}\eta_{\mu\nu}.
\end{equation}

La medida de integración invariante bajo cambios generales de coordenadas más sencillas 
es $d^2\sigma \sqrt{-\det\gamma}$, por lo que la acción es
\begin{equation}
  S_{NG}=-T\int d^2\sigma \sqrt{-\det\gamma}.
\end{equation}

El parámetro $T$ se corresponde con la tensión de la cuerda y se puede expresar como
\begin{equation}
  T=\frac{1}{2\pi\alpha'}
\end{equation}

donde $\alpha'$ es la pendiente de Regge.
Para cuantizar la teoría, la raíz cuadrada es problemática, por lo que se introduce
el campo $h$ definido sobre la \emph{worldsheet} en la llamada acción de Polyakov
\begin{equation}
  S_P=-T\int d^2  \sqrt{-h}h^{\alpha\beta}\partial_\alpha X^\mu \partial_\beta X_\mu.
\end{equation}

Este campo $h$ se comporta como una métrica en dos dimensiones y queda fijada por las
ecuaciones de movimiento
\begin{equation}
  h_{\alpha\beta}=2f(\sigma)\partial_\alpha X^\mu \partial_\beta X_\mu
\end{equation}

donde $f(\sigma)$ es una función cualquiera.

%Simetría diffxWeyl y conformal transormations.

Debido a la simetría gauge de la teoría, la cuantización no es directa.
Para obtener directamente una teoría unitaria, se cuantizan solo los grados de libertad 
físicos. Como contrapartida, se pierde la invariancia de Lorentz explícita.

Escogamos las coordenadas en el cono de luz
\begin{equation}
  \sigma^\pm=\tau\pm\sigma
\end{equation}

y
\begin{equation}
  X^\pm=\frac{1}{\sqrt 2} (X^0 \pm X^{d-1}).
\end{equation}

Las ecuaciones de movimiento llevan a que 
\begin{equation}
  X^+=x^+\alpha p^+ \tau
\end{equation}

y
\begin{equation}
  X^-=X^-_L(\sigma^+)+X^-_R(\sigma^-)
\end{equation}

donde 
\begin{equation}
  X^-_L(\sigma^+)=\frac 1 2 x^- + \frac 1 2 \alpha p^- \sigma^+ + i\sqrt{\frac \alpha 2}
  \sum_{n\neq0} \frac 1 n \tilde{\alpha}^-_n e^{-in\sigma^+}
\end{equation}

y
\begin{equation}
  X^-_R(\sigma^+)=\frac 1 2 x^- + \frac 1 2 \alpha p^- \sigma^- + i\sqrt{\frac \alpha 2}
  \sum_{n\neq0} \frac 1 n \alpha^-_n e^{-in\sigma^-}
\end{equation}

Definiendo
\begin{equation}
  L_n=\frac 1 2 \sum_m \alpha_{n-m} \cdot \alpha_m
\end{equation}

y
\begin{equation}
  \tilde{L}_n=\frac 1 2 \sum_m \tilde{\alpha}_{n-m} \cdot \tilde{\alpha}_m.
\end{equation}

Puesto que $M^2=-p_\mu p^\mu$.

La cuantización conduce a la fórmula de masas 
\begin{equation}
  M^2=\frac{4}{\alpha}\sum_{i=1}^{d-2}\sum_{n>0}\qty(\alpha^i_{-n}\alpha^i_n - \frac{d-2}{24})=
  \frac{4}{\alpha}\sum_{i=1}^{d-2}\sum_{n>0}\qty(\tilde{\alpha}^i_{-n}\tilde{\alpha}^i_n -\frac{d-2}{24})
\end{equation}

Es importante ver que hay una expresión en términos de modos moviéndose a la derecha
y otra con modos moviéndose a la izquierda.

\section{Cuantización}


\chapter{Termodinámica de cuerdas}

\section{Cálculo de la densidad de estados para una cuerda excitada}

\section{Modelo random walk}
Intuitivamente, la masa (energía) de una cuerda es su densidad $\sigma$ por la longitud $L$.
En unidades naturales, la masa tiene unidades inversas a la longitud, por lo que si pensamos
que la cuerda está dividida en segmentos de longitud $l_s$, 
\begin{equation}
  E\sim  \frac{L}{l_s^2}.
\end{equation}

Cada segmento podrá tener una orientación arbitraria. Para simplificar los cálculos,
supongamos que toma direcciones ortogonales. Entonces el número de orientaciones posibles
de cada segmento es $2n$, donde $n$ es el número de dimensiones espaciales accesibles.
El número de segmentos es $L/l_s$.
Por tanto, el número de microestados para una energía dada es
\begin{equation}
  \omega(E)=(2n)^{(L/l_s)}\propto e^{l_s E \ln (n)}.
\end{equation}

La entropía de la cuerda es $S\propto El_s$, con temperatura $\beta_H\sim l_S$.

Se puede calcular que el radio del random walk es $R=\expval{r}\sim E^{1/2}$. Esto
nos dice que cuerdas muy largas estarán contenidas en un volumen pequeño $V\sim E^{D/2}$.

Para una cuerda cerrada, habría que dividir el número de microestados por el 
volumen de walk, multiplicar por el volumen accesible y dividir por la longitud de la cuerda.
Entonces,
\begin{equation}
  \omega(E)\sim V \frac{e^{\beta_H E}}{E^{1+D/2}}
\end{equation}

El cálculo de la extensión de una cuerda altamente excitada con respecto a su centro de masas 
conduce a que $R^2_i$ es proporcional a la suma de las inversas de los números oscilatorios para
la dirección $i$.
Por ello, para un nivel energético $N$ fijo, $(\alpha^i_{-1})^N\ket{0}$ tendría la mayor extensión en la dirección $i$, $\sim N$ y
$\alpha^i_N\ket{0}$ tendría la menor extensión.
Tomando el promedio sobre todas las direcciones espaciales, $R^2\sim (d-1)\sqrt{N}$, siendo
$d-1$ el número de direcciones espaciales. Volvemos a  observa la importante peculariedad $R\sim \sqrt{L}$,
por lo que las cuerdas largas tiende a enmarañarse.

\section{Coalescencia multicuerda}

\section{Termodinámica de cuerda en espacio plano}

%Random walk de una cuerda. Cálculo de la densidad de estados.
%Coalescencia de cuerdas. Microcanónico. Macrocanónico.
%Termodinámica de cuerdas. Energía libre. Invariancia modular explícita. Escalar termal.

La función de partición bosónica en un espectro discreto es $z=\prod_i 1/(1-\exp(-\beta E_i))$.
La energía libre viene dada por $F=-1/\beta \ln z=\beta\sum_i \ln(1-\exp(-\beta E_i))$.
En el caso continuo, donde $E=E(k)$, reemplazamos el sumatorio por $V\int d^{d-1}k/(2\pi)^{d-1}$.
Por tanto,
\begin{equation}
  F=\frac{V}{\beta}\int \frac{d^{d-1}k}{(2\pi)^{d-1}}\ln(1-\exp(-\beta E))
\end{equation}

Mediante el desarrollo de Taylor $\ln(1-x)=-\sum^\infty_{r=1} x^r/r$
\begin{equation}
  F=-\frac{V}{\beta}\sum^\infty_{r=1} \frac 1 r\int \frac{d^{d-1}k}{(2\pi)^{d-1}}\exp(-r\beta E)
\end{equation}

Aplicando la identidad
\begin{equation}
  \frac 1 r \exp(-\beta r E)=\frac{\beta}{\sqrt{2\pi}}\int_0^\infty \frac{ds}{s^{3/2}}\exp\qty(-\frac{E^2s}{2}-\frac{r^2\beta^2}{2s})
\end{equation}

y como $E^2=k^2+m^2$

\begin{equation}
  F=-\frac{V}{\sqrt{2\pi}}\sum^\infty_{r=1} \int_0^\infty \frac{ds}{s^{3/2}} \exp \qty(-\frac{m^2s}{2} -\frac{r^2\beta^2}{2s})
  \int \frac{d^{d-1}k}{(2\pi)^{d-1}}\exp\qty(-\frac{k^2s}{2})
\end{equation}

La última integral tiene como resultado $(2\pi s)^{\frac{d-1}{2}}$, por lo que
\begin{equation}
  F=-V \int_0^\infty \frac{ds}{s(2\pi s)^{d/2}}\sum^\infty_{r=1} \exp \qty(-\frac{m^2s}{2}- \frac{r^2\beta^2}{2s})
\end{equation}

Por otro lado, la función de partición de estado multicuerda es $Z=e^z=-\beta F$. 
Entonces podemos hacer la asociación de la energía libre con integral de camino de 
una partícula tras aplicar una rotación de Wick en la dimensión temporal y compactificando el
tiempo con periodo $\beta$
\begin{equation}
  Z=-\beta F= \int_0^\infty \frac{ds}{2s}\sum^{'\infty}_{w=-\infty} \int \mathcal{D}x 
  \exp\qty(-\frac 1 2 \int_0^s d\tau \qty[\qty(\pdv{X^\mu}{t})^2+m^2])
\end{equation}

Se llega al mismo resultado calculando la función de partición de una partícula  si identificamos 
el parámetro $s$ con el tiempo propio  de la partícula y compactificamos 
la coordenada $X^0$ con periodo $\beta$. El número de vueltas que da la
partícula en la dimensión temporal es $w$. La función de partición se suma sobre
todas las vueltas posibles y se integra en todos los tiempos propios.

En una teoría de cuerdas bosónica, $d=26$ y la energía libre se obtendría sumando las
energías libres para tdo el espectro posible, con masas
\begin{equation}
  m^2=\frac{2}{\alpha'}(N+\bar N-2)
\end{equation}

Puesto que se tiene que cumplir el level-matching $N=\bar N$, habrá que incluir en la 
integral la expresión de la delta de Kronecker
\begin{equation}
   \delta_{N\bar N}=\int_{-1/2}^{1/2}d\tau_1 \exp(2\pi i\tau_1 (N-\bar N))
\end{equation}

Entonces
\begin{equation}
  F=\sum_i \delta_{N_i \bar N_i} F(N_i,\bar N_i)
\end{equation}

donde $i$ recorre todas las posibles combinaciones de $N_i$ y $\bar N_i$.
Haciendo el cambio de variable $s=2\pi\alpha'\tau_2$
\begin{equation}
  F=-V \int_0^\infty \frac{d\tau_2}{\tau_2(4\pi^2\alpha'\tau_2)^{d/2}}\frac 1 2\sum_{r=-\infty}^{'\infty} 
  \sum_i e^{-(N_i+\bar N_i -2)2\pi\tau_2} e^{-\frac{r^2\beta^2}{4\pi\alpha'\tau_2}}\int_{-1/2}^{1/2} d\tau_1 e^{2\pi i\tau_1(N_i-\bar N_i)}
\end{equation}

Definiendo $\tau=\tau_1+i\tau_2$ y $q=e^{2\pi i\tau_1}$, se introduce la $\eta$ de 
Dedekind como
\begin{equation}
  \eta(\tau)=q^{1/24}\prod_{n=1}^{\infty} (1-q^n)
\end{equation}

Se llega a
\begin{equation}
  F=-V \sum_{r=-\infty}^{'\infty} \int_0^\infty\frac{d\tau_2}{2\tau_2} \int_{-1/2}^{1/2} d\tau_1  \frac{1}{(4\pi^2\alpha'\tau_2)^{d/2}}
  \abs{\eta(\tau)}^{-2d+4}\exp\qty(-\frac{r^2\beta^2}{4\pi\alpha'\tau_2})
\end{equation}

El mismo resultado se obtiene calculla la integral de camino en el toro
\begin{equation}
  Z= \int_0^\infty\frac{d\tau_2}{2\tau_2} \int_{-1/2}^{1/2} d\tau_1 \Delta_{FP} \int \mathcal Dx
  \exp\qty(-\frac{1}{4\pi\alpha'}\int d^2\sigma \sqrt h h^{\alpha\beta}\partial_\alpha X^\mu \partial_\beta X_\mu)
\end{equation}

donde la región de 

Se ha llegado a una expresión cuyo integradndo no es modular invariante. Es decir, 
$\tau\to\tau+1$ y $\tau\to1/\tau$ no son invariancias.
Para conseguir la invariancia modular se restringe el dominio de integración al dominio
fundamental y se modifica el integrado, añadiendo el número cuántico $w$ sobre el que 
se suma, de modo que para $d=26$ se tiene
\begin{equation}
  F=-V \sum_{r,w=-\infty}^{\infty} \int_\mathcal{F}  \frac{d\tau_1d\tau_2}{2\tau_2}   \frac{1}{(4\pi^2\alpha'\tau_2)^{13}}
  \abs{\eta(\tau)}^{-48}\exp\qty(-\frac{\abs{r^2-w\tau}^2\beta^2}{4\pi\alpha'\tau_2})
\end{equation}

Haciendo la suma, se llega a
\begin{equation}
  Z=-\beta F=\int_{\matchal F}\frac{d\tau_1d\tau_2}{2\tau_2}\Tr\qty(q^{L_0}\bar q^{\bar L_0})
\end{equation}

donde
\begin{align}
  L_0&=\alpha'\frac{p^2}{4}+\frac{\alpha'}{4}\qty(\frac{2\pi n}{\beta}+\frac{w\beta}{2\pi\alpha'})^2+N\\ 
 \bar L_0&=\alpha'\frac{p^2}{4}+\frac{\alpha'}{4}\qty(\frac{2\pi n}{\beta}-\frac{w\beta}{2\pi\alpha'})^2+\bar N
\end{align}


%\section{Cuerdas en background estático}
%Supongamos una cuerda en presencia de un background $G_{\mu\nu}$ independiente del tiempo.
%La función de partición calculada como la integral de camino en el toro es
%\begin{equation}
%  Z= \int_0^\infty\frac{d\tau_2}{2\tau_2} \int_{-1/2}^{1/2} d\tau_1 \Delta_{FP} \int \mathcal Dx \sqrt G
%  \exp\qty(-\frac{1}{4\pi\alpha'}\int d^2\sigma \sqrt h h^{\alpha\beta}\partial_\alpha X^\mu \partial_\beta X^\nu G_{\mu\nu})
%\end{equation}
%
%La métrica de la worldsheet


\chapter{QFT en espacio curvo}

\section{Agujeros negros}
El caso más sencillo de agujero negro es el agujero negro de Schwarzschild, que
describe una distribución de masa con simetría esférica y estática. La métrica asociada
es 
\begin{equation}
  ds^2= \qty(1-\frac{2MG}{r})dt^2-\qty(1-\frac{2MG}{r})^{-1}dr^2-r^2d\Omega^2.
\end{equation}

Observamos que en el radio de Schwarzschild $r_S=2MG$ la componente temporal de la 
métrica se anula mientras que la componente radial va a infinito.
Se denomina horizonte de sucesos a la esfera con radio $r_S$ centrada en $r=0$.

Para estudiar la física cerca del horizonte es más conveniente emplear las coordenadas
de Rindler. Para ello sustituimos la coordenada radial por la distancia propia hasta
el horizonte
\begin{equation}
  \rho(r)=\int_0^r dr' \sqrt{g_{rr}(r')}.
\end{equation}

La métrica se transforma en 
\begin{equation}
  ds^2=\qty(1-\frac{2MG}{r(\rho)})dt^2-d\rho^2-r(\rho)^2d\Omega^2.
\end{equation}

A distancias próximas al horizonte $r\approx r_S$, $\rho\approx 2\sqrt{2MG(r-2MG)}$
\begin{equation}
  ds^2\approx \rho^2\qty(\frac{dt}{4MG})^2 -d\rho^2-r(\rho)^2d\Omega^2.
\end{equation}

Cerca de la coordenada azimutal $\theta=0$, podemos reemplzar las coordenadas
angulares $\theta,\phi$ por las coordenadas cartesianas $x,y$. Definiendo
también un tiempo adimensional $\omega=t/(4MG)$ obtenemos la métrica de Rindler
\begin{equation}
  ds^2=\rho^2d\omega^2 -d\rho^2 -dx^2-dy^2.
\end{equation}

Esta métrica describe en realidad un espacio de Minkowski en coordenadas hiperbólicas.
Si escogemos la transformación
\begin{align}
  T&=\rho \sinh \omega, \\
  Z&=\rho \cosh \omega.
\end{align}
la métrica es
\begin{equation}
  ds^2=dT^2-dZ^2-dx^2-dy^2.
\end{equation}

Es conveniente definir un observador fijo para cada punto del espacio, llamado
FIDO. Estos observadores para mantenerse
en reposo dentro del campo gravitatorio necesitan una fuerza que los mantenga
en su sitio. Para los FREFOs, su aceleración es, para $\rho<<MG$ es aproximadamente $1/\rho$.

Si dividimos el espacio de Minkowski en cuatro regiones, el espacio de Rindler ocupa
la región I. Los FREFOs emplean el las coordenadas del (T,Z,x,y), mientras que
los FREFOS emplean $(\omega, \rho, x, y)$. El horizonte se sitúa en $T=Z=0$, o
en coordenadas de Rindler, $\rho=0$.
Gráficamente, se observa que una translación espacial de $\omega$ para un equivale
a un boost en el espacio de Minkowski.

Al espacio de Rindler solo le llega información de I y IV, pero la región II está 
desconectada causalmente por culpa del horizonte.
Además, un partícula que pase de IV a I será vista como proveniente del tiempo 
$\omega=-\infty$, por lo que se pueden interpretar como condiciones iniciales.

Estudiamos un campo escalar masivo cuántico $\xi$ en un espacio de Rindler.
La evolución temporal de un sistema cualquiera viene dada por un hamiltoniano. En el
espacio de Rindler la evolución en $\omega$ se determina mediante el hamiltoniano de
Rindler, cuya expresión en función del tensor energía-momento es
\begin{equation}
  H_R=\int_0^\infty d\rho dxdy \rho T^{00} (\rho, x, y).
\end{equation}

En el caso de un campo escalar masivo sometido a un potencial $V$, la densidad de energía
es
\begin{equation}
  T^{00}=\frac{\Pi^2}{2}+\frac{1}{2}\qty(\nabla \xi)^2+V(\xi).
\end{equation}

Entonces el hamiltoniano es
\begin{equation}
  H_R=\int_0^\infty d\rho dx_\perp \frac{\rho}{2}\qty(\Pi^2+\qty(\pdv{\xi}{\rho})^2+\qty(\pdv{\xi}{x_\perp})^2+2V(\xi)).
\end{equation}

Desde el punto de vista del espacio de Minkowski, $H_R$ es el generador de boosts en 
la dirección Z.

En una teoría cuántica, el hecho que haya correlación entre el campo para dos puntos
del espacio origina un fenómeno interesante en el espacio de Rindler. 
Hemos visto que la región III está desconectada de la región I, por lo que consideramos
al campo en cada región como subsistemas distintos. Pero debido a que el campo en I está
correlacionado con el campo III, están entrelazados entre sí.
Por esta razón, no podemos describir al campo en I como un sistema puro como una
matriz de densidad obtenida tomando la traza parcial en III de la matriz de densidad
del sistema completo.

En general, si el estado asociado a dos subsistemas A y B que no están interactuando 
es $\ket{\Psi}$, su matriz de densidad es
\begin{equation}
  \rho_{AB}=\ket{\Psi}\bra{\Psi}.
\end{equation}

La matriz de densidad que describe el sistema A es
\begin{equation}
  \rho_{A}=\Tr_B \rho_{AB}=\expval{\rho_{AB}}{\beta} .
\end{equation}
 
A cada matriz de densidad le corresponde la entropía de von Neumann
\begin{equation}
  S=-\Tr (\rho_A \ln \rho_A)=-\Sum_j \rho_j \ln \rho_j.
\end{equation}

Donde $\rho_j$ son los valores propios de $\rho_A$.
Esta entropía se debe a que estamos perdiendo información al ignorar el subsistema $B$, que 
está entrelazado con $A$, por lo que también se denomina entropía de entrelazamiento.
La intepretación de $\rho_j$ es que estamos describiendo un sistema cuyo estado no 
conocemos, pero sabemos que a cada estado le corresponde una probabilidad $\rho_j$.

Por otro lado un sistema termodinámico en equilibrio a temperatura $\beta$ tiene 
asociado la matriz densidad 
\begin{equation}
  \rho=\frac{e^{-\beta H}}{\Tr e^{\beta H}}.
\end{equation}

Retomamos el campo escalar en un espacio de Rindler. 
Para $T=0$, los campos en cada punto forman un conjunto completo de observables que 
conmutan. Denominamos $\xi_L$ a los campos en $Z<0$ y $\xi_R$ si $Z>0$.
En teoría cuántica de campos, un estado se describe como una distribución que 
depende de los campos $\Psi[\xi]=\Psi[\xi_L,\xi_R]$.

Supondremos que $\Psi[\xi_L,\xi_R]$ está en el estado fundamental (vacío) del hamiltoniano
de Minkowski. Para hallar el estado fundamental mediante la integral de camino 
en mecánica cuántica de partículas aplicamos
\begin{align}
  \braket{y,T}{x,0}\sim \int_{X(0)=x,X(T)=y} \mathcal D X e^{-S} \sim 
  \mel{y}{e^{-HT}}{x}=\sum_{n,n'} \ip{y}{n}\mel{n}{e^{-HT}}{n'}\ip{n'}{n}=\\
  =\sum_{n,n'} \Psi_n(y) e^{-E_n T} \bar{\Psi}_n(x)\sim   \Psi_0(y) e^{-E_0 T} \bar{\Psi}_0(x).
\end{align}

Donde hemos aplicado una rotación de Wick $t\to T=it$ por continuación analítica.
Por lo que $\bar{\Psi}_0(x)\sim \int_{X(0)=x} \mathcal D x e^{-S}$.
De forma similar en teoría cuántica de campos 
\begin{equation}
  \Psi[\xi_L,\xi_R]=\frac{1}{\sqrt{Z}}\int_{X^0>0, \xi(X^0=0)=(\xi_L,\xi_R)} e^{-S}
\end{equation}

Para evaluar la integral, tenemos en cuenta que como la translación en $\omega$
equivale a un boost en el espacio de Minkowski, en el espacio euclídeo se traduce
en una rotación.
El ángulo $\theta$ con respecto al eje $Z$ en el plano $(Z,X^0)$ se corresponde con
el tiempo de Rindler $\omega$.

\nocite{*}
\bibliographystyle{babunsrt}
\bibliography{mybib}

\end{document}
