\documentclass[secnumarabic,nobalancelastpage,10pt,nofootinbib,a4paper]{memoir}
%\documentclass[aps,secnumarabic,nobalancelastpage,amsmath,amssymb,
%nofootinbib]{revtex4}

% secnumarabic is a particularly nice way of identifying sections by
% number to aid electronic review and commentary.


\usepackage[spanish, activeacute]{babel} %Definir idioma español
\usepackage[utf8]{inputenc} %Codificacion utf-8
\usepackage[T1]{fontenc}

\usepackage{graphics}      % standard graphics specifications
\usepackage{graphicx}      % alternative graphics specifications
\usepackage{url}           % for on-line citations
\usepackage{hyperref}
%\usepackage{cleveref}
\usepackage{bm}            % special 'bold-math' package

\usepackage{amsmath,amssymb,amsfonts}
\usepackage{physics}
\usepackage[exponent-product=\cdot,range-phrase=--,range-units=single]{siunitx}

\usepackage{longtable}     % helps with long table options
\usepackage{booktabs}
\usepackage{tabularx}
\usepackage{babelbib}
\usepackage{epigraph}

%\setlength{\paperheight}{11in}
\usepackage{todonotes}
\usepackage{microtype}

\graphicspath{{images/}}

\newcommand{\ra}[1]{\renewcommand{\arraystretch}{#1}}
\setlength{\parindent}{0pt}


\begin{document}

\listoftodos

\title{La temperatura de Hagedorn y la entropía de agujeros negros}
\author         {John Liu Anta
                \\ Universidad de Oviedo}
\date{\today}

\maketitle
\thispagestyle{empty}

\frontmatter

\tableofcontents*

\mainmatter

\renewcommand{\baselinestretch}{1.50}\normalsize

\chapter*{Resumen}

Este trabajo tiene como objetivo entender cómo la temperatura de Hagedorn, 
caractéristica de todas las teorías de cuerdas, se ve afectada por la presencia de agujeros
negros.

La base de este trabajo es la tesis \cite{}.
En el capítulo 1 se introducen los conceptos fundamentales de teoría de cuerdas.
En el capítulo 2, se analiza la termodinámica de una gas de cuerdas en
espacio de Minkowski, donde aparece la temperatura de Hagedorn.
El capítulo 3 trata dos aspectos de considerar una teoría cuántica de campos en 
sistemas de refencia no inerciales: la temperatura de Unruh y la temperatura de Hawking.
El concepto de partícula es dependiente del observador.
El estudio de cuerdas en espacios curvos se realiza en el capítulo 4. 



\ldots



\chapter{Introducción a la teoría de cuerdas}


\section{Motivación}

\todo{Describir brevemente objetivos, utilidad y estado actual}


\todo{Convenio}

\section{La cuerda relativista}

\todo{Métrica, linea de universo}

La teoría de cuerdas parte de considerar que las entidades fundamentales son cuerdas
en vez de partículas. 


La trayectoria $\mathbf x(t)$ de una partícula satisface que su acción $S$ es un extremal.
Coloquialmente, esto significa que la variación de la acción a primer orden es nula bajo
variaciones pequeñas de la trayectoria, supuestas fijas las posiciones iniciales y finales.
La acción de una partícula libre es
\begin{equation}
  S=-m\int dt \sqrt{1-\dot {\vec{x}} \cdot \dot {\vec{x}}},
\end{equation}
donde $\dot x = \dv{x}{t}$.

Para que en la acción aparezca el tiempo y la posición en igualdad de condiciones,
parametrizamos el tiempo y la posición por el tiempo propio $\tau$. 
El tiempo propio es el tiempo que mediría un reloj que se moviese con la partícula.
Con esta transformación, la acción es
\begin{equation}
 S = -m\int d\tau \sqrt{-\dv{x_\mu}{\tau}\dv{x^\mu}{\tau}}=-m\int d\tau .
\end{equation}


Generalizaremos la acción de una partícula libre a una cuerda libre.
Una cuerda está parametrizada por una variable temporal $\tau$ y una variable espacial $\sigma$, adimensional.
De forma más compacta, $(\sigma^0,\sigma^1)=(\tau,\sigma)$. 
La trayectoria de la cuerda en el espacio-tiempo genera un superficie llamada \emph{worldsheet}.
Las coordenadas en la worldsheet determinan un punto del espacio-tiempo $X^\mu$, también llamado
espacio \emph{target} para evitar confusión.

Para una cuerda cerrada con periodicidad $2\pi$, identificamos $X^\mu(\tau,\sigma)=X^\mu(\tau,\sigma+2\pi)$.

La acción de una partícula es proporcional a la longitud de su línea de universo.
De forma análoga, la acción de una cuerda debería ser proporcional al área de la
worldsheet.
Para poder medir el área, es necesario definir una métrica $\gamma_{\alpha\beta}$ en la worldsheet.
La forma natural de definir una métrica en una superficie, a partir de la métrica del espacio que
contiene a la superficie, es mediante el concepto de \emph{pull-back}.
En este caso, la métrica en la worldsheet el pull-back de la métrica de Minkowski
\begin{equation}
  \gamma_{ab}=\pdv{X^\mu}{\sigma^\alpha}\pdv{X^\nu}{\sigma^\beta}\eta_{\mu\nu}.
\end{equation}

La propiedad esencial de la métrica $\gamma_{\alpha\beta}$ es que la distancia entre dos puntos próximos 
coincide con la calcula con la métrica $\eta_{\mu\nu}$, ya que
\begin{equation}
  g_{\mu\nu}(X(\sigma)) dX^\mu(\sigma)dX^\nu(\sigma) = \gamma_{\alpha\beta} d\sigma^\alpha d\sigma^\beta.
\end{equation}



La medida de integración invariante bajo cambios generales de coordenadas más sencilla 
es $d^2\sigma \sqrt{-\det\gamma}$, por lo que la acción, llamada de Nambu-Goto, es
\begin{equation}
  S_{NG}=-T\int d^2\sigma \sqrt{-\det\gamma}.
\end{equation}

El parámetro $T$ se corresponde con la tensión de la cuerda y se puede expresar como
\begin{equation}
  T=\frac{1}{2\pi\alpha'},
\end{equation}
donde $\alpha'$ es la pendiente de Regge.

A la hora de cuantizar la teoría, la raíz cuadrada es problemática, por lo que se introduce
un campo tensorial $h$ definido sobre la worldsheet en la llamada acción de Polyakov
\begin{equation}
  S_P=-\frac{1}{4\pi\alpha'}\int d^2\sigma  \sqrt{-h}h^{\alpha\beta}\partial_\alpha X^\mu \partial_\beta X_\mu.
\end{equation}

Este campo $h$ se comporta como una métrica en dos dimensiones y queda fijada por las
ecuaciones de movimiento
\begin{equation}
  h_{\alpha\beta}=2f(\sigma)\partial_\alpha X^\mu \partial_\beta X_\mu
\end{equation}
donde $f(\sigma)$ es una función cualquiera. La libertad de poder escoger $f(\sigma)$
es una simetría gauge


\section{Cuantización}

Debido a la simetría gauge de la teoría, la cuantización no es directa.
Hay grados de libertad que no son físicos y en algún momento hemos de deshacernos de ellos.
Para obtener directamente una teoría unitaria, cuantizamos solo los grados de libertad 
físicos, buscando primero las soluciones clásicas. 
Como contrapartida, perdemos la invariancia de Lorentz explícita.

Definimos las coordenadas en el cono de luz
\begin{equation}
  \sigma^\pm=\tau\pm\sigma
\end{equation}

y
\begin{equation}
  X^\pm=\frac{1}{\sqrt 2} (X^0 \pm X^{d-1}).
\end{equation}

Gracias a la simetría gauge, imponemos $h_{\alpha\beta}=\eta_{\alpha\beta}$, de forma que 
la acción de Polyakov es
\begin{equation}
  S = -\frac{1}{4\pi\alpha'} \int d^2\sigma \partial_\alpha X^\mu \partial^\alpha X_\mu,
\end{equation}
que da lugar a la ecuaciones de movimiento de ondas libres
\begin{equation}
  \partial_\alpha \partial^\alpha X^\mu=0.
\end{equation}

La solución general de la ecuación de movimiento para $X^+$ se descompone
en una onda moviéndose hacia la izquierda $X^+_L$ y otra hacia la derecha $X^+_R$,
\begin{equation}
   X^+ =X^+_L(\sigma^+) + X^+_R(\sigma^-).
\end{equation}

Teniendo en cuenta la periodicidad en $\sigma$, la expansión de Fourier conduce a
\begin{equation}
  \begin{gathered}
    X^+_L=\frac 1 2 x^+ + \frac 1 2 \alpha' p^+ \sigma^+ + i\sqrt{\frac{\alpha'}{2}}\sum_{n\neq 0} \frac{1}{n}\tilde \alpha^+_n e^{-in\sigma^+},\\
    X^+_R=\frac 1 2 x^+ + \frac 1 2 \alpha' p^+ \sigma^- +i\sqrt{\frac{\alpha'}{2}}\sum_{n\neq 0} \frac{1}{n}\alpha^+_n e^{-in\sigma^-}.
  \end{gathered}
\end{equation}
Donde $x^+$ y $p^+$ se corresponden con la posición y el momento del centro de masas, respectivamente.


Debido a la invariancia baja reparametrizaciones, escogemos
\begin{equation}
  X^+_L=\frac 1 2 x^+ + \frac 1 2 \alpha' p^+ \sigma^+,
\end{equation}
y
\begin{equation}
  X^+_R=\frac 1 2 x^+ + \frac 1 2 \alpha' p^+ \sigma^+.
\end{equation}

Por tanto, 
\begin{equation}
  X^+ = x^+ + \alpha' p^+ \tau.
\end{equation}

Con esta elección gauge, la solución $X^-$ que determinada casi por completo.
La ecuación de movimiento para la métrica $h$ se obtiene variando a acción
\begin{equation}
  \delta S = \frac{1}{4\pi\alpha'}\int d^2\sigma \delta h^{\alpha\beta}
  \qty(
  \sqrt{-h}\partial_\alpha X^\mu \partial_\beta X_\mu -
  \frac 1 2 \sqrt{-h}h_{\alpha\beta}h^{\rho\sigma} \partial_\rho X^\mu \partial_\sigma X_\mu).
\end{equation}
Como habíamos elegido $h_{\alpha\beta}=\eta_{\mu\nu}$, la ecuación de movimiento impone
\begin{equation}
  \partial_\alpha X^\mu \partial_\beta X_\mu - \frac 1 2 \eta_{\alpha\beta} \eta^{\sigma\rho} 
  \partial_\rho X^\mu  \partial_\sigma X_\mu = 0.
\end{equation}

En las coordenadas del cono de luz, esto se traduce en
\begin{equation}
  (\partial_+ X)^2 = (\partial_- X)^2 = 0.
  \label{eq:lig}
\end{equation}

Haciendo la descomposición de la solución $X^-$ 
\begin{equation}
  X^-=X^-_L(\sigma^+)+X^-_R(\sigma^-),
\end{equation}
la ecuación \ref{eq:lig} conduce a 
  \begin{align}
    \partial_+ X_L^- = \frac{1}{\alpha'p^+}\sum_{i=1}^{D-2} \partial_+ X^i \partial_+ X^i, \label{eq:lig1}\\
    \partial_- X_R^- = \frac{1}{\alpha'p^-}\sum_{i=1}^{D-2} \partial_- X^i \partial_- X^i. \label{eq:lig2}
  \end{align}

En el desarrollo de Fourier
\begin{equation}
  \begin{gathered}
    X^-_L(\sigma^+)=\frac 1 2 x^- + \frac 1 2 \alpha' p^- \sigma^+ + i\sqrt{\frac{ \alpha'}{ 2}}
    \sum_{n\neq0} \frac 1 n \tilde{\alpha}^-_n e^{-in\sigma^+} \\                             
    X^-_R(\sigma^+)=\frac 1 2 x^- + \frac 1 2 \alpha' p^- \sigma^- + i\sqrt{\frac{ \alpha'}{ 2}}
    \sum_{n\neq0} \frac 1 n \alpha^-_n e^{-in\sigma^-},
  \end{gathered}
\end{equation}
las constantes $p^-$, $\tilde \alpha^-_n$ y $\alpha^-_n$ quedan determinadas por las ecuaciones
\ref{eq:lig1} y \ref{eq:lig2}.

El momento $p^-$ se puede expresar a través de $\alpha^i_n$ y de $\tilde \alpha^i_n$ como
\begin{equation}
  p^- = \frac{1}{\alpha' p^+}\sum_{i=1}^{D-2} \qty(\frac{1}{2}\alpha'p^i p^i +\sum_{n\neq0}\alpha_n^i\alpha_{-n}^i) 
   = \frac{1}{\alpha' p^+}\sum_{i=1}^{D-2} \qty(\frac{1}{2}\alpha'p^i p^i +\sum_{n\neq0}\tilde \alpha_n^i\tilde \alpha_{-n}^i) .
\end{equation}

La masa de la cuerda es por tanto
\begin{equation}
  M^2=-p_\mu p^\mu = 2p^+p^- - \sum_{i=1}^{D-2} p^i p^i = 
  \frac{2}{\alpha'}\sum_{i=1}^{D-2} \sum_{n\neq 0} \alpha_{-n}^i \alpha_n^i
  =\frac{2}{\alpha'}\sum_{i=1}^{D-2} \sum_{n\neq 0}\tilde \alpha_{-n}^i \tilde\alpha_n^i.
\end{equation}

Hemos obtenido la solución clásica mediante los $2(D-2)$ modos de oscilación transversos 
$\alpha^i_n$ y $\tilde \alpha_n^i$ y las constantes $x^i$, $p^i$,$p^+$ y $x^-$.
La cuantización consiste en promover los grados de libertad a operadores que satisfacen
una reglas de conmutación. Estas reglas son
\begin{equation}
  \begin{gathered}
    \qty[x^i,p^j] = i\delta^{ij}\qquad , \qquad [x^-,p^+] = -i,\\
    [\alpha_n^i,\alpha_m^j]
  \end{gathered}
\end{equation}

\begin{equation}
  M^2 = \frac{4}{\alpha'}\qty(\sum_{i=1}^{D-2} \sum_{n>0} \alpha_{-n}^i\alpha_n -a)
   = \frac{4}{\alpha'}\qty(\sum_{i=1}^{D-2} \sum_{n>0} \tilde \alpha_{-n}^i\tilde \alpha_n -a)
\end{equation}

La cuantización conduce a la fórmula de masas 
\begin{equation}
  M^2=\frac{4}{\alpha}\sum_{i=1}^{d-2}\sum_{n>0}\qty(\alpha^i_{-n}\alpha^i_n - \frac{d-2}{24})=
  \frac{4}{\alpha}\sum_{i=1}^{d-2}\sum_{n>0}\qty(\tilde{\alpha}^i_{-n}\tilde{\alpha}^i_n -\frac{d-2}{24})
\end{equation}

\subsection{Espectro de una cuerda}
\begin{equation}
  M^2 = -\frac{1}{\alpha'}\frac{D-2}{6}
\end{equation}

\begin{equation}
  M^2 = \frac{4}{\alpha'}\qty(1-\frac{D-2}{24})
\end{equation}

Definiendo
\begin{equation}
  L_n=\frac 1 2 \sum_m \alpha_{n-m} \cdot \alpha_m
\end{equation}

\begin{equation}
  \tilde{L}_n=\frac 1 2 \sum_m \tilde{\alpha}_{n-m} \cdot \tilde{\alpha}_m.
\end{equation}

Es importante ver que hay una expresión en términos de modos moviéndose a la derecha
y otra con modos moviéndose a la izquierda.

\subsection{Cuerdas supersimétricas}


\subsection{La integral de camino}

Función beta

Dualidad T



\chapter{Termodinámica de cuerdas}

\section{Cálculo de la densidad de estados para una cuerda excitada}

La fórmula de masas para una cuerda abierta es
\begin{equation}
  \alpha' m^2=-1+N
\end{equation}
y para una cuerda cerrada
\begin{equation}
  \alpha' m^2=-4+2(N+\bar N).
\end{equation}

\todo{Obtención de la densidad de estados en el paso al continuo}
Queremos calcular la densidad de estados de una cuerda en función del número de 
oscilador, 
\begin{equation}
  p(N)\approx \alpha N^ {\frac{d-1}{2}}\exp{\qty(2\pi\sqrt{\frac{N(d-2)}{6})}}.
\end{equation}

Expresando la densidad de estados en función de la masa
\begin{equation}
  p(M)\approx \alpha N^ {\frac{d-1}{2}}\exp{\qty(2\pi\sqrt{\frac{N(d-2)}{6})}}.
\end{equation}

\section{Modelo random walk}

\todo{Revisar planteamiento}
Intuitivamente, la masa (energía) de una cuerda es su densidad lineal $\sigma$ por su longitud $L$.
En unidades naturales, la masa tiene unidades inversas a la longitud, por lo que si pensamos
que la cuerda está dividida en $N$ segmentos de longitud $l_s$, 
\begin{equation}
  E\sim  \frac{N}{l_s} \sim \frac{L}{l_s^2}.
\end{equation}

Cada segmento podrá tener una orientación arbitraria. Para simplificar los cálculos,
supongamos que toma direcciones ortogonales. Entonces el número de orientaciones posibles
de cada segmento es $2n$, donde $n$ es el número de dimensiones espaciales accesibles.
El número de segmentos es $L/l_s$.
Por tanto, el número de microestados para una energía dada es
\begin{equation}
  \omega(E)=(2n)^{(L/l_s)}\propto e^{l_s E \ln (n)}.
\end{equation}

La entropía de la cuerda es $S\propto El_s$, con temperatura $\beta_H\sim l_S$.

Se puede calcular que el radio del random walk es $R=\expval{r}\sim E^{1/2}$. Esto
nos dice que cuerdas muy largas estarán contenidas en un volumen pequeño $V\sim E^{D/2}$.

Para una cuerda cerrada, habría que dividir el número de microestados por el 
volumen de walk, multiplicar por el volumen accesible y dividir por la longitud de la cuerda.
Entonces,
\begin{equation}
  \omega(E)\sim V \frac{e^{\beta_H E}}{E^{1+D/2}}.
\end{equation}

\todo{No necesitamos conocer los detalles de la teoría}
De esta forma, obtenemos la densidad de estados 

El cálculo de la extensión de una cuerda altamente excitada con respecto a su centro de masas 
conduce a que $R^2_i$ es proporcional a la suma de las inversas de los números oscilatorios para
la dirección $i$.
Por ello, para un nivel energético $N$ fijo, $(\alpha^i_{-1})^N\ket{0}$ tendría la mayor extensión en la dirección $i$, $\sim N$ y
$\alpha^i_N\ket{0}$ tendría la menor extensión.
Tomando el promedio sobre todas las direcciones espaciales, $R^2\sim (d-1)\sqrt{N}$, siendo
$d-1$ el número de direcciones espaciales. Volvemos a  observa la importante peculariadad $R\sim \sqrt{L}$,
por lo que las cuerdas largas tiende a enmarañarse.

\section{Coalescencia multicuerda}

\todo{Repaso de las colectividades estadísticas}

\subsection{Formulación microcanónica}


\subsection{Formulación canónica}

\section{Termodinámica de cuerdas en espacio plano}

%Termodinámica de cuerdas. Energía libre. Invariancia modular explícita. Escalar termal.

La función de partición bosónica en un espectro discreto es $z=\prod_i 1/(1-\exp(-\beta E_i))$.
La energía libre viene dada por $F=-1/\beta \ln z=\beta\sum_i \ln(1-\exp(-\beta E_i))$.
En el caso continuo, donde $E=E(k)$, reemplazamos el sumatorio por $V\int d^{d-1}k/(2\pi)^{d-1}$.
Por tanto,
\begin{equation}
  F=\frac{V}{\beta}\int \frac{d^{d-1}k}{(2\pi)^{d-1}}\ln(1-\exp(-\beta E))
\end{equation}

Mediante el desarrollo de Taylor $\ln(1-x)=-\sum^\infty_{r=1} x^r/r$
\begin{equation}
  F=-\frac{V}{\beta}\sum^\infty_{r=1} \frac 1 r\int \frac{d^{d-1}k}{(2\pi)^{d-1}}\exp(-r\beta E)
\end{equation}

Aplicando la identidad
\begin{equation}
  \frac 1 r \exp(-\beta r E)=\frac{\beta}{\sqrt{2\pi}}\int_0^\infty \frac{ds}{s^{3/2}}\exp\qty(-\frac{E^2s}{2}-\frac{r^2\beta^2}{2s})
\end{equation}

y como $E^2=k^2+m^2$

\begin{equation}
  F=-\frac{V}{\sqrt{2\pi}}\sum^\infty_{r=1} \int_0^\infty \frac{ds}{s^{3/2}} \exp \qty(-\frac{m^2s}{2} -\frac{r^2\beta^2}{2s})
  \int \frac{d^{d-1}k}{(2\pi)^{d-1}}\exp\qty(-\frac{k^2s}{2})
\end{equation}

La última integral tiene como resultado $(2\pi s)^{\frac{d-1}{2}}$, por lo que
\begin{equation}
  F=-V \int_0^\infty \frac{ds}{s(2\pi s)^{d/2}}\sum^\infty_{r=1} \exp \qty(-\frac{m^2s}{2}- \frac{r^2\beta^2}{2s})
\end{equation}

Por otro lado, la función de partición de estado multicuerda es $Z=e^z=-\beta F$. 
Entonces podemos hacer la asociación de la energía libre con integral de camino de 
una partícula tras aplicar una rotación de Wick en la dimensión temporal y compactificando el
tiempo con periodo $\beta$
\begin{equation}
  Z=-\beta F= \int_0^\infty \frac{ds}{2s}\sum^{'\infty}_{w=-\infty} \int \mathcal{D}x 
  \exp\qty(-\frac 1 2 \int_0^s d\tau \qty[\qty(\pdv{X^\mu}{t})^2+m^2])
\end{equation}

Se llega al mismo resultado calculando la función de partición de una partícula  si identificamos 
el parámetro $s$ con el tiempo propio  de la partícula y compactificamos 
la coordenada $X^0$ con periodo $\beta$. El número de vueltas que da la
partícula en la dimensión temporal es $w$. La función de partición se suma sobre
todas las vueltas posibles y se integra en todos los tiempos propios.

En una teoría de cuerdas bosónica, $d=26$ y la energía libre se obtendría sumando las
energías libres para todo el espectro posible, con masas
\begin{equation}
  m^2=\frac{2}{\alpha'}(N+\bar N-2)
\end{equation}

Puesto que se tiene que cumplir el level-matching $N=\bar N$, habrá que incluir en la 
integral la expresión de la delta de Kronecker
\begin{equation}
   \delta_{N\bar N}=\int_{-1/2}^{1/2}d\tau_1 \exp(2\pi i\tau_1 (N-\bar N))
\end{equation}

Entonces
\begin{equation}
  F=\sum_i \delta_{N_i \bar N_i} F(N_i,\bar N_i)
\end{equation}

donde $i$ recorre todas las posibles combinaciones de $N_i$ y $\bar N_i$.
Haciendo el cambio de variable $s=2\pi\alpha'\tau_2$
\begin{equation}
  F=-V \int_0^\infty \frac{d\tau_2}{\tau_2(4\pi^2\alpha'\tau_2)^{d/2}}\frac 1 2\sum_{r=-\infty}^{'\infty} 
  \sum_i e^{-(N_i+\bar N_i -2)2\pi\tau_2} e^{-\frac{r^2\beta^2}{4\pi\alpha'\tau_2}}\int_{-1/2}^{1/2} d\tau_1 e^{2\pi i\tau_1(N_i-\bar N_i)}
\end{equation}

Definiendo $\tau=\tau_1+i\tau_2$ y $q=e^{2\pi i\tau_1}$, se introduce la $\eta$ de 
Dedekind como
\begin{equation}
  \eta(\tau)=q^{1/24}\prod_{n=1}^{\infty} (1-q^n)
\end{equation}

Se llega a
\begin{equation}
  F=-V \sum_{r=-\infty}^{'\infty} \int_0^\infty\frac{d\tau_2}{2\tau_2} \int_{-1/2}^{1/2} d\tau_1  \frac{1}{(4\pi^2\alpha'\tau_2)^{d/2}}
  \abs{\eta(\tau)}^{-2d+4}\exp\qty(-\frac{r^2\beta^2}{4\pi\alpha'\tau_2})
\end{equation}

El mismo resultado se obtiene calculando la integral de camino en el toro
\begin{equation}
  Z= \int_0^\infty\frac{d\tau_2}{2\tau_2} \int_{-1/2}^{1/2} d\tau_1 \Delta_{FP} \int \mathcal Dx
  \exp\qty(-\frac{1}{4\pi\alpha'}\int d^2\sigma \sqrt h h^{\alpha\beta}\partial_\alpha X^\mu \partial_\beta X_\mu)
\end{equation}

donde la región de 

Se ha llegado a una expresión cuyo integrando no es modular invariante. Es decir, 
$\tau\to\tau+1$ y $\tau\to1/\tau$ no son invariancias.
Para conseguir la invariancia modular se restringe el dominio de integración al dominio
fundamental y se modifica el integrado, añadiendo el número cuántico $w$ sobre el que 
se suma, de modo que para $d=26$ se tiene
\begin{equation}
  F=-V \sum_{r,w=-\infty}^{\infty} \int_\mathcal{F}  \frac{d\tau_1d\tau_2}{2\tau_2}   \frac{1}{(4\pi^2\alpha'\tau_2)^{13}}
  \abs{\eta(\tau)}^{-48}\exp\qty(-\frac{\abs{r^2-w\tau}^2\beta^2}{4\pi\alpha'\tau_2})
\end{equation}

Haciendo la suma, se llega a
\begin{equation}
  Z=-\beta F=\int_{\mathcal F}\frac{d\tau_1d\tau_2}{2\tau_2}\Tr\qty(q^{L_0}\bar q^{\bar L_0})
\end{equation}

donde
\begin{align}
  L_0&=\alpha'\frac{p^2}{4}+\frac{\alpha'}{4}\qty(\frac{2\pi n}{\beta}+\frac{w\beta}{2\pi\alpha'})^2+N\\ 
 \bar L_0&=\alpha'\frac{p^2}{4}+\frac{\alpha'}{4}\qty(\frac{2\pi n}{\beta}-\frac{w\beta}{2\pi\alpha'})^2+\bar N
\end{align}


%\section{Cuerdas en background estático}
%Supongamos una cuerda en presencia de un background $G_{\mu\nu}$ independiente del tiempo.
%La función de partición calculada como la integral de camino en el toro es
%\begin{equation}
%  Z= \int_0^\infty\frac{d\tau_2}{2\tau_2} \int_{-1/2}^{1/2} d\tau_1 \Delta_{FP} \int \mathcal Dx \sqrt G
%  \exp\qty(-\frac{1}{4\pi\alpha'}\int d^2\sigma \sqrt h h^{\alpha\beta}\partial_\alpha X^\mu \partial_\beta X^\nu G_{\mu\nu})
%\end{equation}
%
%La métrica de la worldsheet

\section{Discrepancia entre colectividad microcanónica y canónica}

\todo{Comentar más allá de Hagedorn?}


\chapter{QFT en sistemas acelerados y espacio curvo}
%\todo{Corregir numeracion}
C teoría cuántica de campos (QFT) en sistemas acelerados y espacio con curvatura.

\section{Efecto Unruh}

Una propiedad sorprendente de las teorías cuánticas de campos es que el estado de vacío
puede depender del observador.
Concretamente, el efecto Unruh establece que el vacío para un observador inercial,
visto por un observador con aceleración constante corresponde con un estado térmico a temperatura
\begin{equation}
  T_U=\frac{a}{2\pi k_B}.
\end{equation}

En la deducción del efecto Unruh estudiaremos un campo escalar sin masa $\phi(t,z)$ en un espacio de Minkowski
con una dimensión espacial $t$ y una dimensión espacial $z$.  
\footnote{Es posible derivar el efecto Unruh para teorías más generales empleando la integral
  de camino euclídea, véase \cite{Susskind}}
La ecuación que describe el campo es la ecuación de Klein-Gordon
\begin{equation}
  -\pdv[2]{\phi}{t}+\pdv[2]{\phi}{z}=0.
  \label{eq:kg}
\end{equation}

La solución general de esta ecuación de ondas toma la forma
\begin{equation}
  \phi(x,t)=f(t-x)+g(t+x).
\end{equation}

Las funciones $f$ y $g$ representan dos ondas que se mueven a la velocidad de la luz en sentidos
opuestos. 
Hacemos el cambio de variable a coordenadas nulas
\begin{equation}
  U=t-z,   \qquad V=t+z.
\end{equation}

De este modo, la solución general se expresa como
\begin{equation}
  \phi(U,V)=\phi_U(U)+\phi_V(V).
\end{equation}

Nos interesa expandir la solución en términos de ondas armónicas
\begin{gather}
  \phi_\omega^U(U)=\frac{1}{\sqrt{4\pi\omega}} e^{-i\omega U},  \\
  \phi_\omega^V(V)=\frac{1}{\sqrt{4\pi\omega}} e^{-i\omega V}.
\end{gather}

Puesto que la solución $\phi_V$ está desacoplada de $\phi_U$, basta considerar $\phi_U$ ya que
el tratamiento de $\phi_V$ sería análogo.

La expansión del campo en modos es
\begin{equation}
\phi_U(U) = \int_0^\infty d\omega \qty[a_\omega^U \phi_\omega^\phi (U)+a_\omega^{U*} \phi_\omega^\phi (U)^*].
\end{equation}

El proceso de cuantización canónica consiste en reemplazar en valor del campo en cada
punto del espacio-tiempo $\phi(x,t)$, por un operador $\widehat \phi(x,t)$ que satisfará unas
relaciones de conmutación particulares. 
Esto significa que los coeficientes de la expansión de $\phi_U$ pasan a ser los operadores
$\widehat a_\omega^U$ y $\widehat a_\omega^{U\dagger}$ y por tanto
\begin{equation}
  \widehat \phi_U(U) = \int_0^\infty d\omega \qty[\widehat a_\omega^U\phi_\omega^\phi (U)+\widehat a_\omega^{U\dagger} \phi_\omega^\phi (U)^*].
\end{equation}

El operador $\widehat a_\omega^{U\dagger}$ se denomina operador creación, pues veremos que crea 
partículas de frecuencia $\omega$ y $\widehat a_\omega^U$ se conoce como operador destrucción porque
aniquila partículas de frecuencia $\omega$.
Las relaciones de conmutación que cumplen son
\begin{gather}
  [\widehat a_\omega^{U\dagger},\widehat a_{\omega'}^{U\dagger}]=[\widehat a_\omega^U,\widehat a_{\omega'}^U]=0, \\
  [\widehat a_\omega^U,\widehat a_{\omega'}^{U\dagger}]=\delta(\omega-\omega').
\end{gather}

Todavía no hemos especificado el espacio de Hilbert sobre el que actúa el operador del campo, el cual
se denota por $\mathcal H_\phi$. 
El estado del campo queda especificado descrito por un elemento de $\mathcal H_\phi$.
Como en este caso los modos $U$ y $V$ están desacoplados, podemos considerar independientemente el espacio
de Hilbert asociado a cada uno, $\mathcal H_U$ y $\mathcal H_V$.
El espacio de Hilbert del campo es el producto tensorial de ambos $\mathcal H_\phi=\mathcal H_U\otimes \mathcal H_V$.

%Construcción espacio de Fock
La base del espacio $\mathcal H_U$ se puede construir mediante la representación de Fock. 
Para ello, se define el estado de vacío $\ket{0_U}$, como el estado que no contiene ningún tipo
de partícula, por tanto
\begin{equation}
  a^U_\omega \ket{0_U} = 0.
\end{equation}

Donde omitimos el acento circunflejo de los operadores por comodidad.
Luego procedemos a crear estados con $n_i$ partículas de frecuencia $\omega_i$, mediante
aplicación repetida del operador creación $a_{\omega_i}^{U\dagger}$, con la normalización apropiada
\begin{equation}
  \ket{n_{1,\omega_1}, n_{2,\omega_2},\cdots,n_{N,\omega_N}} = \frac{1}{\sqrt{n_1!n_2!\cdots n_N!}}(a_{\omega_1}^{U\dagger})^{n_1}
  (a_{\omega_2}^{U\dagger})^{n_2}\cdots a_{\omega_N}^{U\dagger})^{n_N} \ket{0_U}.
\end{equation}

Los estados construidos son estados propios del operador número de partículas $N_{\omega_i}^U=a_{\omega_i}^{U\dagger}a_{\omega_i}^U$
con valor propio $n_i$.
La base de $\mathcal H_U$ se obtiene juntando los estados con un número arbitrario de partículas
de todas las frecuencias posibles.
%\todo{Definición exacta? La suma directa es no numerable, al tener que considerar todas las frecuencias?}
%Formalmente, $\mathcal H_U$ es la completitud de la suma directa de 
%\todo{Rigor}
%\begin{equation}
%  \mathcal H_U = \overline{\bigoplus  \otimes \ket{n_i}} .
%\end{equation}

%Expansión en otra base. Transformación Bogoliubov. Vacío.
Supongamos que queremos expandir el campo en otra base de modos $u$
\begin{equation}
  \phi_\omega^u(U)=\frac{1}{\sqrt{4\pi\omega}}e^{-i\omega u(U)}, 
\end{equation}

entonces
\begin{equation}
  \phi_U(U) = \int_0^\infty d\omega [a_\omega^u \phi^u_\omega(U) +  a_\omega^{u*} \phi_\omega^u(U)^*].
\end{equation}

Expandiendo la nueva base en términos de la anterior
\begin{equation}
  \phi_\omega^u(U) = \int_0^\infty d\omega'\qty[ \alpha_{\omega\omega'} \phi_{\omega'}^U(U) 
  +\beta_{\omega\omega'} \phi_{\omega'}^U(U)^*].
\end{equation}

Los coeficientes $\alpha_{\omega\omega'}$ y $\beta_{\omega\omega'}$ se denominan coeficientes
de Bogoliubov y vienen dados por
\begin{gather}
  \alpha_{\omega\omega'} = -\frac{1}{2\pi}\sqrt{\frac{\omega'}{\omega}}\int_{-\infty}^\infty dU e^{-i(\omega u(U) -\omega' U)},\\
  \beta_{\omega\omega'} = -\frac{1}{2\pi}\sqrt{\frac{\omega'}{\omega}}\int_{-\infty}^\infty dU e^{-i(\omega u(U) +\omega' U)}.
\end{gather}

Cuantizando la teoría y formando el espacio de Fock, comprobamos que el vacío obtenido mediante
los modos $\phi^U_\omega (U)$, puede contener partículas asociadas a los modos $u$
si el coeficiente $\beta_{\omega\omega'}$ no es nulo
\begin{equation}
  \ev{0_U}{N_\omega^u}=\int_0^\infty d\omega' \abs{\beta_{\omega\omega'}}^2.
\end{equation}

Esto quiere decir que en una teoría cuántica de campos, el vacío depende de la base de modos
que se haya escogido antes de la cuantización.
La ambigüedad se puede resolver escogiendo el vacío que tenga la mínima energía.
En el espacio de Minkowski la energía está bien definida y coincide para todos los observadores
inerciales, al ser invariante de Lorentz.
Sin embargo, en un espacio-tiempo curvo el concepto de energía puede no estar bien definido
y por tanto no hay un estado de vacío privilegiado.

Con el fin de estudiar cuál es el vacío dado por un observador acelerado en un espacio de 
Minkowski, introducimos las coordenadas de Rindler $(\eta,\xi)$ definidas por
\begin{gather}
  t=\frac{1}{a} e^{a\xi} \sinh a\eta ,\\
  z=\frac{1}{a}e^{a\xi} \cosh a\eta,
  \label{eq:rindler}
\end{gather}
donde $\abs{t}<z$ y $a>0$.

Las coordenadas de Rindler solo cubren la región $\abs{t}<z$, denominada cuña derecha de Rindler.
De forma análoga, se puede cubrir la cuña izquierda de Rindler ($\abs{t}<-z$) mediante
las coordenadas $(\tilde \eta, \tilde \xi)$ dadas por
\begin{gather}
  t=-\frac{1}{a} e^{a\tilde \xi} \sinh a\tilde \eta \\
  z=-\frac{1}{a} e^{a\tilde \xi} \cosh a\tilde \eta.
\end{gather}

%Usar imagen libre, centrar
\begin{figure}[htb]
\includegraphics[width=0.8\textwidth]{rindler.png}\hfill
 \caption{Espacio de Minkowski en coordenadas de Rindler (usar imagen libre)
}                   
  \label{fig:rindler}
\end{figure}

Las trayectorias con $\eta=\eta_0$ constante corresponden con observadores que se mueven con
aceleración propia $ae^{-a\xi_0}$.
%\todo{Pq $\xi_0=0$}
El tiempo propio que miden estos observadores es $\tau = e^{a\xi_0}\eta$.
Escogiendo apropiadamente el origen de coordenadas, un observador con aceleración constante
viene dado por $\eta=\tau$ y $\xi=0$.

Fijándonos en el diagrama de Rindler (figura \ref{fig:rindler}),
un observador con aceleración constante moviéndose a la derecha no puede percibir los 
efectos producidos  en III ni influir en II. 
Además, la información que le llegue de II será percibida como proveniente de un tiempo
infinitamente anterior, por lo que la recta $t=z$ define un horizonte de sucesos futuro y 
la recta $t=-z$ un horizonte de sucesos pasado.

Como el observador acelerado hacia la derecha desconoce el estado del campo en la 
cuña izquierda de Rindler, el estado que percibe no es puro, si no mixto.
La descripción de estados mixtos se realiza mediante una matriz de densidad, que al trazar
sobre los estados desconocidos conduce a que el vacío sea un estado térmico.
Con el fin de hacer la deducción explícita, partimos de la ecuación de Klein-Gordon en
coordenadas de Rindler para la cuña derecha
\begin{equation}
  -\pdv[2]{\phi}{\eta}+\pdv[2]{\phi}{\xi} = 0.
  \label{eq:kgr}
\end{equation}

Definiendo las coordenadas nulas de Rindler 
  \begin{align}
    u&=\eta-\xi=-\frac{1}{a}\ln(-aU)\label{eq:coorind} ,\\
    v&=\eta+\xi=\frac{1}{a}\ln(aV), 
  \end{align}

la solución general de \ref{eq:kgr} es $\phi(u,v)=\phi_u(u)+\phi_v(v)$, que se expande en los
modos normales
\begin{equation}
  \begin{aligned}
    \phi^u_\omega(u) &=\frac{1}{\sqrt{4\pi\omega}}e^{-i\omega u},\\
    \phi^u_\omega(v) &=\frac{1}{\sqrt{4\pi\omega}}e^{-i\omega v}.
  \end{aligned}
\end{equation}

Aplicaríamos un tratamiento análogo a la cuña izquierda.

El campo propagándose hacia la derecha en coordenadas de Minkowski se expande como
\begin{equation}
  \begin{aligned}
    \phi_U(U)=&\int_0^\infty \Theta(-U)\qty[a^u_\omega \phi^u_\omega(u(U))+a_\omega^{u*}\phi_\omega^u(u(U))^*] \\
    &+\Theta(U)\qty[a^{\tilde u}_\omega \phi^{\tilde u}_\omega(\tilde u(U))+a_\omega^{\tilde u*}\phi_\omega^{\tilde u}(\tilde u(U))^*].
  \end{aligned}
\end{equation}

Donde $\Theta(U)$ es la función de Heaviside 
\begin{equation}
   \Theta(U)=
  \begin{cases}
    0\qquad \text{si }U<0 \\
    1\qquad \text{si }U>0
  \end{cases}
  .
\end{equation}

El cálculo de los coeficientes de Bogoliubov conduce a 
%\todo{Tal vez separar}
\begin{equation}
  \begin{aligned}
    &\alpha_{\omega,\omega'}^{ u} = -\frac{e^{\frac{\pi\omega}{2 a}}}{2\pi a}\sqrt{\frac{\omega}{\omega'}}\qty(\frac{\omega'}{a})^{-i\omega/a}
    \Gamma(i\omega/a), \quad
    \beta_{\omega,\omega'}^{ u} = \frac{e^{-\frac{\pi\omega}{2 a}}}{2\pi a}\sqrt{\frac{\omega}{\omega'}}\qty(\frac{\omega'}{a})^{-i\omega/a}
    \Gamma(i\omega/a)\\
    &\alpha_{\omega,\omega'}^{\tilde u} = -\frac{e^{\frac{\pi\omega}{2 a}}}{2\pi a}\sqrt{\frac{\omega}{\omega'}}\qty(\frac{\omega'}{a})^{i\omega/a}
    \Gamma(-i\omega/a), \quad
    \beta_{\omega,\omega'}^{\tilde u} = \frac{e^{-\frac{\pi\omega}{2 a}}}{2\pi a}\sqrt{\frac{\omega}{\omega'}}\qty(\frac{\omega'}{a})^{i\omega/a} 
    \Gamma(-i\omega/a)
  \end{aligned}
\end{equation}

Por conveniencia, los modos de Unruh se definen como
\begin{equation}
  \begin{aligned}
    \phi^I_\omega(U)=\Theta(-U)\phi^u_\omega(u(U)) + e^{-\frac{\pi\omega}{a}}\Theta(U)\phi_\omega^{\tilde u}(\tilde u(U))^*,\\
    \phi^{II}_\omega(U)=\Theta(U)\phi^{\tilde u}_\omega(\tilde u(U)) + e^{-\frac{\pi\omega}{a}}\Theta(-U)\phi_\omega^{u}(u(U))^*,
  \end{aligned}
\end{equation}
los cuales están definidos en todo el espacio de Minkowski.

La cuantización de los modos de Unruh conduce a los operadores de destrucción
\begin{equation}
  \begin{aligned}
    a_\omega^I = -2\sinh\frac{\pi\omega}{a}\int_0^\infty d\omega' \beta_{\omega\omega'}^{\tilde u} a_{\omega'}^U,\\
    a_\omega^{II} = -2\sinh\frac{\pi\omega}{a}\int_0^\infty d\omega' \beta_{\omega\omega'}^{u} a_{\omega'}^U.\\
  \end{aligned}
\end{equation}

El vacío de estos modos es el vacío de Minkowski, es decir
\begin{equation}
  a_\omega^I \ket{0_H} = a_\omega^{II}\ket{0_H} = 0.
\end{equation}
 
Para buscar el valor esperado de partículas que observa un observador acelerado
en el vacío de Minkowski, primero calculamos
\begin{equation}
  \ev{a_\omega^{u\dagger}a_\omega^u}{0_H} = \int_0^\infty d\omega'' \beta_{\omega\omega''}^u
  \beta_{\omega'\omega''}^{u*} = \frac{1}{e^{\frac{2\pi\omega}{a}}-1}\delta(\omega-\omega').
\end{equation}

En el límite $\omega'\to\omega$, esta cantidad es el valor esperado de partículas que se
detectarían, pero debido a la delta, se obtiene una divergencia.
El motivo es que calcular el número de partículas en toda la cuña derecha de Rindler, 
implica que hay que considerar un observador acelerado eternamente, lo que requiere infinita
energía.
En una situación real, la delta se transformaría en un valor finito.
Por tanto,
\begin{equation}
  \ev{N_\omega^u}{0_H} \sim \frac{1}{e^{\frac{2\pi\omega}{a}}-1}.
\end{equation}

Este valor esperado se corresponde con un sistema de bosones a temperatura $T=a/(2\pi k_B)$.
Sin embargo, todavía no conocemos la forma exacta del vacío de Minkowski.

En vez de considerar ondas armónicas con frecuencia bien definida, trabajamos con paquetes
de onda con resolución $\Delta \omega$ centrados en $(n+1/2)\Delta \omega$,
\begin{equation}
  \phi_{n\bar n}^u(u)=\frac{1}{\sqrt{\Delta \omega}}\int_{n\Delta\omega}^{(n+1)\Delta \omega}d\omega e^{i\frac{2\pi \bar n n}{\Delta \omega}\omega}
  \phi^u_\omega(u).
\end{equation}

Al cuantizar estos paquetes se introduce para la cuña derecha el operador destrucción $a_{n\bar n}^u$
y en la cuña izquierda $a_{n\bar n}^{\tilde u}$. 
Construyendo los modos de Unruh asociados a paquetes de ondas, se obtiene que
los operadores que los operadores $a_{n\bar n}^I$ y $a_{n\bar n}^{II}$ cumplen
\begin{equation}
  a_{n\bar n}^I\ket{0_U} = a_{n\bar n}^{II} \ket{0_U} = 0.
\end{equation}

De la expresión de $a_{n\bar n}^I$ y $a_{n\bar n}^{II}$ en términos de $a_{n\bar n}^u$ y $a_{n\bar n}^u$, se llega a 
\begin{equation}
  a_{n\bar n}^{u\dagger} a_{n\bar n}^u \ket{0_U} = a_{n\bar n}^{\tilde u\dagger}a_{n\bar n}^{\tilde u} \ket{0_U},
\end{equation}
y por tanto hay el mismo número de partículas asociadas a paquetes de ondas de Rindler en ambas
cuñas.

Teniendo en cuenta que el estado de vacío ha contener el mismo número de partículas en cada cuña,
\begin{equation}
  \ket{0_H} = \prod_{n,\bar n} \qty(\sum_{m=0}^\infty K_{nm} \ket{m_{n\bar n}}_u\otimes \ket{m_{n\bar n}}_{\bar u}).
\end{equation}
\todo{Es $\prod$ un producto tensorial/ suma directa?}

%\begin{equation}
%  K_{n,m+1}=e^{-\frac{\pi n\Delta \omega }{a}}K_{nm}.
%\end{equation}

%\begin{equation}
%  \ket{0_H} = \prod_{n,\bar n} \qty(\sum_{m=0}^\infty e^{-\frac{\pi n\Delta \omega }{a}}K_{nm}\ket{m_{n\bar n}}_u\otimes \ket{m_{n\bar n}}_{\bar u}).
%\end{equation}

Tras calcular los coeficientes $K_{n \bar n}$ y hacer la traza parcial sobre los estados de
la cuña izquierda, obtenemos que la matriz densidad
\begin{equation}
  \rho  \propto \prod_{n,\bar n} \qty(\sum_{m=0}^\infty e^{-2\pi m n\Delta  \omega/a}\ket{m_{n\bar n}}_u \bra{m_{n\bar n}}_u),
\end{equation}
que describe una distribución de Planck a temperatura $T_U=a/(2\pi k_B)$.

%\begin{equation}
%  \ev{0_U}{N^u_{n,\bar n}} =
%\end{equation}


\section{Radiación de Hawking}
La radiación de Hawking consiste en la emisión de partículas por un agujero negro.
Esta radiación sigue una distribución de cuerpo negro a la temperatura de Hawking
\begin{equation}
  T_{haw}= \frac{1}{8\pi k_B G M}.
\end{equation}

Trabajaremos con el caso más sencillo de agujero negro, el agujero negro de Schwarzschild, que
describe una distribución de masa con simetría esférica y estática. La métrica asociada
es 
\begin{equation}
  ds^2= -\qty(1-\frac{2MG}{r})dt^2+\qty(1-\frac{2MG}{r})^{-1}dr^2+r^2d\Omega^2.
\end{equation}

Observamos que en el radio de Schwarzschild $r_s=2MG$, la componente temporal de la 
métrica se anula mientras que la componente radial va a infinito, pero esta singularidad se debe
a la elección particular de coordenadas.
Se denomina horizonte de sucesos a la esfera de radio $r_s$ centrada en $r=0$.
Desde el punto de vista de la relatividad general, la  relevancia del horizonte de sucesos es
que la materia que lo atraviese no podrá salir, sino que se caerá inevitablemente en la singularidad en $r=0$.
Por otro lado, la dilatación temporal de la gravedad  hace que un observador fuera del agujero negro nunca
vea materia atravesar el horizonte, solo la verá moverse arbitrariamente cerca del horizonte.

Es conveniente introducir la coordenada tortuga
\begin{equation}
  r^*=r+2MG\ln\qty(\frac{r}{r_s}-1),
\end{equation}
que toma el valor $r^*=-\infty$ en el horizonte.
La métrica en estas coordenadas es
\begin{equation}
  ds^2= \qty(1-\frac{2MG}{r})[-dt^2+dr^{*2}]+r^2d\Omega^2.
\end{equation}

Si estudiamos un campo escalar no masivo con simetría esférica $\psi(t,r)$, la ecuación de Klein-Gordon
en un espacio de Schwarzschild es
\begin{equation}
  -\pdv[2]{\psi}{t} + \pdv[2]{\psi}{r^{*}} + \frac{2}{r}\qty(1-\frac{r_s}{r}) \pdv{\psi}{r^{*}} = 0.
\end{equation}

Para ver cómo difiere esta ecuación de la ecuación de ondas unidimensional en espacio de Minkowski, hacemos el cambio $\psi=\phi/r$,
\begin{equation}
  -\pdv[2]{\phi}{t} + \pdv[2]{\phi}{r^*} - \frac{r_s}{r^3}\qty(1-\frac{r_s}{r}) \phi= 0.
\end{equation}

Esta ecuación se corresponde con la ecuación de Klein-Gordon que vimos en \ref{eq:kg}, salvo 
por el término de potencial efectivo
\begin{equation}
  V(r) = \frac{r_s}{r^3}\qty(1-\frac{r_s}{r}).
\end{equation}

En una primera aproximación podemos ignorar este potencial, pues su único efecto es distorsionar
la radiación de Hawking emitida (\emph{backscattering}).

Definiendo las coordenadas nulas Eddington-Finkelstein
\begin{equation}
  \begin{aligned}
    u=t-r^*,\\
    v=t+r^*,
  \end{aligned}
\end{equation}
la solución general es
\begin{equation}
  \phi(u,v)=\phi_u(u) + \phi_v(v).
\end{equation}

El campo se descompone en los modos de Boulware
\begin{equation}
  \begin{aligned}
    \phi_\omega^u(u)=\frac{1}{\sqrt{4\pi \omega}}e^{-i\omega u}\\
    \phi_\omega^v(v)=\frac{1}{\sqrt{4\pi \omega}}e^{-i\omega v}.
  \end{aligned}
\end{equation}

El vacío asociado a la cuantización con estos modos se denomina vacío de Boulware y 
corresponde con el vacío que describe un observador en caída libre muy alejado de la fuente
gravitacional.

\subsection{Formación de un agujero negro}

La energía radiada por el agujero negro proviene de la materia que colapsa para formar el agujero negro.
Por tanto, tenemos que describir la formación de un agujero negro y ver cómo varía el estado de vacío
para un observador asintótico.
Suponemos que antes de formarse el agujero negro, la materia ocupa un volumen arbitrariamente grande,
de forma que la métrica fuera del objeto es aproximadamente la métrica de Minkowski y el vacío es el correspondiente
a cuantizar los modos
\begin{equation}
  \begin{gathered}
    \phi_\omega^U (U)=\frac{1}{\sqrt{4\pi\omega}} e^{-i\omega U},\\
    \phi_\omega^V (V)=\frac{1}{\sqrt{4\pi\omega}} e^{-i\omega V},\\
  \end{gathered}
\end{equation}
donde $U=t-r$ y $V=t+r$.

Una vez formado el agujero negro, todo el espacio viene descrito por la métrica de Schwarzschild y
el estado de vacío para observadores inerciales alejados del agujero negro es el vacío de Boulware.
Las partículas que emite el agujero negro y son detectadas por estos observadores se corresponden a los
modos salientes
\begin{equation}
  \phi^u_\omega(u) = \frac{1}{\sqrt{4\pi \omega}}e^{-i\omega u}.
\end{equation}

Tenemos que estudiar cómo se comportan los modos salientes al propagarlos al pasado, cuando todo
el espacio es de Minkowski.
Consideremos los modos que escapan la materia en colapso poco antes de formarse el horizonte.
Cerca del horizonte, si definimos la coordenadas de Kruskal-Szeckeres
\begin{equation}
  \begin{aligned}
    u'&=-4MGe^{-\frac{u}{4MG}},\\
    v'&=4MGe^{-\frac{v}{4MG}},
  \end{aligned}
\end{equation}
los modos salientes son
\begin{equation}
  \phi^{u}_\omega = \frac{1}{\sqrt{4\pi\omega}}e^{i\omega 4MG\ln \abs{\frac{u'}{4MG}}}.
\end{equation}

Vemos que al aproximarse al horizonte, $u'\to 0$, la frecuencia de estos modos tiende a infinito.
Por tanto, podemos despreciar la interacción con la materia en colapso.
La aparición de frecuencias arbitrariamente altas, las cuales deberían ser tratadas mediante
una teoría de gravedad cuántica, se denomina problema transplanckiano.

Como cerca del horizonte la frecuencias son muy altas, los modos se pueden aproximar por rayos (aproximación de la óptica geométrica).
Si definimos $V_H$ de forma que $V=V_H$ es la trayectoria del último rayo que pasa por $r=0$
y logra escapar justo antes de formarse el horizonte, podemos relacionar la coordenada
$u'$ con $V$.
El modo a la distancia nula $u'$ del horizonte se mantendrá en el pasado a la misma distancia de la trayectoria $V=V_H$.
Por tanto,
\begin{equation}
  \phi^{u}_\omega = \frac{1}{\sqrt{4\pi\omega}}e^{i\omega 4MG\ln \abs{A(V_H-V)}}.
\end{equation}

La constante $A$ aparece por la evolución del campo gravitatorio durante el colapso.
Entonces llegamos a la relación entre coordenadas de nulas de Eddington-Finkelstein futuras y de Minkowski pasadas,
que describen la propagación de un rayo que atraviesa el objeto en colapso,
\begin{equation}
  u'=-4MG\ln (A(V_H-V)).
\end{equation}
Esta ecuación es equivalente a la relación entre coordenadas nulas de Rindler y de Minkowski (\ref{eq:coorind}) si identificamos $a=\frac{1}{4MG}$, salvo por las constantes $A$ y $V_H$.
Escogiendo los orígenes de coordenadas de $u$ y $V$ apropiados, la equivalencia se cumple exactamente.
Haciendo un tratamiento análogo al del efecto Unruh, deduciríamos que para tiempo suficientemente tardíos,
los observadores situados en $r\to\infty$ medirían un espectro de radiación correspondiente a un 
cuerpo negro a la temperatura de Hawking,
\begin{equation}
  T_{haw}=\frac{1}{8\pi k_B G M}.
\end{equation}

En el siguiente capítulo, veremos cómo aparece esta misma temperatura al tratar el problema de Hagedorn
cerca de agujeros negros.

\subsection{Deducción alternativa}

La métrica de Schwarzschild, en coordenadas de Rindler, cerca del horizonte se aproxima por 
\begin{equation}
  ds^2= -\frac{\rho^2}{(4GM)^2}dt^2 + d\rho^2 +r^2 d\Omega^2
\end{equation}
donde $\rho = \sqrt{8GM (r-2GM)}$.

Si pasamos a un tiempo imaginario, $t\to \tau=it$, la métrica se convierte en
\begin{equation}
  ds^2= \frac{\rho^2}{(4GM)^2}d\tau^2 + d\rho^2 +r^2 d\Omega^2.
\end{equation}

Esta métrica se corresponde al espacio euclídeo descrito en coordenadas polares.
Para que la métrica no sea singular, el tiempo euclídeo debe tener periodo $\beta=8\pi G M$.



\chapter{Cuerdas en espacios curvos}

Este capítulo se centra en el estudio de las cuerdas en la proximidad de un agujero negro.
Como  la temperatura de Hagedorn se extrae del comportamiento divergente de la energía libre, 
necesitamos encontrar una forma de calcular la energía libre en un espacio-tiempo curvo.
Veremos dos métodos distintos: mediante la integral de camino toroidal de una cuerda
que se enrolla en un tiempo periódico y la integral de camino de un campo taquiónico.
Aplicando este último método a cuerdas de tipo II cerca de un agujero negro, encontraremos que
la divergencia de Hagedorn aparece a la temperatura de Hawking.
Concluiremos con la justificación de la entropía de un agujero negro a partir de los resultados anteriores.

\section{Integral de camino en la worldsheet}

Partimos de suponer que la integral de camino toroidal de una cuerda cerrada, con
un tiempo compactificado de periodo $\beta$, se relaciona con la energía libre de un gas de cuerdas mediante
\begin{equation}
  Z =  -\beta F.
\end{equation}

Esta relación se ha demostrado exacta para ciertos casos concretos, pero no hay una demostración general.

La integral de camino toroidal en un campo $G_{\mu\nu}$, el cual representa la curvatura
del espacio-tiempo, en $d$ dimensiones es
\begin{equation}
  Z=\int_0^\infty \frac{d\tau_2}{2\tau_2} \int_{-1/2}^{1/2} d\tau_1 \Delta_{FP} \int \mathcal DX
  \sqrt G e^{-S},
  \label{eq:toruspi}
\end{equation}
donde el determinante de Faddeev-Popov se introduce debido a la libertad gauge bajo difeomorfismos
y trasformaciones de Weyl.
La acción es 
\begin{equation}
  S= \frac{1}{4\pi \alpha'}\int_{[0,1]^2} d^2\sigma \sqrt h h^{\alpha\beta} \partial_\alpha X^\mu\partial_\beta X^\nu G_{\mu\nu}.
\end{equation}
Escogemos la métrica de la worldsheet como
\begin{equation}
  h_{\alpha\beta} =
\qty(
\begin{array}{cc}
  1 & \tau_1 \\
  \tau_1 & \tau_1^2+\tau_2^2
\end{array}
).
\end{equation}

La divergencia de Hagedorn aparece calcular la integral de camino de una que se enrolla en el tiempo 
a lo largo de la coordenada  $\sigma_2$, pues diverge en el límite $\tau_2\to0$.
Las condiciones de periodicidad son
\begin{equation}
  \begin{aligned}
    &X^\mu(\sigma_1+1,\sigma_2)=X^\mu(\sigma_1,\sigma_2),\\
    &X^i(\sigma_1,\sigma_2+1)=X^i(\sigma_1,\sigma_2),\\
    &X^0(\sigma_1,\sigma_2+1) = X^0(\sigma_1,\sigma_2)\pm \beta.
  \end{aligned}
\end{equation}

Haciendo la transformación modular $\tau\to-1/\tau$ e intercambiando $\sigma_1$ y $\sigma_2$, las
condiciones de periodicidad son
\begin{equation}
  \begin{aligned}
    &X^\mu(\sigma_1,\sigma_2+1)=X^\mu(\sigma_1,\sigma_2),\\
    &X^i(\sigma_1+1,\sigma_2)=X^i(\sigma_1,\sigma_2),\\
    &X^0(\sigma_1+1,\sigma_2) = X^0(\sigma_1,\sigma_2)\pm \beta.
  \end{aligned}
\end{equation}

Definimos las nuevas coordenadas $\sigma=\sigma_1/\tau_2$ y $\tau=\sigma_2$, de modo 
que la acción se convierte en 
\begin{equation}
  \begin{aligned}
    S= \frac{1}{4\pi \alpha'}\int_0^{1/\tau_2} d\sigma  \int_0^1 d\tau 
    &\biggl[
      \qty(1+\frac{\tau_1^2}{\tau^2_2})\partial_\sigma X^\mu \partial_\sigma X^\nu G_{\mu\nu} +
      2\frac{\tau_1}{\tau_2}\partial_\sigma X^\mu \partial_\tau X^\nu G_{\mu\nu}  \\
      &+ \partial_\tau X^\mu \partial_\tau X^\nu G_{\mu\nu}\biggl].
  \end{aligned}
\end{equation}


Haciendo un desarrollo de Fourier de $X^\mu(\sigma,\tau)$ en la coordenada $\sigma$, obtenemos
\begin{equation}
  \begin{aligned}
    X^i(\sigma,\tau) =& \sum_{n=-\infty}^\infty X_n^i(\tau) e^{in2\pi \tau_2 \sigma}, \\
    X^0(\sigma,\tau) =& \pm \beta \tau_2 \sigma +  \sum_{n=-\infty}^\infty X_n^0(\tau) e^{in2\pi \tau_2 \sigma}.
    \label{eq:four}
  \end{aligned}
\end{equation}

Definimos $X^i(\tau)=X^i_0(\tau)$ y $X^0(\tau)=X_0^0(\tau)$.
En la sección \ref{sec:free}, vimos que la divergencia de Hagedorn se produce al integrar en la región $\tau_2\to\infty$.
Por tanto, nos quedamos con la contribución dominante, $n=0$.
Al integrar en $\tau_2$, obtenemos la acción de una partícula
\begin{equation}
  \begin{aligned}
    S=\frac{1}{4\pi\alpha'\tau_2}\int_0^1 d\tau \big[
      \beta^2 (\tau_1^2+\tau_2^2)G_{00} \pm 2\beta\tau_1 G_{0\mu}\partial_\tau X^\mu  +G_{\mu\nu} \partial_\tau X^\mu \partial_\tau X^\nu
    \big].
  \end{aligned}
\end{equation}

Despreciar el resto de modos en la serie de Fourier hace que a la acción resultante le falten
términos adicionales. 
En una métrica plana, la corrección a la acción es
\begin{equation}
  \Delta S = - \frac{\tau_2^2 \beta^2_{H0}}{4\pi\alpha' \tau_2},
\end{equation}

siendo $\beta_{H0}$ la temperatura de Hagedorn en espacio plano.
Incluyendo esta corrección y definiendo un nuevo parámetro $t=\tau_2 \tau$ 
\begin{equation}
  \begin{aligned}
    S=\frac{1}{4\pi\alpha'}\biggl[ -\beta_{H0}^2\tau_2 + \int_0^{\tau_2} dt \biggl\{
      \beta^2 \frac{\tau_1^2+\tau_2^2}{\tau_2^2}G_{00} \pm 2\beta\frac{\tau_1}{\tau_2} G_{0\mu}\partial_t X^\mu  +G_{\mu\nu} \partial_t X^\mu \partial_t X^\nu
    \biggl\}\biggl].
  \end{aligned}
\end{equation}

En una métrica estacionaria y que cumpla $G_{0i}=0$
\begin{equation}
  \begin{aligned}
    S=\frac{1}{4\pi\alpha'}\biggl[ -\beta_{H0}^2\tau_2 + \int_0^{\tau_2} dt \biggl\{&
      \beta^2 \frac{\tau_1^2+\tau_2^2}{\tau_2^2}G_{00} \pm 2\beta\frac{\tau_1}{\tau_2} G_{00}\partial_t X^0  +G_{00} \qty(\partial_t X^0)^2 \\
    &+G_{ij} \partial_t X^i \partial_t X^j
    \biggl\}\biggl].
  \end{aligned}
\end{equation}

Si descomponemos la coordenada temporal en una parte clásica y una corrección adicional,
$X^0=X^{c}+\tilde X^0$, podemos hacer la integral de camino sólo sobre $\tilde X^0$. 
Primero hemos de buscar la acción clásica.
La ecuación de Euler-Lagrange es 
\begin{equation}
  \partial_t\qty[G_{00}\partial_t X^c \pm \beta \frac{\tau_1}{\tau_2}G_{00}]=0.
\end{equation}

Por lo que 
\begin{equation}
  G_{00}\partial_t X^c \pm \beta \frac{\tau_1}{\tau_2}G_{00}=C.
\end{equation}

Como la coordenada $X^0$ es periódica,
\begin{equation}
  0=X^0(\tau_2) -X^0(0) = X^0(t)\eval_0^{\tau_2}= \int_0^{\tau_2}\partial_t X^0(t) = \ev{\partial_t X^0} 
\end{equation}

Promediando,
\begin{equation}
  \ev{\partial_t X^c} \pm \ev{\beta \frac{\tau_1}{\tau_2}}=\ev{\frac{C}{G_{00}}} .
\end{equation}

Obtenemos la constante
\begin{equation}
  C = \pm \beta \frac{\tau_1}{\ev{G_{00}^{-1}}}
\end{equation}

La acción clásica es
\begin{equation}
  S^{cl}=\frac{1}{4\pi\alpha'}\ev{G_{00}(\partial_t X^0)^2 \pm 2\beta \frac{\tau_1}{\tau_2}G_{00}\partial_t X^0}.
\end{equation}

Aplicando 
\begin{equation}
  \ev{G_{00}(\partial_t X^0)^2 \pm \beta \frac{\tau_1}{\tau_2}G_{00}\partial_t X^0} = C\ev{\partial_t X^0} = 0,
\end{equation}
y
\begin{equation}
  \ev{G_{00}}\partial_t X^0 = \tau_2 C\mp \frac{\tau_1}{\tau_2}\beta \ev{G_{00}},
\end{equation}
la acción clásica puede escribirse como
\begin{equation}
  S^{cl}= \frac{1}{4\pi\alpha'} \qty[\frac{\tau_1^2\beta^2}{\ev{G_{00}^{-1}}}-\beta^2\frac{\tau_1^2}{\tau_2^2}\ev{G_{00}}].
\end{equation}

Como la trayectoria clásica hace estacionaria la acción, los términos lineales en $\tilde X^0$
se anulan y la integral de camino en $\tilde X^0$, tras varios cálculos, es
\begin{equation}
  Z_0=\frac{1}{\sqrt{4\pi^2\alpha'\ev{G_{00}^{-1}}}}\beta.
\end{equation}

La integral en $\tau_1$ en la ecuación \ref{eq:toruspi}, da lugar al término 
\begin{equation}
  \frac{2\pi\sqrt{\alpha'}\sqrt{\ev{G_{00}^{-1}}}}{\beta}.
\end{equation}

La integral de camino se reduce a 
\begin{equation}
  Z=\int_0^\infty \frac{d\tau_2}{\tau_2}\int \mathcal D \vec {X} \sqrt{\prod_t \det G_{ij}}e^{-Sp}
  \label{eq:final}
\end{equation}
donde la acción es
\begin{equation}
  S_p=\frac{1}{4\pi\alpha'}\qty[\int_0^{\tau_2}dt \qty(-\beta^2_{H0}+\beta^2 G_{00}+G_{ij}\partial_t X^i\partial_t X^j)].
\end{equation}

Hemos reducido la función de partición de una cuerda a la integral de camino de
una partícula no relativista moviéndose en un espacio curvo.

Aunque este método parece una opción viable, haremos los cálculos con un campo efectivo, razonando
que ambos procedimientos deberían conducir a los mismos resultados.

\section{Campo efectivo}
En la sección \ref{sec:taq} vimos que en un espacio plano, la divergencia de Hagedorn se manifiesta
como una campo que se vuelve taquiónico a temperaturas por encima de la temperatura de
Hagedorn.
Por tanto, intentaremos hallar la energía libre a partir de la acción de un campo taquiónico
en espacio curvo.

La acción de un campo taquiónico a orden más bajo en $\alpha'$ es
\begin{equation}
  S=\frac{1}{2}\int d^d x\sqrt G e^{-2\Phi}  \qty(G^{\mu\nu}\partial_\mu T\partial_\nu T+m^2T^2),
\end{equation}
donde $T$ es un campo escalar real y $\Phi$ el campo del dilatón.

Si el tiempo tiene periodo $2\pi R=\beta$, el desarrollo de Fourier de $T$ es
\begin{equation}
  T(x^0,x^i)=\sum_{n=-\infty}^{\infty} T_n(x^i)e^{\frac{inx^0}{R}}.
\end{equation}
Hay que tener en cuenta que $T_n(x^i)$ es un campo escalar complejo
y $n$ es el momento discretizado en la dimensión compacta.
Suponemos una métrica estacionaria y $G_{i0}=0$.
Sustituyendo,
\begin{equation}
  \begin{aligned}
    S=\sum_{n,m} \int d^{d-1} \sqrt{G}e^{-2\Phi}\biggl(&-\frac{nm}{R^2}G^{00}\partial_0 T_n \partial_0 T_{m} 
    + G^{ij}\partial_i T_n \partial_j T_{m} \\
    &+ m^2 T_n T_{m}\biggl) \int_{-\pi R}^{\pi R} \frac{dx^0}{2}e^{i\frac{n+m}{R}x^0}.
  \end{aligned}
\end{equation}

La integral en $x^0$ es $\pi R \delta_{-n,m}$, por lo que
\begin{equation}
  S=\sum_n\pi R \int d^{d-1} \sqrt{G}e^{-2\Phi}\qty(G^{ij}\partial_i T_n \partial_j T_{-n} + \frac{n^2 G^{00}}{R^2}T_n T_{-n} + m^2 T_n T_{-n}).
\end{equation}

Como $T$ es un campo real, $T_n=T^*_{-n}$ y por tanto
\begin{equation}
  S=\sum_n \pi R \int d^{d-1} \sqrt{G}e^{-2\Phi}\qty(G^{ij}\partial_i T_n \partial_j T_{n}^* + \frac{n^2 G^{00}}{R^2}T_n T_{n}^* + m^2 T_n T_{n}^*).
\end{equation}

Aunque en la acción solo aparece el momento $n$ y no los enrollamientos $w$, la acción completa debería satisfacer la dualidad T, 
la cual apareció en la sección \ref{sec:dual}.
Aprovechando la dualidad T, las  transformaciones necesarias para intercambiar el momento discreto por enrollamientos son
\begin{equation}
  \begin{aligned}
    G_{00}&\to \frac{1}{G_{00}},\\
    \Phi&\to \Phi -\frac{1}{2}\ln G_{00},\\
    T_n&\to T_w,\\
    R &\to \frac{\alpha'}{R}.
  \end{aligned}
\end{equation}

La acción en términos de enrollamientos del campo taquiónico es
\begin{equation}
  S \sim\sum_w  \int d^{d-1} x \sqrt G e^{-2\Phi}\qty(G^{ij}\partial_i T_w \partial_j T_{w}^* + \qty(\frac{w^2 R^2 G_{00}}{{\alpha'}^2} + m^2) T_w T_{w}^*) .
\end{equation}

Vemos que multiplicando a $\abs{T_w}^2$, aparece un término de masa efectiva $m^2_{loc} = m^2 +\frac{w^2 R^2 G_{00}}{{\alpha'}^2}$
que depende de la posición espacio-temporal.
\footnote{Haciendo el cálculo para cuerdas heteróticas, hay una corrección a esta fórmula.}

La contribución dominante de vendrá dada por $w=\pm 1$, por lo que definimos $T=T_1$.
Ignorando el campo del dilatón,
\begin{equation}
  S  =  \int d^{d-1} x \sqrt{G_{00}}\sqrt{G_{ij}} \qty(G^{ij}\partial_i T \partial_j T^* +m^2_{local} TT^*) .
\end{equation}

Aquí $\sqrt{G_{ij}}$ indica la raíz cuadrada del determinante de la parte espacial de la métrica.
Intentaremos interpretar esta acción como la de una partícula que se mueve en un espacio-tiempo con 
curvatura solo en la parte espacial y no en la temporal, para poder compararla con la acción obtenida
en la sección anterior.
Con este fin, hemos de eliminar el factor $\sqrt{G_{00}}$ del término cinético $G^{ij}\partial_i T \partial_j T^*$.

Integrando por partes la acción,
\begin{equation}
  S = \int d^{d-1} x \sqrt G e^{-2\Phi} T^* \qty[-\nabla^2-G^{ij}\frac{\partial_j \sqrt{G_{00}}}{\sqrt G_{00}}\partial_i + m^2_{loc}]T,
\end{equation}
donde $\nabla^2 = G^{ij}\nabla_i \partial_j$.
La integral de camino asociada a está acción se calcula buscando los
propios del operador 
\begin{equation}
  \widehat{O} =-\nabla^2-G^{ij}\frac{\partial_j \sqrt{G_{00}}}{\sqrt G_{00}}\partial_i + m^2_{loc}.
\end{equation}

Definiendo el producto escalar
\begin{equation}
  \braket{\psi_1}{\psi_2} = \int d^{d-1} x \sqrt G \psi_1(x)^* \psi_2(x),
\end{equation}
el operador $\widehat O$ es hermítico y por tanto posee un conjunto ortonormal de funciones
propias $\psi_n$ con valores propios reales, $\lambda_n$.

La integral de camino da como resultado
\begin{equation}
  Z=\int \mathcal D T e^{-S} =\prod_n \frac{1}{\lambda_n} = \det(\widehat O)^{-1}.
\end{equation}

La contribución taquiónica a la energía libre de un gas de cuerdas es
\begin{equation}
  F=-\frac{1}{\beta}\ln Z. 
\end{equation}

Teniendo en cuenta que $\Tr \ln X = \ln \det X$, la energía libre es
\begin{equation}
  F=\frac{1}{\beta} \Tr \ln \widehat O.
\end{equation}

Aplicando la fórmula de Schwinger del logaritmo, 
\begin{equation}
  \ln a = -\int_0^\infty \frac{dT}{T}\qty(e^{-aT}-e^{-T}),
\end{equation}
y despreciando el término $e^{-T}$, la energía libre es
\begin{equation}
  F=-\frac{1}{\beta}\int_0^\infty \frac{dT}{T} \Tr e^{-T\widehat O}.
  \label{eq:fO}
\end{equation}

Sigue habiendo una contribución de $\sqrt{G_{00}}$ al término cinético, que aparece al tomar la traza.
Para solucionarlo, hacemos el cambio de base
\begin{equation}
  \ket{\phi_n}=G_{00}^{1/4}  \ket{\psi_n}.
\end{equation}

El producto escalar asociado es
\begin{equation}
  \braket{\phi_1}{\phi_2} = \int d^{d-1} x \sqrt G_{ij} \phi_1(x)^* \phi_2(x).
\end{equation}

Tras el cambio de base,
\begin{equation}
  F=-\frac{1}{\beta}\int_0^\infty \frac{dT}{T} \Tr e^{-T\widehat D},
\end{equation}

donde aparece el operador hermítico
\begin{equation}
  \widehat D = -\nabla^2  + m_{loc}^2- \frac{3}{16}\frac{G^{ij} \partial_i G_{00}\partial_j G_{00}}{G_{00}^2}
  +\frac{\nabla^2 G_{00}}{4G_{00}}.
\end{equation}
Al tomar la traza, ya no aparece $\sqrt{G_{00}}$.
Los últimos dos términos se corresponden con un potencial efectivo
\begin{equation}
  K(x) = -\frac{3}{16}\frac{G^{ij} \partial_i G_{00}\partial_j G_{00}}{G_{00}^2}
  +\frac{\nabla^2 G_{00}}{4G_{00}}.
\end{equation}

Ahora se puede interpretar el término $\Tr e^{-T\widehat{D}}$ como una integral de camino
de una partícula 
\begin{equation}
Z=  \int_{S_1} \mathcal D x \sqrt{G_{ij}} e^{-\int_0^T dt \qty(\frac{1}{4}G_{ij}\dot x^i\dot x^j+m_{loc}^2+K(x))}.
\end{equation}

En resumen, haciendo el cambio de variable en la integral $t\to\pi \alpha't$, la energía libre es
\begin{equation}
  F=-\frac{1}{\beta}\int_0^\infty \frac{dT}{T} \int_{S^1} \mathcal D x\sqrt G_{ij}  e^{ -\frac{1}{4\pi\alpha'}\int_0^T dt \qty(  
  G_{ij}\dot x^i \dot x^j  + 4\pi^2{\alpha'}^2 (m_{loc}^2+K(x)))},
\end{equation}
donde
\begin{equation}
  \begin{aligned}
    m_{loc}^2 &= -\frac{4}{\alpha'}+\frac{\beta^2G_{00}}{4\pi^2 \alpha'^2} \quad\text{para cuerdas bosónicas},\\
    m_{loc}^2 &= -\frac{2}{\alpha'}+\frac{\beta^2G_{00}}{4\pi^2 \alpha'^2} \quad\text{para cuerdas de tipo II},\\
    m_{loc}^2 &= -\frac{3}{\alpha'}+\frac{\pi^2}{\beta^2 G_{00}}+\frac{\beta^2G_{00}}{4\pi^2 \alpha'^2}\quad\text{para cuerdas heteróticas}.
  \end{aligned}
\end{equation}

Si comparamos con la energía libre que obtendríamos a partir de la acción de la cuerda obtenida
mediante \ref{eq:final},
\begin{equation}
  F=-\frac{1}{\beta} \int_0^\infty \frac{d\tau_2}{\tau_2} \int_{S^1} \mathcal D X\sqrt G_{ij}  e^{ -\frac{1}{4\pi\alpha'}\int_0^{\tau_2} dt \qty(  
  G_{ij}\dot X^i \dot X^j  + \beta^2 G_{00} - \beta^2_{H0})},
\end{equation}
ambas energías libres coinciden al sustituir el valor apropiado de la temperatura de Hagedorn
en espacio plano, $\beta_{H0}$ salvo por el término $K(x)$. 
Cabe esperar que este término apareciese al tener en cuenta todos los términos del desarrollo de 
Fourier \ref{eq:four}.

Además, podrían estudiarse posibles correcciones a la acción del campo taquiónico a orden superior en $\alpha'$.

\section{Cuerdas cerca de agujeros negros}

Aplicaremos ahora el cálculo de la energía libre a un gas de cuerdas en las proximidades
de un agujero negro de Schwarzschild. 
Tomaremos como ejemplo la teoría supersimétrica de tipo II, se puede seguir un procedimiento
similar para cuerdas heteróticas y bosónicas, pero presentan más sutilezas.
Partimos de la expresión de la energía libre deducida a partir del campo taquiónico \ref{eq:fO},
\begin{equation}
  F=-\frac{1}{\beta}\int \frac{dT}{T} \Tr e^{-T\qty(  -\nabla^2-G^{ij}\frac{\partial_j \sqrt{G_{00}}}{\sqrt G_{00}}\partial_i + m^2_{loc}) }.
\end{equation}

Cerca del agujero negro, es conveniente emplear las coordenadas de Rindler, introducidas en \ref{eq:ri}.
El laplaciano en coordenadas de Rindler es
\begin{equation}
  \nabla^2  = \partial_\rho^2 + \frac{1}{\rho}\partial_\rho.
\end{equation}

La masa efectiva para cuerdas de tipo II es 
\begin{equation}
  m_{loc}^2=-\frac{2}{\alpha'}+\frac{\beta^2G_{00}}{4\pi^2\alpha'^2}.
\end{equation}

Para calcular la traza buscamos los valores propios del operador en el exponente,
pues la traza de un operador es la suma de sus valores propios.
Las funciones propias regulares (descartamos las que divergen) son del tipo
\begin{equation}
  \psi_n(\rho)\propto e^{-\frac{\beta \rho^2}{4\pi\alpha'^{3/2}} }
  L_n\qty(\frac{\beta\rho^2}{2\pi\alpha'^{3/2}}),
\end{equation}
donde $L_n$ es un polinomio de Laguerre de grado $n\geq0$.
Sus correspondientes valores propios son
\begin{equation}
  \lambda_n=\frac{\beta-2\pi\sqrt{\alpha'}+2\beta n }{\pi\alpha'^{3/2}}.
\end{equation}

La contribución dominante a la energía libre viene dada por el modo más bajo, $n=0$.
Esto no basta para calcular la traza pues tenemos que tener en cuenta las dimensiones adicionales al espacio de Rindler. 
En una teoría supersimétrica, tenemos ocho dimensiones descritas por una variedad que tenemos que especificar.
Suponiendo dimensiones planas, la energía libre es
\begin{equation}
  F= -\frac{V_T}{\beta}\int_0^\infty \frac{dT}{T}\qty(\frac{1}{4\pi T})^4 e^{\frac{\beta-2\pi\sqrt{\alpha'}}{\pi\alpha'^{3/2}}T}.
\end{equation}

En el límite superior de $T$, la energía libre diverge si
\begin{equation}
  \beta \leq 2\pi\sqrt{\alpha'}=\beta_R.
\end{equation}

Concluimos que la temperatura de Hagedorn en un espacio de Rindler coincide con la temperatura
de Hawking.
Hemos supuesto cuerdas de tipo II y que las dimensiones adicionales son planas.
De haber escogido otras condiciones, la temperatura de Hagedorn podría recibir correcciones.

Identificando las funciones propias como funciones de onda de una cuerda, el estado fundamental
\begin{equation}
  \psi_0(\rho)\propto e^{-\frac{\beta \rho^2}{4\pi\alpha'^{3/2}} },
  \label{eq:func}
\end{equation}
representa una cuerda localizada a la distancia $l_s$ del horizonte de sucesos.
Intuitivamente, las cuerdas sufren una presión radial que compensa la atracción gravitatoria del agujero negro
Este resultado refuerza la idea de la existencia de un horizonte \emph{estirado}, comentada en 
\cite{Susskind1993}.

\section{Entropía de agujeros negros}

La igualdad entre la temperatura de Hagedorn y la temperatura Hawking permite justificar
la entropía Bekenstein-Hawking de un agujero negro,
\begin{equation}
  S = \frac{A}{4G},
\end{equation}
donde $A$ es el área del agujero negro.
Pero antes, debemos entender si tiene sentido la existencia del equilibro termodinámico alrededor
de un agujero negro. 
En teoría cuántica de campos, si colocamos un agujero en una fuente de calor y partículas, 
la materia que cae al agujero negro entra en equilibrio con la radiación de Hawking.
En cambio, si mantenemos aislado un sistema formado por materia en caída y un agujero
negro, nunca se alcanzará el equilibrio.
La materia se iría aproximando arbitrariamente cerca del horizonte, ocupando un volumen muy
pequeño, de modo que cerca del horizonte se puede almacenar una cantidad infinita de información.
Esta el la causa de la divergencia de la entropía de un agujero negro en teoría cuántica de campos.

En teoría de cuerdas, el problema se soluciona al considerar que las cuerdas se mantienen a
una cierta distancia del horizonte, como refleja \ref{eq:func}.
De esta forma, las cuerdas alcanzan el equilibrio y al ocupar un volumen extenso, la entropía
permanece finita.

Para deducir la entropía de un agujero negro, calculamos el incremento de entropía asociado 
a lanzar cuerdas a un agujero negro, suponiendo que la interacción de la materia añadida con la
ya presente es pequeña. 
La densidad de estados de una cuerda es
$\omega(E) = E^\alpha e^{\beta_H E}$.

La entropía, en función del número de microestados $\Omega(E)$ es 
\begin{equation}
  S(E) = k_B\ln\Omega(E).
\end{equation}

Como el número de microestados en el intervalo $(E,E+dE)$ es $\Omega(E) = \omega(E)\delta E$,
\begin{equation}
  S(E) = k_B \ln \omega(E) + k_B \ln \delta E.
\end{equation}

El segundo término de la suma es una constante arbitraria que podemos ignorar, luego
la entropía en función de la densidad de estados es
\begin{equation}
  S(E) = k_B \ln \omega(E).
\end{equation}

A altas energías ($E \gg k_B T_H$) solo es relevante la contribución exponencial y por 
tanto el incremento de entropía del agujero negro es
\begin{equation}
  \delta S = k_B  \frac{\delta \omega}{\omega} = k_B \beta_H \delta E
\end{equation}

Teniendo en cuenta que $\beta_H = \beta_{haw}$, en términos de la masa,
\begin{equation}
  \delta S =  8\pi GM\delta M
\end{equation}

El radio de un agujero negro de Schwarzschild es $R=2GM$, lo que conduce a la entropía
de Bekenstein-Hawking,
\begin{equation}
  S=\frac{A}{4G}.
\end{equation}

Una interpretación alternativa de este resultado es que para que las cuerdas proporcionen
la entropía de un agujero negro, se ha de cumplir la igualdad $\beta_H = \beta_{haw}$.


\chapter{Conclusión}

La divergencia a la temperatura Hagedorn de la energía libre de cuerdas en espacio plano,
se puede extraer a partir de cuerdas en un estado taquiónico que se enrollan alrededor de
un tiempo periódico.
Asumiendo que esta correspondencia se mantiene en espacio curvo, haciendo la integral 
de camino de un campo taquiónico en espacio de Rindler se obtuvo que la temperatura de Hagedorn
coincide con la temperatura de Hawking.
Por otro lado, este resultado es consistente con la entropía predicha para agujeros negros.

Es interesante notar que los problemas de agujeros negros en gravedad cuántica suelen ser
no perturbativos, sin embargo los cálculos que hemos hecho se basan completamente en teoría de 
cuerdas perturbativa.
Sería conveniente aplicar un enfoque no perturbativo para tratar un gas de cuerdas en presencia de agujeros negros.

Por otra parte, se podría estudiar cómo se vincula la entropía de agujeros negros debida a cuerdas cercanas al horizonte con la deducción 
de Strominger y Vafa, basada en branas.

Algunos caminos de investigación posibles son las relaciones con el problema de la información, la hipótesis de firewalls y
otras propuestas como los fuzzballs.

De forma recíproca, entender la termodinámica de agujeros negros tal vez arroje luz sobre el 
comportamiento de las cuerdas altas temperaturas.



%\nocite{*}
\bibliographystyle{babunsrt}
\bibliography{mybib}
\end{document}
