\chapter{QFT en espacio curvo}

\section{Agujeros negros}
El caso más sencillo de agujero negro es el agujero negro de Schwarzschild, que
describe una distribución de masa con simetría esférica y estática. La métrica asociada
es 
\begin{equation}
  ds^2= \qty(1-\frac{2MG}{r})dt^2-\qty(1-\frac{2MG}{r})^{-1}dr^2-r^2d\Omega^2.
\end{equation}

Observamos que en el radio de Schwarzschild $r_S=2MG$ la componente temporal de la 
métrica se anula mientras que la componente radial va a infinito.
Se denomina horizonte de sucesos a la esfera con radio $r_S$ centrada en $r=0$.

Para estudiar la física cerca del horizonte es más conveniente emplear las coordenadas
de Rindler. Para ello sustituimos la coordenada radial por la distancia propia hasta
el horizonte
\begin{equation}
  \rho(r)=\int_0^r dr' \sqrt{g_{rr}(r')}.
\end{equation}

La métrica se transforma en 
\begin{equation}
  ds^2=\qty(1-\frac{2MG}{r(\rho)})dt^2-d\rho^2-r(\rho)^2d\Omega^2.
\end{equation}

A distancias próximas al horizonte $r\approx r_S$, $\rho\approx 2\sqrt{2MG(r-2MG)}$
\begin{equation}
  ds^2\approx \rho^2\qty(\frac{dt}{4MG})^2 -d\rho^2-r(\rho)^2d\Omega^2.
\end{equation}

Cerca de la coordenada azimutal $\theta=0$, podemos reemplzar las coordenadas
angulares $\theta,\phi$ por las coordenadas cartesianas $x,y$. Definiendo
también un tiempo adimensional $\omega=t/(4MG)$ obtenemos la métrica de Rindler
\begin{equation}
  ds^2=\rho^2d\omega^2 -d\rho^2 -dx^2-dy^2.
\end{equation}

Esta métrica describe en realidad un espacio de Minkowski en coordenadas hiperbólicas.
Si escogemos la transformación
\begin{align}
  T&=\rho \sinh \omega, \\
  Z&=\rho \cosh \omega.
\end{align}
la métrica es
\begin{equation}
  ds^2=dT^2-dZ^2-dx^2-dy^2.
\end{equation}

Es conveniente definir un observador fijo para cada punto del espacio, llamado
FIDO. Estos observadores para mantenerse
en reposo dentro del campo gravitatorio necesitan una fuerza que los mantenga
en su sitio. Para los FREFOs, su aceleración es, para $\rho<<MG$ es aproximadamente $1/\rho$.

Si dividimos el espacio de Minkowski en cuatro regiones, el espacio de Rindler ocupa
la región I. Los FREFOs emplean el las coordenadas del (T,Z,x,y), mientras que
los FREFOS emplean $(\omega, \rho, x, y)$. El horizonte se sitúa en $T=Z=0$, o
en coordenadas de Rindler, $\rho=0$.
Gráficamente, se observa que una translación espacial de $\omega$ para un equivale
a un boost en el espacio de Minkowski.

Al espacio de Rindler solo le llega información de I y IV, pero la región II está 
desconectada causalmente por culpa del horizonte.
Además, un partícula que pase de IV a I será vista como proveniente del tiempo 
$\omega=-\infty$, por lo que se pueden interpretar como condiciones iniciales.

Estudiamos un campo escalar masivo cuántico $\xi$ en un espacio de Rindler.
La evolución temporal de un sistema cualquiera viene dada por un hamiltoniano. En el
espacio de Rindler la evolución en $\omega$ se determina mediante el hamiltoniano de
Rindler, cuya expresión en función del tensor energía-momento es
\begin{equation}
  H_R=\int_0^\infty d\rho dxdy \rho T^{00} (\rho, x, y).
\end{equation}

En el caso de un campo escalar masivo sometido a un potencial $V$, la densidad de energía
es
\begin{equation}
  T^{00}=\frac{\Pi^2}{2}+\frac{1}{2}\qty(\nabla \xi)^2+V(\xi).
\end{equation}

Entonces el hamiltoniano es
\begin{equation}
  H_R=\int_0^\infty d\rho dx_\perp \frac{\rho}{2}\qty(\Pi^2+\qty(\pdv{\xi}{\rho})^2+\qty(\pdv{\xi}{x_\perp})^2+2V(\xi)).
\end{equation}

Desde el punto de vista del espacio de Minkowski, $H_R$ es el generador de boosts en 
la dirección Z.

En una teoría cuántica, el hecho que haya correlación entre el campo para dos puntos
del espacio origina un fenómeno interesante en el espacio de Rindler. 
Hemos visto que la región III está desconectada de la región I, por lo que consideramos
al campo en cada región como subsistemas distintos. Pero debido a que el campo en I está
correlacionado con el campo III, están entrelazados entre sí.
Por esta razón, no podemos describir al campo en I como un sistema puro como una
matriz de densidad obtenida tomando la traza parcial en III de la matriz de densidad
del sistema completo.

En general, si el estado asociado a dos subsistemas A y B que no están interactuando 
es $\ket{\Psi}$, su matriz de densidad es
\begin{equation}
  \rho_{AB}=\ket{\Psi}\bra{\Psi}.
\end{equation}

La matriz de densidad que describe el sistema A es
\begin{equation}
  \rho_{A}=\Tr_B \rho_{AB}=\expval{\rho_{AB}}{\beta}.
\end{equation}
 
A cada matriz de densidad le corresponde la entropía de von Neumann
\begin{equation}
  S=-\Tr (\rho_A \ln \rho_A)=-\sum_j \rho_j \ln \rho_j.
\end{equation}

Donde $\rho_j$ son los valores propios de $\rho_A$.
Esta entropía se debe a que estamos perdiendo información al ignorar el subsistema $B$, que 
está entrelazado con $A$, por lo que también se denomina entropía de entrelazamiento.
La intepretación de $\rho_j$ es que estamos describiendo un sistema cuyo estado no 
conocemos, pero sabemos que a cada estado le corresponde una probabilidad $\rho_j$.

Por otro lado un sistema termodinámico en equilibrio a temperatura $\beta$ tiene 
asociado la matriz densidad 
\begin{equation}
  \rho=\frac{e^{-\beta H}}{\Tr e^{\beta H}}.
\end{equation}

Retomamos el campo escalar en un espacio de Rindler. 
Para $T=0$, los campos en cada punto forman un conjunto completo de observables que 
conmutan. Denominamos $\xi_L$ a los campos en $Z<0$ y $\xi_R$ si $Z>0$.
En teoría cuántica de campos, un estado se describe como una distribución que 
depende de los campos $\Psi[\xi]=\Psi[\xi_L,\xi_R]$.

Supondremos que $\Psi[\xi_L,\xi_R]$ está en el estado fundamental (vacío) del hamiltoniano
de Minkowski. Para hallar el estado fundamental mediante la integral de camino 
en mecánica cuántica de partículas aplicamos
\begin{align}
  \braket{y,T}{x,0}\sim \int_{X(0)=x,X(T)=y} \mathcal D X e^{-S} \sim 
  \mel{y}{e^{-HT}}{x}=\sum_{n,n'} \ip{y}{n}\mel{n}{e^{-HT}}{n'}\ip{n'}{n}=\\
  =\sum_{n,n'} \Psi_n(y) e^{-E_n T} \bar{\Psi}_n(x)\sim   \Psi_0(y) e^{-E_0 T} \bar{\Psi}_0(x).
\end{align}

Donde hemos aplicado una rotación de Wick $t\to T=it$ por continuación analítica.
Por lo que $\bar{\Psi}_0(x)\sim \int_{X(0)=x} \mathcal D x e^{-S}$.
De forma similar en teoría cuántica de campos 
\begin{equation}
  \Psi[\xi_L,\xi_R]=\frac{1}{\sqrt{Z}}\int_{X^0>0, \xi(X^0=0)=(\xi_L,\xi_R)} e^{-S}
\end{equation}

Para evaluar la integral, tenemos en cuenta que como la translación en $\omega$
equivale a un boost en el espacio de Minkowski, en el espacio euclídeo se traduce
en una rotación.
El ángulo $\theta$ con respecto al eje $Z$ en el plano $(Z,X^0)$ se corresponde con
el tiempo de Rindler $\omega$.
